\hypertarget{the-infinite-horizon}{}
\section{The Infinite Horizon}\label{sec:the-infinite-horizon}

All of the solution methods presented so far have involved period-by-period iteration from an assumed last period of life, as is appropriate for life cycle problems.  However, if the parameter values for the problem satisfy certain conditions (detailed in \cite{BufferStockTheory}), the consumption rules (and the rest of the problem) will converge to a fixed rule as the horizon (remaining lifetime) gets large, as illustrated in Figure~\ref{fig:PlotCFuncsConverge}.  Furthermore, Deaton~\citeyearpar{deatonLiqConstr}, Carroll~\citeyearpar{carroll:brookings,carrollBSLCPIH} and others have argued that the `buffer-stock' saving behavior that emerges under some further restrictions on parameter values is a good approximation of the behavior of typical consumers over much of the lifetime.  Methods for finding the converged functions are therefore of interest, and are dealt with in this section.

Of course, the simplest such method is to solve the problem as
specified above for a large number of periods.  This is feasible, but
there are much faster methods.

\subsection{Convergence}

In solving an infinite-horizon problem, it is necessary to have some
metric that determines when to stop because a solution that is `good
enough' has been found.

A natural metric is defined by the unique `target' level of wealth that \cite{BufferStockTheory} proves
will exist in problems of this kind \href{https://llorracc.github.io/BufferStockTheory#GICNrm}{under certain conditions}: The $\mTrgNrm$ such that
\begin{equation}
  \Ex_t [{m}_{t+1}/m_t] = 1 \mbox{~if~} m_t = \mTrgNrm  \label{eq:mTrgNrmet}
\end{equation}
where the accent is meant to signify that this is the value
that other $m$'s `point to.'

Given a consumption rule $\cFunc(m)$ it is straightforward to find
the corresponding $\mTrgNrm$.  So for our problem, a solution is declared
to have converged if the following criterion is met:
$\left|\mTrgNrm_{t+1}-\mTrgNrm_{t}\right| < \epsilon$, where $\epsilon$ is
a very small number and defines our degree of convergence tolerance.

Similar criteria can obviously be specified for other problems.
However, it is always wise to plot successive function differences and
to experiment a bit with convergence criteria to verify that the
function has converged for all practical purposes.

\begin{comment} % at suggestion of WW, this section was removed as unnecessary for the current model, which solves for the converged rule very fast
  \subsection{The Last Period}

  For the last period of a finite-horizon lifetime, in the absence of a
  bequest motive it is obvious that the optimal policy is to spend
  everything.  However, in an infinite-horizon problem there is no last
  period, and the policy of spending everything is obviously very far
  from optimal.  Generally speaking, it is much better to start off with
  a `last-period' consumption rule and value function equal to those
  corresponding to the infinite-horizon solution to the perfect
  foresight problem (assuming such a solution is known).

  For the perfect foresight infinite horizon consumption problem,
  the solution is
  \begin{equation}\begin{gathered}\begin{aligned}
        \bar{\cFunc}(m_{t})  & = \overbrace{(1-R^{-1}(R
          \beta)^{1/\rho})}^{\equiv
          \underline{\kappa}}\left[{m}_{t}-1+\left(\frac{1}{1-1/R}\right)\right]
        \label{eq:pfinfhorc}
      \end{aligned}\end{gathered}\end{equation}
  where $\underline{\kappa}$ is the MPC in the
  infinite-horizon perfect foresight problem.  In our baseline problem,
  we set $\PermGroFac = \pLvl_{t} = 1$.  It is straightforward to show that the
  infinite-horizon perfect-foresight value function and marginal value
  function are given by
  \begin{equation}\begin{gathered}\begin{aligned}
        \bar{v}(m_{t})
        & =                                 \left(\frac{\bar{\cFunc}(m_{t})^{1-\rho}}{
            (1-\rho)\underline{\kappa} }\right)
        \\  \bar{v}^{m}(m_{t})  & =       (\bar{\cFunc}(m_{t}))^{-\rho}
        \\  \Opt{v}^{m}(a_{t})  & = \beta R \PermGroFac_{t+1}^{-\rho} \bar{v}^{m}(\RNrm_{t+1} a_{t}+1).
      \end{aligned}\end{gathered}\end{equation}

\end{comment}

\begin{comment}% At suggestion of WW this section was deleted because the technique is obvious and can be captured by the footnote that has been added
  \subsection{Coarse Then Fine \code{aVec} }

  The speed of each iteration is directly proportional to the number
  of gridpoints at which the problem must be solved.  Therefore
  reducing the number of points in \code{aVec} can increase
  the speed of solution greatly.  Of course, this also decreases the
  accuracy of the solution.  However, once the converged solution is
  obtained for a coarse \code{aVec}, the density of the grid
  can be increased and iteration can continue until a converged
  solution is found for the finer \code{aVec}.

  \subsection{Coarse then Fine \texttt{$\TranShkEmp$Vec}}

  The speed of solution is roughly proportionate\footnote{It is also
    true that the speed of each iteration is directly proportional to
    the number of gridpoints in \code{aVec}, at which the problem must
    be solved. However given our method of moderation, now the problem
    could be solved very precisely based on five gridpoints only. Hence
    we do not pursue the process of ``Coarse then Fine \code{aVec}.''}
  to the number of points used in approximating the distribution of
  shocks.  At least 3 gridpoints should probably be used as an initial
  minimum, and my experience is that increasing the number of gridpoints
  beyond 7 generally yields only very small changes in the solution.  The program
  \texttt{multiperiodCon\_infhor.m}
  begins with three gridpoints, and then solves for successively finer
  \texttt{$\TranShkEmp$Vec}.
\end{comment}
