% -*- mode: LaTeX; TeX-PDF-mode: t; -*- # Tell emacs the file type (for syntax)
% -*- mode: LaTeX; TeX-PDF-mode: t; -*- # Tell emacs the file type (for syntax)
% LaTeX path to the root directory of the current project, from the directory in which this file resides
% and path to econtexPaths which defines the rest of the paths like \FigDir
\providecommand{\econtexRoot}{}\renewcommand{\econtexRoot}{.}
\providecommand{\econtexPaths}{}\renewcommand{\econtexPaths}{\econtexRoot/Resources/econtexPaths}
% -*- mode: LaTeX; TeX-PDF-mode: t; -*- 
% The \commands below are required to allow sharing of the same base code via Github between TeXLive on a local machine and Overleaf (which is a proxy for "a standard distribution of LaTeX").  This is an ugly solution to the requirement that custom LaTeX packages be accessible, and that Overleaf prohibits symbolic links
\providecommand{\packages}{\econtexRoot/Resources/texmf-local/tex/latex}
\providecommand{\econtex}{\packages/econtex}
\providecommand{\econark}{\econtexRoot/Resources/texmf-local/tex/latex/econark}
\providecommand{\econtexSetup}{\econtexRoot/Resources/texmf-local/tex/latex/econtexSetup}
\providecommand{\econtexShortcuts}{\econtexRoot/Resources/texmf-local/tex/latex/econtexShortcuts}
\providecommand{\econtexBibMake}{\econtexRoot/Resources/texmf-local/tex/latex/econtexBibMake}
\providecommand{\econtexBibStyle}{\econtexRoot/Resources/texmf-local/bibtex/bst/econtex}
\providecommand{\econtexBib}{economics}
\providecommand{\notes}{\econtexRoot/Resources/texmf-local/tex/latex/handout}
\providecommand{\handoutSetup}{\econtexRoot/Resources/texmf-local/tex/latex/handoutSetup}
\providecommand{\handoutShortcuts}{\econtexRoot/Resources/texmf-local/tex/latex/handoutShortcuts}
\providecommand{\handoutBibMake}{\econtexRoot/Resources/texmf-local/tex/latex/handoutBibMake}
\providecommand{\handoutBibStyle}{\econtexRoot/Resources/texmf-local/bibtex/bst/handout}

\providecommand{\FigDir}{\econtexRoot/Figures}
\providecommand{\CodeDir}{\econtexRoot/Code}
\providecommand{\DataDir}{\econtexRoot/Data}
\providecommand{\SlideDir}{\econtexRoot/Slides}
\providecommand{\TableDir}{\econtexRoot/Tables}
\providecommand{\ApndxDir}{\econtexRoot/Appendices}

\providecommand{\ResourcesDir}{\econtexRoot/Resources}
\providecommand{\rootFromOut}{..} % APFach back to root directory from output-directory
\providecommand{\LaTeXGenerated}{\econtexRoot/LaTeX} % Put generated files in subdirectory
\providecommand{\econtexPaths}{\econtexRoot/Resources/econtexPaths}
\providecommand{\LaTeXInputs}{\econtexRoot/Resources/LaTeXInputs}
\providecommand{\LtxDir}{LaTeX/}
\providecommand{\EqDir}{\econtexRoot/Equations} % Put generated files in subdirectory

\providecommand{\local}{Resources/LaTeXInputs/local}

 % Set paths (like, \LaTeXInputs)
\newcommand{\texname}{SolvingMicroDSOPs}% Keyname for the paper
\documentclass[titlepage, headings=optiontotocandhead]{\econtex}

\usepackage{\local}% LaTeX config in Resources/LaTeXInputs
\usepackage{switches}% booleans that control whether certain features are on or off

\bibliographystyle{\econtexBibStyle}

\begin{document}

% Redefine \onlyinsubfile command defined in local.sty file:
% This lets any submaterial called from here know that it is not standalone
% If not called from here and IS standalone, can make bib (or other content)
\renewcommand{\onlyinsubfile}[1]{}\renewcommand{\notinsubfile}[1]{#1}

\hfill{\tiny \jobname, \today}

\begin{verbatimwrite}{./\texname.title}
  Solution Methods for Microeconomic Dynamic Stochastic Optimization Problems
\end{verbatimwrite}

\title{Solution Methods for Microeconomic \\ Dynamic Stochastic Optimization Problems}

\author{Christopher D. Carroll\authNum}

\keywords{Dynamic Stochastic Optimization, Method of Simulated Moments, Structural Estimation, Indirect Inference}
\jelclass{E21, F41}

\date{2024-03-27}
\maketitle
\footnotesize

\noindent  Note: The GitHub repo {\SMDSOPrepo} associated with this document contains python code that produces all results, from scratch, except for the last section on indirect inference.  The numerical results have been confirmed by showing that the answers that the raw python produces correspond to the answers produced by tools available in the \href{https://econ-ark.org}{Econ-ARK} toolkit, more specifically those in the {\HARKrepo} which has full {\HARKdocs}.  The MSM results at the end have have been superseded by tools in the {\EMDSOPrepo}.

\normalsize

\hypertarget{Abstract}{}
\begin{abstract}
  These notes describe tools for solving microeconomic dynamic stochastic optimization problems, and show how to use those tools for efficiently estimating a standard life cycle consumption/saving model using microeconomic data.  No attempt is made at a systematic overview of the many possible technical choices; instead, I present a specific set of methods that have proven useful in my own work (and explain why other popular methods, such as value function iteration, are a bad idea).  Paired with these notes is \textit{Mathematica}, Matlab, and Python software that solves the problems described in the text.
\end{abstract}

\begin{footnotesize}
  \begin{center}
    \begin{tabbing}
      \texttt{~~~~PDF:~} \= \= {\urlPDF} \\
      \texttt{~Slides:~} \> \> {\urlSlides} \\
      \texttt{~~~~Web:~} \> \> {\urlHTML} \\
      \texttt{~~~Code:~} \> \> {\urlCode} \\
      \texttt{Archive:~} \> \> {\urlRepo} \\
      \texttt{~~~~~~~~~} \> \> \textit{(Contains LaTeX code for this document and software producing figures and results)}
    \end{tabbing}
  \end{center}
\end{footnotesize}

\begin{authorsinfo}
  \name{Carroll: Department of Economics, Johns Hopkins University, Baltimore, MD, \\
    \href{mailto:ccarroll@jhu.edu}{\texttt{ccarroll@jhu.edu}}}
\end{authorsinfo}

\thanksFooter{The notes were originally written for my Advanced Topics in Macroeconomic Theory class at Johns Hopkins University; instructors elsewhere are welcome to use them for teaching purposes.  Relative to earlier drafts, this version incorporates several improvements related to new results in the paper \href{http://econ-ark.github.io/BufferStockTheory}{``Theoretical Foundations of Buffer Stock Saving''} (especially tools for approximating the consumption and value functions).  Like the last major draft, it also builds on material in ``The Method of Endogenous Gridpoints for Solving Dynamic Stochastic Optimization Problems'' published in \textit{Economics Letters}, available at \url{http://www.econ2.jhu.edu/people/ccarroll/EndogenousArchive.zip}, and by including sample code for a method of simulated moments estimation of the life cycle model \textit{a la} \cite{gpLifecycle} and Cagetti~\citeyearpar{cagettiWprofiles}.  Background derivations, notation, and related subjects are treated in my class notes for first year macro, available at \url{http://www.econ2.jhu.edu/people/ccarroll/public/lecturenotes/consumption}.  I am grateful to several generations of graduate students in helping me to refine these notes, to Marc Chan for help in updating the text and software to be consistent with \cite{carrollEGM}, to Kiichi Tokuoka for drafting the section on structural estimation, to Damiano Sandri for exceptionally insightful help in revising and updating the method of simulated moments estimation section, and to Weifeng Wu and Metin Uyanik for revising to be consistent with the `method of moderation' and other improvements.  All errors are my own.  This document can be cited as \cite{SolvingMicroDSOPs} in the references.}

\titlepagefinish
\setcounter{page}{1}

\ifpdf % For some reason, the table of contents does not work if not in pdf mode
  \tableofcontents \addtocontents{toc}{\vspace{1em}}\newpage
\fi

\hypertarget{Introduction}{}
\section{Introduction}

\begin{comment}
  Calculating the mathematically optimal amount to save is remarkably difficult.  Under well-founded assumptions about the nature of risk (and attitudes toward risk), the problem cannot be solved analytically; computational solutions are the only option.  To avoid having to solve this hard problem, past generations of economists showed impressive ingenuity in reformulating the question.  Budding graduate students are still taught a host of tricks whose purpose is partly to avoid the resort to numerical solutions: Quadratic or Constant Absolute Risk Aversion utility, perfect markets, perfect insurance, perfect foresight, the ``timeless perspective,'' the restriction of uncertainty to very special kinds,\footnote{E.g., lognormally distributed rate-of-return risk -- but no labor income risk -- under CRRA utility (the \cite{merton:restat}-\cite{samuelson:portfolio} model).} and more.

  The motivation for these reformulations is to exchange an intractable general problem
  for a tractable specific alternative.  Unfortunately, the burgeoning
  literature on numerical solutions has shown that the features that
  yield tractability also profoundly change the essence of the solution.  These tricks
  are excuses to solve a problem that has defined away the central
  difficulty: Understanding the proper role of uncertainty (and other
  complexities like constraints) in optimal intertemporal choice.


  These points are not unique to the consumption/saving problem; the
  same propositions apply to almost any question that involves both
  intertemporal choice and uncertainty, including many aspects of the
  behavior of firms and governments.
\end{comment}

These lecture notes provide a gentle introduction to a particular set
of solution tools for the canonical consumption-saving/portfolio allocation problem.
Specifically, the notes describe and solve optimization problems for a
consumer facing uninsurable idiosyncratic risk to nonfinancial income
(e.g., labor or transfer income), first without and then with optimal portfolio choice,\footnote{See
  \cite{merton:restat} and \cite{samuelson:portfolio} for a solution
  to the problem of a consumer whose only risk is rate-of-return risk
  on a financial asset; the combined case (both financial and
  nonfinancial risk) is solved below, and much more closely resembles
  the case with only nonfinancial risk than it does the case with only
  financial risk.}  with detailed intuitive discussion of the various
mathematical and computational techniques that, together, speed the
solution by many orders of magnitude compared to
``brute force'' methods.  The problem is solved with and without
liquidity constraints, and the infinite horizon solution is
obtained as the limit of the finite horizon solution.  After the basic
consumption/saving problem with a deterministic interest rate is
described and solved, an extension with portfolio choice between a
riskless and a risky asset is also solved.  Finally, a simple example
shows how to use these methods (via the statistical `method
of simulated moments' (`MSM') to estimate structural parameters like the
coefficient of relative risk aversion (\textit{a la} Gourinchas and
Parker~\citeyearpar{gpLifecycle} and
Cagetti~\citeyearpar{cagettiWprofiles}).  % The tricks and techniques used in solving these problems have broad applicability to many dynamic stochastic optimization problems.

\renewcommand{\DiscAlt}{\beta} % Erase the distinction between the alternative and the standard discount factor
\hypertarget{the-problem}{}
\section{The Problem}\label{sec:the-problem}

\begin{verbatimwrite}{./Excerpts/the-problem.texinput}
  The usual analysis of dynamic stochastic programming problems packs a great many events (intertemporal choice, stochastic shocks, intertemporal returns, income growth, the taking of expectations, and more) into a single step in which the agent makes an optimal choice taking account of all of these elements. For the detailed analysis here, we will be careful to disarticulate everything that happens in the problem explicitly into separate steps so that each element can be scrutinized and understood in isolation.

  We are interested in the behavior a consumer who begins {\interval} $t$ with a certain amount of `capital' $\kLvl_{t}$, which is immediately rewarded by a return factor $\Rfree_{t}$  with the proceeds deposited in a \textbf{b}ank account \textbf{b}alance:
  \begin{equation}\begin{gathered}\begin{aligned}
    \bLvl_{t} & = \kLvl_{t}\Rfree_{t}. \label{eq:bLvl}
  \end{aligned}\end{gathered}\end{equation}
\write18{if [ ! -f \texname.bib ]; then touch \texname.bib  ; fi}\write18{if [ ! -f \texname-Add.bib ]; then touch \texname-Add.bib  ; fi}\bibliography{economics,\texname,\texname-Add}\end{document}\endinput
   
Simultaneously with the realization of the capital return, the consumer also receives noncapital income $\yLvl_{t}$, which is determined by multiplying the consumer's `permanent income' $\pLvl_{t}$ by a transitory shock $\TranShkEmp_{t}$:
\begin{equation}\begin{gathered}\begin{aligned}
  \yLvl_{t} & = \pLvl_{t}\TranShkEmp_{t} \label{eq:yLvl}
\end{aligned}\end{gathered}\end{equation}
whose whose expectation is 1 (that is, before realization of the transitory shock, the consumer's expectation is that actual income will on average be equal to permanent income $\pLvl_{t}$).

The combination of bank balances $\bLvl$ and income $\yLvl$ define's the consumer's `market resources' (sometimes called `cash-on-hand,' following~\cite{deatonUnderstandingC}):
\begin{equation}\begin{gathered}\begin{aligned}
  \mLvl_{t} & = \bLvl_{t}+\yLvl_{t} \label{eq:mLvl},
\end{aligned}\end{gathered}\end{equation}
available to be spent on consumption $\cLvl_{t}$ for a consumer subject to a liquidity constraint that requires $\cLvl \leq \mLvl$.

The consumer's goal is to maximize discounted utility from consumption over the rest of a lifetime ending at date $T$:
\end{verbatimwrite}
\input{./Excerpts/the-problem.texinput}

\begin{verbatimwrite}{./Equations/MaxProb.tex}
  \begin{equation}\label{eq:MaxProb}
    \max ~ \Ex_{t}\left[ \sum_{n=0}^{T-t} {\DiscAlt}^{n} \uFunc(\cLvl_{t+n})\right].
  \end{equation}
\end{verbatimwrite}
  \begin{equation}\label{eq:MaxProb}
    \max~{\Ex}_{\prd}\left[\sum_{n=0}^{\trmT-\prd}{\DiscFac}^{n} \uFunc({\cLvl}_{\prd+n})\right].
  \end{equation}
\unskip

Income evolves according to: 
\begin{verbatimwrite}{./Equations/ExogVars.tex}
  \begin{equation}\begin{gathered}\begin{aligned}
        % \Rfree_{t}   & = \Rfree~\forall~t & \text{- constant interest factor = $1+\rfree$}
         \pLvl_{t+1}  = \PermGroFac_{t+1}\pLvl_{t}  &  \text{~~~ - permanent labor income dynamics} \label{eq:permincgrow}
        \\ \log ~ \TranShkEmp_{t+n} \sim ~\mathcal{N}(-\sigma_{\TranShkEmp}^{2}/2,\sigma_{\TranShkEmp}^{2}) & \text{~~~ - lognormal transitory shocks}~\forall~n>0 .
      \end{aligned}\end{gathered}\end{equation}
\end{verbatimwrite}
  \begin{equation}\begin{gathered}\begin{aligned}
        \pLvl_{\prd+1}   = \PermGroFac_{\prd+1}\pLvl_{\prd}                                        & \text{~~ -- permanent labor income dynamics} \label{eq:permincgrow}
        \\ \log ~ \tranShkEmp_{t+n}  \sim ~\Nrml(-\std_{\tranShkEmp}^{2}/2,\std_{\tranShkEmp}^{2}) & \text{~~ -- lognormal transitory shocks}~\forall~n>0 .
      \end{aligned}\end{gathered}\end{equation}
\unskip

Equation \eqref{eq:permincgrow} indicates that we are allowing for a predictable average profile of income growth over the lifetime $\{\PermGroFac\}_{0}^{T}$ (to capture typical career wage paths, pension arrangements, etc).\footnote{For expositional and pedagogical purposes, this equation assumes that there are no shocks to permanent income.  A large literature finds that, in reality, permanent (or at least extremely highly persistent) shocks exist and are quite large; such shocks therefore need to be incorporated into any `serious' model (that is, one that hopes to match and explain empirical data), but the treatment of permanent shocks clutters the exposition without adding much to the intuition, so permanent shocks are omitted from the analysis until the last section of the notes, which shows how to match the model with empirical micro data.  For a full treatment of the theory including permanent shocks, see \cite{BufferStockTheory}.}  %\ifthenelse{\boolean{MyNotes}}{\marginpar{\tiny However, note the $t$    subscript indicating that a life cycle profile is possible. It's    fairly easy to modify this to allow permanent shocks as well.}}{}
Finally, the utility function is of the Constant Relative Risk Aversion (CRRA), form, $\uFunc(\bullet) = \bullet^{1-\CRRA}/(1-\CRRA)$.

It is well known that this problem can be rewritten in recursive (Bellman) form
\begin{verbatimwrite}{./Equations/vrecurse.tex}
  \begin{equation}\begin{gathered}\begin{aligned}
       \vLvl
_{t}(\mLvl_{t},\pLvl_{t})  & = \max_{\cLvl_{t}}~ \uFunc(\cLvl_{t}) + {\DiscAlt}\Ex_{t}[\vLvl
_{t+1}(\mLvl_{t+1},\pLvl_{t+1})]\label{eq:vrecurse}
      \end{aligned}\end{gathered}\end{equation}
\end{verbatimwrite}
  \begin{equation}\begin{gathered}\begin{aligned}
        \vFunc_{\prd}(\mLvl_{\prd},\pLvl_{\prd})  & = \max_{\cNrm}~ \uFunc(\cNrm) + \DiscFac \Ex_{\prd}[ \vFunc_{\prd+1}({\mLvl}_{\prd+1},\pLvl_{\prd+1})]\label{eq:vrecurse}
      \end{aligned}\end{gathered}\end{equation}
\unskip
subject to the Dynamic Budget Constraint (DBC) implicitly defined by equations~\eqref{eq:bLvl}-\eqref{eq:mLvl}
and to the transition equation that defines next period's initial capital as this period's end-of-period assets:
\begin{equation}\begin{gathered}\begin{aligned}
      \kLvl_{t+1} & = \aLvl_{t}. \label{eq:transition.state}
\end{aligned}\end{gathered}\end{equation}

\hypertarget{Normalization}{}
\section{Normalization}\label{sec:normalization}
\ifthenelse{\boolean{MyNotes}}{\marginpar{\tiny Goal: Convert 2-state to 1-state problem; makes computational solutions much easier and easier to debug.}}{} The single most powerful method for speeding the solution of such models is to redefine the problem in a way that reduces the number of state variables (if at all possible).  Here, the obvious idea is to see whether the problem can be rewritten in terms of the ratio of various variables to permanent noncapital (`labor') income $\pLvl_{t}$ (henceforth for brevity, `permanent income.')


\ifthenelse{\boolean{MyNotes}}{\marginpar{\tiny Call this `curly V'.}}{}
In the last {\interval} of life, there is no
future, $\vLvl$
_{T+1} = 0$, so
the optimal plan is to consume everything:
\begin{equation}\begin{gathered}\begin{aligned}
     \vLvl
_{T}(\mLvl_{T},\pLvl_{T})  & = \frac{\mLvl_{T}^{1-\CRRA}}{1-\CRRA}. \label{eq:levelTm1}
    \end{aligned}\end{gathered}\end{equation}
Now define nonbold variables as the bold variable divided by
the level of permanent income in the same period, so that, for
example, $m_{T}=\mLvl_{T}/\pLvl_{T}$; and define
$\vFunc_{T}(m_{T}) = \uFunc(m_{T})$.\footnote{Nonbold value is bold value divided by $\pLvl^{1-\CRRA}$ rather than $\pLvl$.}  For our CRRA utility function, $\uFunc(xy)=x^{1-\CRRA}\uFunc(y)$, so (\ref{eq:levelTm1}) can be rewritten as
\begin{equation}\begin{gathered}\begin{aligned}
     \vLvl
      _{T}(\mLvl_{T},\pLvl_{T})  & = \pLvl_{T}^{1-\CRRA}\frac{{m}_{T}^{1-\CRRA}}{1-\CRRA}
      \\ &      = (\pLvl_{T-1}\PermGroFac_{T})^{1-\CRRA}\frac{{m}_{T}^{1-\CRRA}}{1-\CRRA}
      \\ &= \pLvl_{T-1}^{1-\CRRA}\PermGroFac_{T}^{1-\CRRA}\vFunc_{T}(m_{T}). \label{eq:vT}
    \end{aligned}\end{gathered}\end{equation}

Now define a new optimization problem:\ifthenelse{\boolean{MyNotes}}{\marginpar{\tiny Note that in mathematical terms, the $\PermGroFac_{t+1}^{1-\CRRA} {\DiscAlt}$ term is basically just a time-varying discount factor.}}{}
\begin{verbatimwrite}{./Equations/vNormed.tex}
  \begin{equation}\begin{gathered}\begin{aligned}
        \vFunc_{t}(m_{t}) & = \max_{{c}_{t}} ~~ \uFunc(c_{t})+
       \DiscFac\Ex_{t}[ \PermGroFac_{t+1}^{1-\CRRA}\vFunc_{t+1}(m_{t+1})] \label{eq:vNormed} \\
                                   & \text{s.t.}   \\
      a_{t}                      & = m_{t}-c_{t} \\
      k_{t+1}                    & = a_{t} \\
      b_{t+1}                    & = \underbrace{\left(\Rfree/\PermGroFac_{t+1}\right)}_{\equiv \RNrm_{t+1}}k_{t+1} \\
      m_{t+1}                    & = b_{t+1}+\TranShkEmp_{t+1},
      \end{aligned}\end{gathered}\end{equation}
\end{verbatimwrite}
  \begin{equation}\begin{gathered}\begin{aligned}
        {\vFunc}_{\prd}({m}_{\prd}) & = \max_{{c}_{\prd}} ~~ \uFunc({c}_{\prd})+
        {\DiscFac}\Ex_{\prd}[ \PermGroFac_{\prd+1}^{1-\CRRA}{\vFunc}_{\prd+1}({m}_{\prd+1})] \label{eq:vNormed}                   \\
                                         & \text{s.t.}                                                                                 \\
        {a}_{\prd}                       & = {m}_{\prd}-{c}_{\prd}                                                                     \\
        {k}_{\prd+1}                     & = {a}_{\prd}                                                                                \\
        {b}_{\prd+1}                     & = \underbrace{\left(\Rfree/\PermGroFac_{\prd+1}\right)}_{\equiv \RNrm_{\prd+1}}{k}_{\prd+1} \\
        {m}_{t+1}                        & = {b}_{t+1}+\TranShkEmp_{t+1},
      \end{aligned}\end{gathered}\end{equation}
\unskip
where the last equation is the normalized version of the transition equation for $\mLvl_{t+1}$.\footnote{Derivation:
  \begin{equation*}\begin{gathered}\begin{aligned}
        \mLvl_{t+1}/\pLvl_{t+1}              & = (\mLvl_{t}-\cLvl_{t})\Rfree/\pLvl_{t+1} + {\yLvl}_{t+1}/\pLvl_{t+1}
        \\      m_{t+1}                    & = \left(\frac{\mLvl_{t}}{\pLvl_{t}} - \frac{\cLvl_{t}}{\pLvl_{t}}\right)\Rfree\frac{\pLvl_{t}}{\pLvl_{t+1}} + \frac{{\yLvl}_{t+1}}{\pLvl_{t+1}}
        \\                                   & = \underbrace{(m_{t} - c_{t})}_{{a}_{t}}(\Rfree/\PermGroFac_{t+1}) + \TranShkEmp_{t+1}.
      \end{aligned}\end{gathered}\end{equation*}
}  Then it is easy to see that for $t=T-1$,
\begin{equation*}\begin{gathered}\begin{aligned}
      \vLvl_{T-1}(\mLvl_{T-1},\pLvl_{T-1}) & =  \pLvl_{T-1}^{1-\CRRA}\vFunc_{T-1}(m_{T-1})
    \end{aligned}\end{gathered}\end{equation*}
and so on back to all earlier periods.  Hence, if we solve the problem \eqref{eq:vNormed}
which (when optimal consumption is being chosen) has only a single state variable $\mNrm_{t}$, we can obtain the levels of the value function, consumption, and all other variables
from the corresponding permanent-income-normalized solution objects by multiplying each by $\pLvl_{t}$, e.g.\ ${\pmb{\cLvl}}
_{t}(\mLvl_{t},\pLvl_{t})=\pLvl_{t}\cFunc_{t}(\mLvl_{t}/\pLvl_{t})$ (or, for the value function, $\vLvl$
_{t}(\mLvl_{t},\pLvl_{t}) = \pLvl_{t}^{1-\CRRA}\vFunc_{t}(m_{t}))$.  We have thus reduced the
problem from two continuous state variables to one (and thereby enormously simplified its solution).

For some problems it will not be obvious that there is an appropriate `normalizing' variable, but many problems can be normalized if sufficient thought is given.  For example, \cite{valencia:2006} shows how a bank's optimization problem can be normalized by the level of the bank's productivity.

\ifthenelse{\boolean{MyNotes}}{\marginpar{\tiny Point out that right
    thing to normalize by will vary from problem to problem, e.g.
    Geiregat normalized banking problem by size of capital.}}{}

\hypertarget{The-Usual-Theory}{}
\section{The Usual Theory, and A Bit More Notation}
\label{sec:usualtheory}

\subsection{Steps}
%Since we have dissected the problem into individual steps, if we are to discuss each of the steps separately we need a notation to signal which step we are discussing.

%  The preceding analysis made some implicit assumptions about what was known when; difficulty intuiting or inferring those implicit assumptions is a common stumbling block for students trying to learn this material.

Generically, we want to think of the Bellman problem itself as having three {\moves}:
\begin{enumerate}
\item \textbf{\Arrival}: Incoming state variables (e.g., $\kNrm_{t}$) are known, but any shocks associated with the period have not been realized and decision(s) have not yet been made
\item \textbf{\Decision}: All exogenous variables (like income shocks, rate of return shocks, predictable income growth $\PermGroFac$ have been realized (so that, e.g., $\mNrm_{t}$'s value is known) and the agent solves the optimization problem
  \item \textbf{\Continuation}: After all decisions have been made, it is possible to calculate the consequences of the decision, taking as given the `outgoing' state variables (e.g., $\aNrm$) -- sometimes called `post-state' variables.
  \end{enumerate}

  The (implicit) default assumption is often to think of the step of the problem where the agent is solving a decision problem as defining the unique moment at which the problem is defined.  This is what implicitly was done above, when (for example) in \eqref{eq:vNormed} we related current value $\vFunc_{t}$ to the expectation of future value $\vFunc_{t+1}$.
  
  When we want to refer to a specific step within period $t$ we will do so by preceding it by an indicator character:
  \begin{center}
    \mbox{
  \begin{tabular}{r|c|l|l}
    Step     & Indicator & Usage & Explanation \\ \hline
  \Arrival    & ${\large \arvl}$ & $\vBeg({\kNrm}_{t})$ & value of entering $t$ (before shocks)
\\  \Decision & (blank/none) & $\vMid({\mNrm}_{t})$ & value of $t$-decision (after shocks)
\\  \Continuation & ${\large \cntn}$ & $\vEnd({\aNrm}_{t})$ & value of outgoing state (after decision)
  \end{tabular}
  }
\end{center}


This notation allows us to capture the fact that the value of the consumer's circumstances can be computed at any of the three steps (and is a function of a different state variable at each step).

Note that there is no need to use these subscripts for the model's variables; while a variable like $a_{t}=m_{t}-c_{t}$ takes on its value in the transition from the {\Decision} to the {\Continuation} step, $a$ will have only one unique value over the course of the period and therefore a notation like $\aNrm_{\BegStp}$ would be useless because the variable does not have a value until the continuation step is reached. Each variable in our problem has a unique value defined at some point during the period, so there is no ambiguity in referring to them with normal notation like $\aNrm_{t}$.

\subsection{The Usual Theory}

Using this new notation, the first order condition for \eqref{eq:vNormed} with respect to $c_{t}$ is
\begin{equation}\begin{gathered}\begin{aligned}
      \uFunc^{c}(c_{t})  & = \ExEndStp[\DiscFac \RNrm_{t+1}\PermGroFac_{t+1}^{1-\CRRA}{\vFunc}^{m}_{\MidStpNxt}(m_{t+1})]  \label{eq:upceqEvtp1}
      \\                        & =  \ExEndStp[\DiscFac\Rfree\phantom{._{t+1}}\PermGroFac_{t+1}^{\phantom{1}-\CRRA}{\vFunc}^{m}_{\MidStpNxt}(m_{t+1})]
    \end{aligned}\end{gathered}\end{equation}
and because the \handoutC{Envelope} theorem tells us that
\begin{equation}\begin{gathered}\begin{aligned}
     \vFunc^{m}_{\MidStp}(m_{t})  & =  \Ex_{\BegStp} [\DiscFac\Rfree\PermGroFac_{t+1}^{-\CRRA}{\vFunc}^{m}_{\MidStpNxt}(m_{t+1})] \label{eq:envelope}
    \end{aligned}\end{gathered}\end{equation}
we can substitute the LHS of \eqref{eq:envelope} for the RHS of
(\ref{eq:upceqEvtp1}) to get
\begin{verbatimwrite}{./Equations/Envelope.tex}
  \begin{equation}\begin{gathered}\begin{aligned}
        \uFunc^{c}(c_{t})  & =\vFunc^{m}_{\MidStp}(m_{t})\label{eq:Envelope}
      \end{aligned}\end{gathered}\end{equation}
\end{verbatimwrite}
  \begin{equation}\begin{gathered}\begin{aligned}
        \uFunc^{c}(c_{\prd})  & =\vFunc^{m}_{\MidStp}(m_{\prd})\label{eq:upcteqvtp}
      \end{aligned}\end{gathered}\end{equation}
\unskip
and rolling forward one {\interval},
\begin{equation}\begin{gathered}\begin{aligned}
      \uFunc^{c}(c_{t+1})  & =\vFunc^{m}_{\MidStpNxt}(a_{t}\RNrm_{t+1}+\TranShkEmp_{t+1}) \label{eq:upctp1EqVpxtp1}
    \end{aligned}\end{gathered}\end{equation}
and substituting the LHS in equation (\ref{eq:upceqEvtp1})
finally gives us the Euler equation for consumption:
\begin{verbatimwrite}{./Equations/cEuler.tex}
  \begin{equation}\begin{gathered}\begin{aligned}
        \uFunc^{c}(c_{t})  & = \ExEndStp[\DiscFac\Rfree \PermGroFac_{t+1}^{-\CRRA}\uFunc^{c}(c_{t+1})]. \label{eq:cEuler}
                                  \end{aligned}\end{gathered}\end{equation}
\end{verbatimwrite}
  \begin{equation}\begin{gathered}\begin{aligned}
        \uFunc^{c}(c_{\prd})  & = \ExEndStp[\DiscFac\Rfree \PermGroFac_{\prd+1}^{-\CRRA}\uFunc^{c}(c_{\prd+1})] \label{eq:cEuler}.
      \end{aligned}\end{gathered}\end{equation}
\unskip
%\input{./Excerpts/stages.texinput}

% \begin{verbatimwrite}{./Excerpts/stage-notation.texinput}
%   We need to define notation for these three {\moves} within a {\interval}. We will use $\arvl$ as a marker for the {\move} before shocks have been realized, $\dcsn$ to indicate the move in which the choice is made, and $\cntn$ as the indicator for the situation once the choice has been made.
% \end{verbatimwrite}
% \input{./Excerpts/stage-notation.texinput}
For future reference, it may be useful here to write the distinct value functions at each step (using the transition equations from \eqref{eq:vNormed}):
\begin{verbatimwrite}{./Equations/vBeg}
  \begin{equation}\begin{gathered}\begin{aligned}
\vBeg(\kNrm_{\stp}) & = \ExBegStp[\vMid(\overbrace{\kNrm_{\stp} \Rnorm_{\stp} + \TranShkEmp_{\stp}}^{=\mNrm_{\stp}})]  \label{eq:vBeg}
      \end{aligned}\end{gathered}\end{equation}
\end{verbatimwrite}
  \begin{equation}\begin{gathered}\begin{aligned}
        \vBeg(\kNrm_{\prd}) & = \ExBegStp[\vMid(\overbrace{\kNrm_{\prd} \Rnorm_{\prd} + \TranShkEmp_{\prd}}^{\mNrm_{\prd}})]  \label{eq:vBeg}
      \end{aligned}\end{gathered}\end{equation}
\unskip
\begin{verbatimwrite}{./Equations/vMid}
  \begin{equation}\begin{gathered}\begin{aligned}
\vMid(\mNrm_{\stp}) & = \uFunc(\cFunc_{\stp}(\mNrm_{\stp})) + \vEnd(\overbrace{\mNrm_{\stp}-\cNrm_{\stp}}^{\aNrm_{\stp}}) \label{eq:vMid}
      \end{aligned}\end{gathered}\end{equation}
\end{verbatimwrite}
  \begin{equation}\begin{gathered}\begin{aligned}
        \vFunc(\mNrm) & = \max_{\cNrm}~ \uFunc(\cNrm) + \vFunc_{_\cntn}(\overbrace{\mNrm-\cFunc}^{\aNrm}) \label{eq:vMid}
      \end{aligned}\end{gathered}\end{equation}
\unskip
and
\begin{verbatimwrite}{./Equations/vEndtdefn}
  \begin{equation}\begin{gathered}\begin{aligned}
\vEnd(\aNrm_{\stp}) & = \DiscFac \vBegStpNxt(\underbrace{\kNrm_{\stp+1}}_{=\aNrm_{\stp}}) \label{eq:vEndtdefn}
      \end{aligned}\end{gathered}\end{equation}
\end{verbatimwrite}
  \begin{equation}\begin{gathered}\begin{aligned}
        \vEndStp(\aNrm_{\prd}) & = \DiscFac \vBegStpNxt(\overbrace{\kNrm_{\prd+1}}^{\aNrm_{\prd}}) \label{eq:vEndtdefn}
      \end{aligned}\end{gathered}\end{equation}
\unskip
where the last line illustrates the notation for addressing the beginning {\move} of the successor {\interval}.

%Now note that in equation \eqref{eq:upctp1EqVpxtp1} neither $m_{\stp}$ nor $c_{\stp}$ has any \textit{direct} effect on $\vFunc_{\stp+1}$ - only the difference between them (i.e.\ unconsumed market resources or `assets' $a_t$) matters.  It is therefore possible (and will turn out to be convenient) to define a function\footnote{The peculiar letter designating our new function is pronounced `Gothic v'.  Letters in this font will be used for end-of-period quantities.}  \ifthenelse{\boolean{MyNotes}}{\marginpar{\tiny Define $\Alt{\vEnd}(a_{\stp})$.}}{}

%Putting all this together, we can write the `continuation value' function from the perspective of the end of {\interval} $\stp$ as
Putting all this together, from the perspective of the beginning of {\interval} $\stp+1$ we can write the `arrival value' function and its first derivative as
\begin{verbatimwrite}{./Equations/vBegtpdefn.tex}
  \begin{equation}\begin{gathered}\begin{aligned}
        \vBegStpNxt(k_{\stp+1})  & = \Ex_{\BegStpNxt}[\phantom{\Rfree}\PermGroFacAdjVNxt{\vFunc}_{\MidStpNxt}(\RNrm_{\stp+1}k_{\stp}+{\TranShkEmp}_{\stp+1})]  \label{eq:vFuncBegtpdefn}
\\  {\vPBegStpNxt}(k_{\stp+1})  & = \Ex_{\BegStpNxt}[\Rfree \PermGroFac_{\stp+1}^{\phantom{1}-\CRRA} \vFunc_{\MidStpNxt}^{m}(m_{\stp+1})]
  \end{aligned}\end{gathered}\end{equation}
\end{verbatimwrite}\unskip
  \begin{equation}\begin{gathered}\begin{aligned}
        \vBegStpNxt({k}_{\prd+1})    & = \Ex_{\BegStpNxt}[\phantom{\Rfree}\PermGroFacAdjVNxt{\vFunc}_{\MidStpNxt}(\overbrace{\RNrm_{\prd+1}{k}_{\prd+1}+{\TranShkEmp}_{\prd+1}}^{{m}_{\prd+1}})] \label{eq:vFuncBegtpdefn} \\
        \vPBegStgNxt({k}_{\prd+1}) & = \Ex_{\BegStpNxt}[\Rfree \PermGroFac_{\prd+1}^{\phantom{1}-\CRRA} {\vFunc}_{\MidStpNxt}^{{m}}({m}_{\prd+1})]
      \end{aligned}\end{gathered}\end{equation}
\unskip
because they return the expected $t+1$ value and marginal value associated with arriving in {\interval} $\stp+1$ with any given amount of \textbf{k}apital.  %Differentiating with respect to $\aNrm$, we get
%\begin{verbatimwrite}{./Equations/vEndtdefn.tex}  \begin{equation}\begin{gathered}\begin{aligned}    \vEnd(a_{\stp})  & = \DiscFac \vBegStpNxt_{\stp+1}(a_{\stp}) \label{eq:vEndtdefn}   \end{aligned}\end{gathered}\end{equation} \end{verbatimwrite}  \begin{equation}\begin{gathered}\begin{aligned}
        \vEndStp(\aNrm_{\prd}) & = \DiscFac \vBegStpNxt(\overbrace{\kNrm_{\prd+1}}^{\aNrm_{\prd}}) \label{eq:vEndtdefn}
      \end{aligned}\end{gathered}\end{equation}
\unskip
%Differentiating with respect to $\kNrm_{t+1}$, we get
% \begin{equation}\begin{gathered}\begin{aligned}
%   {\vPBegStpNxt}(k_{\stp})  & = \DiscFac \Ex_{\BegStpNxt}[\Rfree \PermGroFac_{\stp+1}^{-\CRRA} \vFunc_{\MidStpNxt}^{m}(m_{\stp})]
% %\\   {\vPEndStp}(a_{\stp})  & = \DiscFac \Ex_{\BegStpNxt}[\Rfree \PermGroFac_{\stp+1}^{-\CRRA} \vFunc_{\MidStpNxt}^{m}(m_{\stp})]                               
%     \end{aligned}\end{gathered}\end{equation}
% or, substituting from equation (\ref{eq:upctp1EqVpxtp1}) (and dropping function arguments for clarity),
% \begin{equation}\begin{gathered}\begin{aligned}
% %  \vPEndStp(a_{\stp})  & = \Ex_{\BegStpNxt}\left[\DiscFac \Rfree \PermGroFac_{\stp+1}^{-\CRRA} \uFunc^{c}\left(\cFunc_{\stp+1}(\RNrm_{\stp+1} a_{\stp}+{\TranShkEmp}_{\stp+1})\right)\right].  \label{eq:vEndPrimetOc}
%       \vPEndStp\astp  & = \DiscFac \Ex_{\BegStpNxt}\left[\Rfree \PermGroFac_{\stp+1}^{-\CRRA} \uFunc^{c}\left(\cFunc_{\stp+1}\mstp\right)\right] .  \label{eq:vEndPrimetOfc}                             
% \\                               & =  \DiscFac \Ex_{\BegStpNxt}[\vFunc^{k}_{\MidStpNxt}(k_{\stp})]
%     \end{aligned}\end{gathered}\end{equation}
Finally, note for future use that since $k_{t+1}={a}_{t}$ and $\vEnd(a_{t})=\DiscFac \vBegStpNxt({{k}_{t+1}})$, the first order condition 
(\ref{eq:upceqEvtp1}) can now be rewritten compactly as
\begin{verbatimwrite}{./Equations/upEqbetaOp.tex}
  \begin{equation}\begin{gathered}\begin{aligned}
        \uFunc^{c}(c_{\stp})   & = \vEnd^{a}(m_{\stp}-c_{\stp}).
        \label{eq:upEqbetaOp}
      \end{aligned}\end{gathered}\end{equation}
\end{verbatimwrite}
  \begin{equation}\begin{gathered}\begin{aligned}
        \uFunc^{{c}}({c}_{\prd})   & = \vEndStg^{{a}}({m}_{\prd}-{c}_{\prd}).
        \label{eq:upEqbetaOp}
      \end{aligned}\end{gathered}\end{equation}
\unskip

\begin{comment} %
\subsection{Implementation in Python}

The code implementing the tasks outlined each of the sections to come is available in the \texttt{\href{https://econ-ark.org/materials/SolvingMicroDSOPs}{SolvingMicroDSOPs}} jupyter notebook, written in \href{https://python.org}{Python}. The notebook imports various modules, including the standard \texttt{numpy} and \texttt{scipy} modules used for numerical methods in Python, as well as some user-defined modules designed to provide numerical solutions to the consumer's problem from the previous section. Before delving into the computational exercise, it is essential to touch on the practicality of these custom modules.

\subsubsection{Useful auxilliary files}

In this exercise, two primary user-defined modules are frequently imported and utilized. The first is the \texttt{gothic\_class} module, which contains functions describing the end-of-period value functions found in equations \eqref{eq:vBeg} - \eqref{eq:vEnd} (and the corresponding first and second derivatives). %The advantage of defining functions in the code which decompose the consumer's optimal behavior in a given period will become evident in section \ref{subsec:transformation}

The \texttt{resources} module is also used repeatedly throughout the notebook. This file has three primary objectives: (i) providing functions that discretize the continuous distributions from the theoretical model that describe the uncertainty a consumer faces, (ii) defining the utility function over consumption under a number of specifications, and (iii) enhancing the grid of end-of-period assets for which functions (such as those from the \texttt{gothic\_class} module) will be defined. These objectives will be discussed in greater detail and with respect to the numerical methods used to the problem in subsequent sections of this document.
\end{comment}


\hypertarget{Solving-the-Next-To-Last-Period}{}\section{Solving the Next-to-Last Period}\label{sec:NextToLast}

\provideboolean{HideG}
\setboolean{HideG}{true}
%\newcommand{\ifHideG}{\ifthenelse{\boolean{hideG}}
\renewcommand{\stp}{T}
\ifthenelse{\boolean{HideG}}{For convenience assuming that $\PermGroFac=1$ so that the $\PermGroFac$ terms disappear from the earlier derivations, t}{T}he problem in the second-to-last period of life can now be expressed as
\begin{equation*}\begin{gathered}\begin{aligned}
      \vFunc_{\stp-1}(m_{\stp-1})  & = \max_{{c}_{\stp-1}} ~~ \uFunc(c_{\stp-1}) +
      \DiscFac \Ex_{\EndStpLst} \left[\PermGroFacAdjV{\vFunc}_{\MidStp}(\underbrace{(m_{\stp-1}-c_{\stp-1})\RNrm_{\stp} + \TranShkEmp_{\stp}}_{{m}_{\stp}})\right],
\end{aligned}\end{gathered}\end{equation*}
where $\Ex_{\EndStpLst}$  indicates that the expectation is taken as of the end of period $\stp-1$.

Using (1) the fact that $\vFunc_{T}=\uFunc(c_{T})$; (2) the definition of $\uFunc(c_{T})$; (3) the
definition of the expectations operator, this becomes: \newcommand{\TranShkEmpDummy}{\vartheta}
\begin{equation*}\begin{gathered}\begin{aligned}
      \vFunc_{T-1}(m_{T-1})   & = \max_{{c}_{T-1}} ~~
      % \{
      \frac{{c}_{T-1}^{1-\CRRA}}{1-\CRRA} + \DiscFac \PermGroFacAdjV\int_{0}^{\infty}
      \frac{\left((m_{T-1}-c_{T-1})\RNrm_{T}+ \TranShkEmpDummy\right)^{1-\CRRA}}{1-\CRRA}
      d\FDist(\TranShkEmpDummy)
      % \}.
    \end{aligned}\end{gathered}\end{equation*}
where $\FDist$ is the cumulative distribution function for ${\TranShkEmp}_{T}$.

The maximization implicitly defines a function $\cFunc_{T-1}(m_{T-1})$ that yields optimal consumption in period $T-1$ for any specific numerical level of resources like $m_{T-1}=1.7$.  But because there is no general analytical solution to this problem, for any given $m_{T-1}$ we must use numerical computational tools to find the $c_{T-1}$ that maximizes the expression.  This is excruciatingly slow because for every potential $c_{T-1}$ to be considered, a definite integral over the interval $(0,\infty)$ must be calculated numerically, and numerical integration is \textit{very} slow (especially over an unbounded domain!).

\hypertarget{Discretizing-the-Distribution}{}
\subsection{Discretizing the Distribution}
Our first speedup trick is therefore to construct a discrete approximation to the lognormal distribution that can be used in place of numerical integration.  That is, we want to approximate the expectation over $\TranShkEmp$ of a function $g(\TranShkEmp)$ by calculating its value at set of $n_{\TranShkEmp}$ points $\TranShkEmp_{i}$, each of which has an associated probability weight $w_{i}$:
\begin{equation*}\begin{gathered}\begin{aligned}
      \Ex[g(\TranShkEmp)] & = \int_{\TranShkEmpMin}^{\TranShkEmpMax}(\TranShkEmpDummy)d\FDist(\TranShkEmpDummy) \\
      & \approx \sum_{\TranShkEmp = 1}^{n}w_{i}g(\TranShkEmp_{i})
    \end{aligned}\end{gathered}\end{equation*}
(because adding $n$ weighted values to each other is enormously faster than general-purpose numerical integration).

Such a procedure is called a `quadrature' method of integration; \cite{Tanaka2013-bc} survey a number of options, but for our purposes we choose the one which is easiest to understand: An `equiprobable' approximation (that is, one where each of the values of $\TranShkEmp_{i}$ has an equal probability, equal to $1/n_{\TranShkEmp}$).

We calculate such an $n$-point approximation as follows.

Define a set of points from $\sharp_{0}$ to $\sharp_{n_{\TranShkEmp}}$ on the $[0,1]$ interval
as the elements of the set $\sharp = \{0,1/n,2/n, \ldots,1\}$.\footnote{These points define intervals that constitute a partition of the domain of $\FDist$.}  Call the inverse of the $\TranShkEmp$ distribution $\FDist^{-1}_{\phantom{\TranShkEmp}}$, and define the
points $\sharp^{-1}_{i} = \FDist^{-1}_{\phantom{\TranShkEmp}}(\sharp_{i})$.  Then
the conditional mean of $\TranShkEmp$ in each of the intervals numbered 1 to $n$ is:
\begin{equation}\begin{gathered}\begin{aligned}
      \TranShkEmp_{i} \equiv \Ex[\TranShkEmp | \sharp_{i-1}^{-1} \leq \TranShkEmp < \sharp_{i}^{-1}]  & = \int_{\sharp^{-1}_{i-1}}^{\sharp^{-1}_{i}} \vartheta ~ d\FDist_{\phantom{\TranShkEmp}}(\vartheta)  ,
\end{aligned}\end{gathered}\end{equation}
and when the integral is evaluated numerically for each $i$ the result is a set of values of $\TranShkEmp$ that correspond to the mean value in each of the $n$ intervals.

The method is illustrated in Figure~\ref{fig:discreteapprox}.  The solid continuous curve represents
the ``true'' CDF $\FDist(\TranShkEmp)$ for a lognormal distribution such that $\Ex[\TranShkEmp] = 1$, $\sigma_{\TranShkEmp} = 0.1$.  The short vertical line segments represent the $n_{\TranShkEmp}$
equiprobable values of $\TranShkEmp_{i}$ which are used to approximate this
distribution.\footnote{More sophisticated approximation methods exist
  (e.g.\ Gauss-Hermite quadrature; see \cite{kopecky2010finite} for a discussion of other alternatives), but the method described here is easy to understand, quick to calculate, and has additional advantages briefly described in the discussion of simulation below.}
\begin{verbatimwrite}{\econtexRoot/Figures/discreteApprox.tex}
  \hypertarget{discreteApprox}{}
  \begin{figure}
    \includegraphics[width=0.8\textwidth]{./Figures/discreteApprox}
    \caption{Equiprobable Discrete Approximation to Lognormal Distribution $\FDist$}
    \label{fig:discreteapprox}
  \end{figure}
\end{verbatimwrite}
  \hypertarget{discreteApprox}{}
  \begin{figure}
    \includegraphics[width=0.8\textwidth]{\econtexRoot/Figures/discreteApprox}
    \caption{Equiprobable Discrete Approximation to Lognormal Distribution $\FDist$}
    \label{fig:discreteapprox}
  \end{figure}
\unskip
\write18{cat \econtexRoot/Figures/discreteApprox.tex >> \econtexRoot/Figures/SolvingMicroDSOPs-Figures-List.tex}
%\write18{if [ ! -f \texname.bib ]; then touch \texname.bib  ; fi}\write18{if [ ! -f \texname-Add.bib ]; then touch \texname-Add.bib  ; fi}\bibliography{economics,\texname,\texname-Add} \end{document}\endinput


\lstset{basicstyle=\ttfamily\footnotesize,breaklines=true,language=Python,frame=single}
\lstinputlisting{./Code/Python/snippets/equiprobable-make.py}

Because one of the purposes of these notes is to connect the math to the code that solves the math, we display here a brief snippet from the notebook that constructs these points:

Substituting into our definition of $\vEndStpLst$,
\begin{verbatimwrite}{./Equations/vDiscrete.tex}
  \begin{equation}\begin{gathered}\begin{aligned}
        \vFunc_{\EndStpLst}(a_{\stp-1})  & =   \DiscFac \PermGroFacAdjV\left(\frac{1}{n_{\TranShkEmp}}\right)\sum_{i=1}^{n_{\TranShkEmp}}   \frac{\left(\RNrm_{\stp} a_{\stp} + \TranShkEmp_{i}\right)^{1-\CRRA}}{1-\CRRA} \label{eq:vDiscrete}
      \end{aligned}\end{gathered}\end{equation}
\end{verbatimwrite}%\renewcommand{\stp}{t}
  \begin{equation}\begin{gathered}\begin{aligned}
        \vFunc_{{\prdLst}_\cntn}(\aNrm)  & =   \DiscFac \PermGroFacAdjV\left(\frac{1}{n_{\tranShkEmp}}\right)\sum_{i=1}^{n_{\tranShkEmp}}   \frac{\left(\RNrm_{\prd} \aNrm + \tranShkEmp_{i}\right)^{1-\CRRA}}{1-\CRRA} \label{eq:vDiscrete}
      \end{aligned}\end{gathered}\end{equation}
\unskip
so we can rewrite the maximization problem that defines the middle stage of the period {$\dcsn$} as
\begin{verbatimwrite}{./Equations/vEndTm1.tex}%\renewcommand{\stp}{T}
  \begin{equation}\begin{gathered}\begin{aligned}
        \vFunc_{\MidStpLst}(m_{\stp-1})   & = \max_{\cNrm_{\stp-1}}
        \left\{
          \frac{{c}_{\stp-1}^{1-\CRRA}}{1-\CRRA} +
          \vFunc_{\BegStpLst}(m_{\stp-1}-c_{\stp-1})
        \right\}.
        \label{eq:vEndTm1}
      \end{aligned}\end{gathered}\end{equation}%\renewcommand{\stp}{t}
\end{verbatimwrite}
\begin{equation}\begin{gathered}\begin{aligned}
  \vFunc_{\MidPrdLsT}(\mStte)  & = \max_{\cCtrl} ~~ \uFunc(\cCtrl) +
                              \vEndPrdLsT(\overbrace{\mStte-\cCtrl}^{\aStte})
                              \label{eq:vEndTm1}
\end{aligned}\end{gathered}\end{equation}
\unskip

\lstinputlisting{./Code/Python/snippets/equiprobable-max-using.py}

\begin{comment}
In the {\SMDSOPntbk} notebook, the section ``Discretization of the Income Shock Distribution'' provides code that instantiates the \texttt{DiscreteApproximation} class defined in the \texttt{resources} module. This class creates a 7-point discretization of the continuous log-normal distribution of transitory shocks to income by utilizing seven points, where the mean value is $-.5 \sigma^2$, and the standard deviation is $\sigma = .5$.

A close look at the \texttt{DiscreteApproximation} class and its subclasses should convince you that the code is simply a computational implementation of the mathematical description of equiprobable discrete approximation in this section. Moreover, the Python code generates a graph of the discretized distribution depicted in \ref{fig:discreteapprox}.
\end{comment}

\hypertarget{The-Approximate-Consumption-and-Value-Functions}{}
\subsection{The Approximate Consumption and Value Functions}

Given a particular value of $m_{T-1}$, a numerical maximization routine can now find the $c_{T-1}$ that maximizes \eqref{eq:vEndTm1} in a reasonable amount of time.

\begin{comment}

%The {\SMDSOPntbk} notebook follows a series of steps to achieve this. Initially, parameter values for the coefficient of relative risk aversion (CRRA, $\rho$), the discount factor ($\beta$), the permanent income growth factor ($\PermGroFac$), and the risk-free interest rate ($R$) are specified in ``Define Parameters, Grids, and the  Utility Function.''

After defining the utility function, the `natural borrowing constraint' is defined as $\underline{a}_{T-1}=-\underline{\TranShkEmp}\RNrm_{T}^{-1}$, which will be discussed in greater depth in section \ref{subsec:LiqConstrSelfImposed}. %Following the reformulation of the maximization problem, an instance of the \texttt{gothic\_class} is created using the specifications and the discretized distribution described in the prior lines of code; this is required to provide the numerical solution.
\end{comment}

The heart of the program responsible for computing an estimated consumption function begins in ``Solving the Model by Value Function Maximization,'' where a grid characterizing the possible values of market resources ($m$) is initialized (in the code, various $m$ vectors have names beginning {\mVec}), and for each of the {\mVec} values, the consumption values $c$ that solve the minimization problem equivalent to \eqref{eq:vEndTm1} are computed.  We arbitrarily pick the first five integers as our five {\mVec}  gridpoints.  (That is, \texttt{mVec\_int}= $\{0.,1.,2.,3.,4.\}$.


%Finally, the previously computed values of optimal $c$ and the grid of market resources are combined to generate a graph of the approximated consumption function for this specific instance of the problem. To reduce the computational challenge of solving the problem, the process is evaluated only at a small number of gridpoints.


\hypertarget{an-interpolated-consumption-function}{}
\subsection{An Interpolated Consumption Function} \label{subsec:LinInterp}

Given a set of points on a function (in this case, the consumption function $\cFunc_{T-1}(\mNrm)$), we can create a piecewise linear `interpolating function' (a `spline') which when applied to an input $m$ will yield the value of $\cNrm$ that corresponds to a linear `connect-the-dots' interpolation of the value of $\cNrm$ from the points, creating a function that is an approximation of the function whose points have been sampled.

This is accompished in ``An Interpolated Consumption Function,'' 
%"where the \texttt{InterpolatedUnivariateSpline} function is called from the \texttt{scipy} module to"
which defines an approximation to the consumption function $\Alt{\cFunc}_{\MidStp}(m_{T-1})$. That is, when called with an $m_{T-1}$ that is equal to one of the points in {{\mVec}\_int}, $\Alt{\cFunc}_{T-1}$ returns the associated value of $\vctr{c}_{\mathtt{T-1}}$, and when called with a value of $m_{T-1}$ that is not exactly equal to one of the \texttt{mVec\_int}, returns the value of $c$ that reflects a linear interpolation between the $\vctr{c}_{\mathtt{T-1}}$ points associated with the two \texttt{mVec\_int} points immediately above and below $m_{T-1}$.  %Thus if the function is called with $m_{T-1} = 1.75$ and the nearest gridpoints \ifthenelse{\boolean{MyNotes}}{\marginpar{\tiny Go to the optimal consumption figure and show the connect-the-dots; point out that it's more obvious for the $\vFunc_{T-1}$.}}{} are $m_{j,T-1}=1$ and $m_{k,T-1}=2$ then the value of $c_{T-1}$ returned by the function would be $(0.25 c_{j,T-1}+0.75 c_{k,T-1})$. We can define a numerical approximation to the value function $\Alt{\vFunc}_{T-1}(m_{T-1})$ in an exactly analogous way.


Figures \ref{fig:PlotcTm1Simple} and~\ref{fig:PlotVTm1Simple} show
plots of the constructed $\Alt{\cFunc}_{T-1}$ and $\Alt{\vFunc}_{T-1}$. While the $\Alt{\cFunc}_{T-1}$ function looks very smooth, the fact that the $\Alt{\vFunc}_{T-1}$ function is a set of line segments is very evident.  This figure provides the beginning of the intuition for why trying to approximate the value function directly is a bad idea (in this context).\footnote{For some problems,especially ones with discrete choices, value function approximation is unavoidable; nevertheless, even in such problems, the techniques sketched below can be very useful across much of the range over which the problem is defined.}

\hypertarget{PlotcTm1Simple}{}
\begin{figure}
  \centerline{\includegraphics[width=6in]{./Figures/PlotcTm1Simple}}
  \caption{$\cFunc_{T-1}(m_{T-1})$ (solid) versus $\Alt{\cFunc}_{T-1}(m_{T-1})$ (dashed)}
  \label{fig:PlotcTm1Simple}
\end{figure}

\hypertarget{PlotvTm1Simple}{}
\begin{figure}
  \centerline{\includegraphics[width=6in]{./Figures/PlotVTm1Simple}}
  \caption{$\vFunc_{T-1}$ (solid) versus $\Alt{\vFunc}_{T-1}(m_{T-1})$ (dashed)}
  \label{fig:PlotVTm1Simple}
\end{figure}

\hypertarget{Interpolating-Expectations}{}
\subsection{Interpolating Expectations}

\ifthenelse{\boolean{MyNotes}}{\marginpar{\tiny Good approximation in the sense that increasing the number of points makes no discernable difference.}}{}

Picewise linear `spline' interpolation as described above works well for generating a good approximation to the true optimal consumption function. However, there is a clear inefficiency in the program: Since it uses equation \eqref{eq:vEndTm1}, for every value of $m_{T-1}$ the program must calculate the utility consequences of various possible choices of $\cNrm_{T-1}$ as it searches for the best choice.

For any given value of $a_{T-1}$, notice that there is a good chance that the program may end up calculating the corresponding $\vMid$ many times while maximizing utility from different $m_{T-1}$'s.  For example, it is possible that the program will calculate the value of ending the period with $a_{T-1}=0$ dozens of times.  It would be much more efficient if the program could make the calculation of $\vMid(0)$ once and then merely recall the value when it is needed again.

Something like this can be achieved using the same interpolation technique used above
to construct a direct numerical approximation to the value function:
Define a vector of possible values for end-of-period assets at time $T-1$,
$\vctr{a}_{\mathtt{T-1}}$
(\texttt{aVec} in the code). %, designating the specific points $\vctr{a}\mathtt{[i]_{T-1}}$.
%For each of these values in $\vctr{a}_{\mathtt{T-1}}$,
Next, construct $\vctr{v}_{\mathtt{\MidStpLst}} =
\vFunc_{\MidStpLst}(\vctr{a}_{\mathtt{T-1}})$ using equation
(\ref{eq:vEndTm1}); then construct an
%\texttt{InterpolatingUnivariateSpline} object %
approximation $\Alt{\vFunc}_{\EndStpLst}(a_{T-1})$ by passing the lists \texttt{aVec} and \texttt{vVec} as arguments.
These lists contain the points of the $\vctr{a}_{\mathtt{\stp-1}}$ and $\vctr{\vFunc}_{\mathtt{\stp-1}}$ vectors, respectively.

As seen in the section ``Interpolating Expectations,'' we are now interpolating for the function
that reveals the expected value of \textit{ending} the period with a given amount
of assets.\footnote{What we are doing here is closely related to `the
  method of parameterized expectations' of
  \cite{denHaanMarcet:parameterized}; the only difference is that our
  method is essentially a nonparametric version.}  %The problem is solved in the same block with the remaining lines of code. 

Figure~\ref{fig:PlotOTm1RawVSInt}
compares the true value function to the
approximation produced following the interpolation procedure; the functions are of course
identical at the gridpoints of $a_{T-1}$ and they appear
reasonably close except in the region below
$m_{T-1}=1$.

\hypertarget{PlotOTm1RawVSInt}{}
\begin{figure}
  \centerline{\includegraphics[width=6in]{./Figures/PlotOTm1RawVSInt}}
  \caption{End-Of-Period Value $\vFunc_{\EndStpLst}(a_{T-1})$ (solid) versus $\Alt{\vFunc}_{\EndStpLst}(a_{T-1})$ (dashed)}
  \label{fig:PlotOTm1RawVSInt}
\end{figure}

\hypertarget{PlotComparecTm1AB}{}
\begin{figure}
  \centerline{\includegraphics[width=6in]{./Figures/PlotComparecTm1AB}}
  \caption{$\cFunc_{T-1}(m_{T-1})$ (solid) versus $\Alt{\cFunc}_{T-1}(m_{T-1})$ (dashed)}
  \label{fig:PlotComparecTm1AB}
\end{figure}

\Fix{\marginpar{\tiny In all figs, replace gothic h with notation corresponding to the lecture notes.}}

Nevertheless, the consumption rule obtained when the approximating $\Alt{\vFunc}_{\EndStpLst}(a_{T-1})$
is used\ifthenelse{\boolean{MyNotes}}{\marginpar{\tiny Don't skip the 2-3-3-4
    example in the text - it will be used again in a moment.}}{}
instead of $\vFunc_{\EndStpLst}(a_{T-1})$ is surprisingly bad, as
shown in figure \ref{fig:PlotComparecTm1AB}.  For example, when
$m_{T-1}$ goes from 2 to 3, $\Alt{\cFunc}_{T-1}$ goes from about 1
to about 2, yet when $m_{T-1}$ goes from 3 to 4, $\Alt{\cNrm}_{T-1}$
goes from about 2 to about 2.05.  The function fails even to be
strictly concave, which is distressing because Carroll and
Kimball~\citeyearpar{ckConcavity} prove that the correct
consumption function is strictly concave in a wide class of problems that
includes this problem.

\hypertarget{Value-Function-versus-First-Order-Condition}{}
\subsection{Value Function versus First Order Condition}\label{subsec:vVsuP}

Loosely speaking, our difficulty reflects the fact that the
consumption choice is governed by the \textit{marginal} value function,
not by the \textit{level} of the value function (which is the object that
we approximated).  To understand this point, recall that a quadratic
utility function\ifthenelse{\boolean{MyNotes}}{\marginpar{\tiny
    Intuitively speaking, if one's goal is to accurately capture
    behavior that is governed by marginal utility or the marginal
    value function, numerical techniques that approximate the \textit{
      marginal} value function are likely to work better.}}{} exhibits
risk aversion because with a stochastic $c$,
\begin{equation}
  \Ex[-(c - \cancel{c})^{2}] < - (\Ex[c] - \cancel{c})^{2}
\end{equation}
(where $\cancel{c}$ is the `bliss point' which is assumed always to exceed feasible $c$). However, unlike the CRRA utility function,
with quadratic utility the consumption/saving \textit{behavior} of consumers
is unaffected by risk since behavior is determined by the first order condition, which
depends on \textit{marginal} utility, and when utility is quadratic, marginal utility is unaffected
by risk:
\begin{equation}
  \Ex[-2(c - \cancel{c})] = - 2(\Ex[c] - \cancel{c}).
\end{equation}

Intuitively, if one's goal is to accurately capture choices
that are governed by marginal value,
numerical techniques that approximate the \textit{marginal} value
function will yield a more accurate approximation to
optimal behavior than techniques that approximate the \textit{level}
of the value function.

The first order condition of the maximization problem in period $T-1$ is:
\begin{verbatimwrite}{./Equations/FOCTm1.tex}
  \begin{equation}\begin{gathered}\begin{aligned}
        \uFunc^{c}(c_{T-1})       & = \DiscFac \Ex_{\cntn(T-1)} [\PermGroFacAdjMu\Rfree \uFunc^{c}(c_{T})]  %\label{eq:focraw}
        \\      c_{T-1}^{-\CRRA}   & = \Rfree \DiscFac \left(\frac{1}{n_{\TranShkEmp}}\right) \sum_{i=1}^{n_{\TranShkEmp}} \PermGroFacAdjMu\left(\Rfree (m_{T-1}-c_{T-1}) + \TranShkEmp_{i}\right)^{-\CRRA} \label{eq:FOCTm1}.
      \end{aligned}\end{gathered}\end{equation}
\end{verbatimwrite}
  \begin{equation}\begin{gathered}\begin{aligned}
        \uFunc^{c}(\cCtrl)       & = \DiscFac \Ex_{\cntn(T-1)} [\PermGroFacAdjMu\Rfree \uFunc^{c}(c_{\prdT})]  %\label{eq:focraw}
        \\      \cCtrl^{-\CRRA}   & = \Rfree \DiscFac \left(\frac{1}{n_{\TranShkEmp}}\right) \sum_{i=1}^{n_{\TranShkEmp}} \PermGroFacAdjMu\left(\Rfree (\mStte-\cCtrl) + \TranShkEmp_{i}\right)^{-\CRRA} \label{eq:FOCTm1}.
      \end{aligned}\end{gathered}\end{equation}
\unskip
\ifthenelse{\boolean{MyNotes}}{\marginpar{\tiny Go from the first to the last equation in \eqref{eq:FOCTm1} by substituting $\uFunc(c)=c^{-\CRRA}$ and use the approximation to the integral.}}{}
\hypertarget{PlotuPrimeVSOPrime}{}
\begin{figure}
  \centerline{\includegraphics[width=6in]{./Figures/PlotuPrimeVSOPrime}}
  \caption{$\uFunc^{c}(c)$ versus $\vFunc_{\EndStpLst}^{a}(3-c), \vFunc_{\EndStpLst}^{a}(4-c), \Alt{\vFunc}_{\EndStpLst}^{a}(3-c), \Alt{\vFunc}_{\EndStpLst}^{a}(4-c)$}
  \label{fig:PlotuPrimeVSOPrime}
\end{figure}

In the notebook, the ``Value Function versus the First Order Condition'' section completes the task of finding the values of consumption which satisfy the first order condition in \eqref{eq:FOCTm1} using the \href{https://docs.scipy.org/doc/scipy/reference/generated/scipy.optimize.brentq.html}{brentq} function from the \texttt{scipy} package. %Notice that the use of \texttt{u.prime} and \texttt{gothic.VP\_Tminus1} is possible since they are already defined in the \texttt{resources} and \texttt{gothic\_class} modules.

The downward-sloping curve in Figure \ref{fig:PlotuPrimeVSOPrime}
shows the value of $c_{T-1}^{-\CRRA}$ for our baseline parameter values
for $0 \leq c_{T-1} \leq 4$ (the horizontal axis).  The solid
upward-sloping curve shows the value of the RHS of (\ref{eq:FOCTm1})
as a function of $c_{T-1}$ under the assumption that $m_{T-1}=3$.
Constructing this figure is time-consuming, because for every
value of $c_{T-1}$ plotted we must calculate the RHS of
(\ref{eq:FOCTm1}).  The value of $c_{T-1}$ for which the RHS and LHS
of (\ref{eq:FOCTm1}) are equal is the optimal level of consumption
given that $m_{T-1}=3$, so the intersection of the downward-sloping
and the upward-sloping curves gives the (approximated) optimal value of $c_{T-1}$.
As we can see, the two curves intersect just below $c_{T-1}=2$.
Similarly, the upward-sloping dashed curve shows the expected value
of the RHS of (\ref{eq:FOCTm1}) under the assumption that $m_{T-1}=4$,
and the intersection of this curve with $\uFunc^{c}(c_{T-1})$ yields the
optimal level of consumption if $m_{T-1}=4$.  These two curves
intersect slightly below $c_{T-1}=2.5$.  Thus, increasing $m_{T-1}$
from 3 to 4 increases optimal consumption by about 0.5.

\ifthenelse{\boolean{MyNotes}}{\marginpar{\tiny Flip back to Figure
    4 to make the point that $\Alt{\vEnd}^{a}$ is a step
    function.}}{} Now consider the derivative of our function
$\Alt{\vFunc}_{\EndStpLst}(a_{T-1})$.  Because we have constructed
$\Alt{\vFunc}_{\EndStpLst}$ as a linear interpolation, the slope of
$\Alt{\vFunc}_{\EndStpLst}(a_{T-1})$ between any two adjacent
points $\{\vctr{a}[i],\vctr{},\vctr{a}[{i+1}]\}$ is constant.  The
level of the slope immediately below any particular gridpoint is
different, of course, from the slope above that gridpoint, a fact
which implies that the derivative of
$\Alt{\vFunc}_{\EndStpLst}(a_{T-1})$ follows a step function.

The solid-line step function in Figure \ref{fig:PlotuPrimeVSOPrime}
depicts the actual value of
$\Alt{\vFunc}_{\EndStpLst}^{a}(3-c_{T-1})$.  When we attempt to find
optimal values of $c_{T-1}$ given $m_{T-1}$ using
$\Alt{\vFunc}_{\EndStpLst}(a_{T-1})$, the numerical optimization
routine will return the $c_{T-1}$ for which $\uFunc^{c}(c_{T-1}) =
\Alt{\vFunc}^{a}_{\EndStpLst}(m_{T-1}-c_{T-1})$.  Thus, for
$m_{T-1}=3$ the program will return the value of $c_{T-1}$ for
which the downward-sloping $\uFunc^{c}(c_{T-1})$ curve intersects with the
$\Alt{\vFunc}_{\EndStpLst}^{a}(3-c_{T-1})$; as the diagram shows,
this value is exactly equal to 2.  Similarly, if we ask the routine
to find the optimal $c_{T-1}$ for $m_{T-1}=4$, it finds the point
of intersection of $\uFunc^{c}(c_{T-1})$ with
$\Alt{\vFunc}_{\EndStpLst}^{a}(4-c_{T-1})$; and as the diagram shows,
this intersection is only slightly above 2.  Hence, this figure
illustrates why the numerical consumption function plotted earlier
returned values very close to $c_{T-1}=2$ for both $m_{T-1}=3$ and
$m_{T-1}=4$.

We would obviously obtain much better estimates of the point of
intersection between $\uFunc^{c}(c_{T-1})$ and
$\vFunc_{\EndStpLst}^{a}(m_{T-1}-c_{T-1})$ if our estimate of
$\Alt{\vFunc}^{a}_{\EndStpLst}$ were not a step function.  In
fact, we already know how to construct linear interpolations
to functions, so the obvious next step is to construct a
linear interpolating approximation to the \textit{expected marginal
  value of end-of-period assets function} $\vFunc^{a}_{\EndStpLst}$:
\begin{equation}\begin{gathered}\begin{aligned}
      \vFunc_{\EndStpLst}^{a}(a_{T-1})  & =  \DiscFac \Rfree \PermGroFacAdjMu \left(\frac{1}{n_{\TranShkEmp}}\right) \sum_{i=1}^{n_{\TranShkEmp}} \left(\RNrm_{T} a_{T-1} + \TranShkEmp_{i}\right)^{-\CRRA} \label{eq:vEndPrimeTm1}
    \end{aligned}\end{gathered}\end{equation}
at the points in \texttt{aVec} yielding $\vctr{v}\mathtt{^{a}_{\EndStpLst}}$ (the vector of expected end-of-period-$(T-1)$ marginal values of assets corresponding to \texttt{aVec}),  %$\{\{\vctr{a}}\mathtt{_{T-1}},\vFunc_{\EndStpLst}^{a}(\vctr{{a}[1]}_{T-1}\},\{\vctr{a}_{(T-1)},\vFunc_{\EndStpLst}^{a}\}\ldots\}$
 and construct
$\Alt{\vFunc}_{\EndStpLst}^{a}(a_{T-1})$ as the linear
interpolating function that fits this set of points.

\hypertarget{PlotOPRawVSFOC}{}
\begin{figure}
  \centerline{\includegraphics[width=6in]{./Figures/PlotOPRawVSFOC}}
  \caption{$\vFunc_{\EndStpLst}^{a}(a_{T-1})$ versus $\Alt{\vFunc}_{\EndStpLst}^{a}(a_{T-1})$}
  \label{fig:PlotOPRawVSFOC}
\end{figure}


%This is done by making a call to the \texttt{InterpolatedUnivariateSpline} function, passing it \texttt{aVec} and \texttt{vpVec} as arguments. Note that in defining the list of values \texttt{vpVec}, we again make use of the predefined \texttt{gothic.VP\_Tminus1} function. These steps are the embodiment of equation~(\ref{eq:vEndPrimeTm1}), and construct the interpolation of the expected marginal value of end-of-period assets as described above.

The results are
shown in Figure \ref{fig:PlotOPRawVSFOC}.  The linear
interpolating approximation looks roughly as good (or bad) for the
\textit{marginal} value function as it was for the level of the value
function. However, Figure \ref{fig:PlotcTm1ABC} shows that the new
consumption function (long dashes) is a considerably better
approximation of the true consumption function (solid) than was the
consumption function obtained by approximating the level of the
value function (short dashes).

\hypertarget{PlotcTm1ABC}{}
\begin{figure}
  \centerline{\includegraphics[width=6in]{./Figures/PlotcTm1ABC}}
  \caption{$\cFunc_{T-1}(m_{T-1})$ (solid) Versus Two Methods for Constructing $\Alt{\cFunc}_{T-1}(m_{T-1})$}
  \label{fig:PlotcTm1ABC}
\end{figure}

\hypertarget{Transformation}{}
\subsection{Transformation}\label{subsec:transformation}

Even the new-and-improved consumption function diverges notably from the true
solution, especially at lower values of $m$.  That is because the
linear interpolation does an increasingly poor job of capturing the
nonlinearity of $\vFunc_{\EndStpLst}^{a}(a_{T-1})$ at
lower and lower levels of $a$.

This is where we unveil our next trick.  To understand the logic,
start by considering the case where $\RNrm_{T} = \DiscFac =
\PermGroFac_{T} = 1$ and there is no uncertainty
\ifthenelse{\boolean{MyNotes}}{\marginpar{\tiny Go over this
    carefully.}}{} (that is, we know for sure that income next period
will be $\TranShkEmp_{T} = 1$).  The final Euler equation is then:
\begin{equation}\begin{gathered}\begin{aligned}
      c_{T-1}^{-\CRRA}  & = c_{T}^{-\CRRA}.
    \end{aligned}\end{gathered}\end{equation}

In the case we are now considering with no uncertainty and no liquidity
constraints, the optimizing consumer does not care whether a unit of
income is scheduled to be received in the future period $T$ or the
current period $T-1$; there is perfect certainty that the income will
be received, so the consumer treats its PDV as equivalent to a unit of
current wealth.  Total resources available at the point when the consumption
decision is made is therefore are comprised of two types:
current market resources $m_{T-1}$ and `human wealth' (the PDV of
future income) of $\hNrm_{\EndStpLst}=1$ (because it is the value of human wealth as of the end of the period, and there is only one more period of income of 1 left).

The optimal solution is to spend half of total lifetime resources in
period $T-1$ and the remainder in period $T$.  Since total resources
are known with certainty to be
$m_{T-1}+\hNrm_{\MidStpLst}= m_{T-1}+1$, and since
$\vFunc_{\MidStp}^{m}(m_{T-1}) = \uFunc^{c}(c_{T-1})$ this
implies that \ifthenelse{\boolean{MyNotes}}{\marginpar{\tiny Crucial
    point: this is \textit{marginal} value function in period $T-1$,
    which we were trying to approximate with a linear interpolating
    function earlier.}}{}
\begin{equation}
  \vFunc^{m}_{\MidStpLst}(m_{T-1})  = \left(\frac{{m}_{T-1}+1}{2}\right)^{-\CRRA} \label{eq:vPLin}.
\end{equation}
Of course, this is a highly nonlinear
function.  However, if we raise both sides of \eqref{eq:vPLin} to the
power $(-1/\CRRA)$ the result is a linear function:
\begin{equation}\begin{gathered}\begin{aligned}
      [\vFunc^{m}_{\MidStpLst}(m_{T-1})]^{-1/\CRRA}  & = \frac{{m}_{T-1}+1}{2}  .
    \end{aligned}\end{gathered}\end{equation}
This is a specific example of a general phenomenon: A theoretical
literature discussed in~\cite{ckConcavity} establishes that under
perfect certainty, if the period-by-period marginal utility function
is of the form $c_{t}^{-\CRRA}$, the marginal value function will be
of the form $(\gamma m_{t}+\zeta)^{-\CRRA}$ for some constants
$\{\gamma,\zeta\}$.  This means that if we were solving the perfect
foresight problem numerically, we could always calculate a numerically
exact (because linear) interpolation.

To put the key insight in intuitive terms,
the problem we are facing springs in large part from the fact that the marginal value function is
highly nonlinear.  But we have a compelling solution to that problem,
because the nonlinearity springs largely from the fact that we are raising
something to the power $-\CRRA$.  In effect, we can `unwind' all of
the nonlinearity owing to that operation and the remaining
nonlinearity will not be nearly so great.  Specifically, applying the foregoing insights
to the end-of-period value function $\vFunc^{a}_{\MidStpLst}(a_{T-1})$, we can define
\begin{equation}\begin{gathered}\begin{aligned}
      \cFunc_{\EndStpLst}(a_{T-1})  & \equiv  \left(\vFunc^{a}_{\EndStpLst}(a_{T-1})\right)^{-1/\CRRA} \label{eq:cGoth}
    \end{aligned}\end{gathered}\end{equation}
which would be linear in the perfect foresight case.  Thus, our
procedure is to calculate the values of $\cFunc_{\stp-1}$ at each
of the $\vctr{a}_{\mathtt{\stp-1}}$ gridpoints, with the idea that we will construct
$\Alt{\cFunc}_{\EndStpLst}$ as the interpolating function connecting
these
points.


Note that this is \emph{not} a consumption function.  It is a `consumed' function - it reveals the amount that must have been consumed for the consumer to have arrived at the end of the period with a given amount of assets.  % A comparison of the result to the true consumed function is presented in figure~\ref{fig:GothVInvVSGothC}.  The two solutions are quite close.


\hypertarget{The-Natural-Borrowing-Constraint-and-the-a-Lower-Bound}{}
\subsection{The Natural Borrowing Constraint and the $a_{T-1}$ Lower Bound} \label{subsec:LiqConstrSelfImposed}


This is the appropriate moment to ask an awkward question we have so far
neglected: How should a function like $\Alt{\cFunc}_{\EndStpLst}$
be evaluated outside the range of points spanned by
$\{\mathtt{{a}_{T-1}[1],...,a_{T-1}[n]}\}$ for which we have calculated
the corresponding $\cFunc_{\EndStpLst}$ values used to produce our
linearly interpolating approximation $\Alt{\cFunc}_{\EndStpLst}$?

For most piecewise-linear interpolation implementations, when the interplating function
is evaluated at a point outside the provided range of values used to construct the
function, the algorithm silently performs extrapolation under the assumption that the
slope of the function remains the same beyond the measured boundaries as within the nearest
piecewise segment to the point.

\newcommand{\aVecMin}{a_{1}}\newcommand{\cMin}{c_{1}}
The easiest answer would be linear extrapolation; for example, if the bottommost gridpoint is $\aVecMin = \vctr{a}\mathtt{_{T-1}[1]}$ and the corresponding level of consumption is $\cMin = \cFunc_{\EndStpLst}(a_1)$ we could calculate the `marginal propensity to have consumed' $\varkappa_{1}=\Alt{\cFunc}_{\EndStpLst}^{a}(\aVecMin)$ and construct the approximation as the linear extrapolation below $\aVecMin$:
\begin{equation}\begin{gathered}\begin{aligned}
      \Alt{\cFunc}_{\EndStpLst}(a_{T-1})  &  \equiv \cMin + (a_{T-1}-\aVecMin)\varkappa_{1}  \label{eq:ExtrapLin}
    \end{aligned}\end{gathered}\end{equation}
for values of $a_{T-1} < \aVecMin$ %, where $\Alt{\cFunc}_{\EndStpLst}^{a}(\aVecMin)$ is the derivative of the $\Alt{\cFunc}_{\EndStpLst}$ function at the bottommost gridpoint.
To see that this approach will lead us into difficulties, consider what happens to the true (not approximated) $\vFunc^{a}_{\EndStpLst}(a_{T-1})$ as $a_{T-1}$ approaches the value
$\underline{a}_{T-1}=-\underline{\TranShkEmp}\RNrm_{T}^{-1}$.  From
\eqref{eq:vEndPrimeTm1} we have
\begin{equation}\begin{gathered}\begin{aligned}
      \lim_{{a}_{T-1} \downarrow \underline{a}_{T-1}} \vFunc^{a}_{\EndStpLst}(a_{T-1})
      & =                                                                                         \lim_{{a}_{T-1} \downarrow \underline{a}_{T-1}} \DiscFac \Rfree \PermGroFacAdjMu \left(\frac{1}{n_{\TranShkEmp}}\right) \sum_{i=1}^{n_{\TranShkEmp}} \left( a_{T-1} \RNrm_{T}+ \TranShkEmp_{i}\right)^{-\CRRA}.
    \end{aligned}\end{gathered}\end{equation}

\providecommand{\TranShkEmpMin}{}\renewcommand{\TranShkEmpMin}{\underline{\TranShkEmp}}
But since $\TranShkEmpMin=\TranShkEmp_{1}$, exactly at
$a_{T-1}=\underline{a}_{T-1}$ the first term in the summation would
be $(-\TranShkEmpMin+\TranShkEmp_{1})^{-\CRRA}=1/0^{\CRRA}$ which is
infinity.  The reason is simple: $-\underline{a}_{T-1}$ is
the PDV, as of $T-1$, of the minimum possible realization of income in
period $T$ ($\RNrm_{T}\underline{a}_{T-1} = -\TranShkEmp_{1}$).  Thus,
if the consumer borrows an amount greater than or equal to
$\underline{\TranShkEmp}\RNrm_{T}^{-1}$ (that is, if the consumer ends
$T-1$ with $a_{T-1} \leq -\underline{\TranShkEmp}\RNrm_{T}^{-1}$) and
then draws the worst possible income shock in period $T$, they will have
to consume zero in period $T$, which yields
$-\infty$ utility and $\infty$ marginal utility.

These reflections reveal that the consumer faces a
`self-imposed' (or `natural') borrowing constraint (which springs from the
precautionary motive): They will never borrow an amount greater
than or equal to $\underline{\TranShkEmp}\RNrm_{T}^{-1}$ (that is,
assets will never reach the lower bound of
$\underline{a}_{T-1}$).\footnote{Another term for a constraint of this
  kind is the `natural borrowing constraint.'}  The constraint is
`self-imposed' in the sense that if the utility function were
different (say, Constant Absolute Risk Aversion), the consumer would
be willing to borrow more than $\underline{\TranShkEmp}\RNrm_{T}^{-1}$
because a choice of zero or negative consumption in period $T$ would
yield some finite amount of utility.\footnote{Though it is very unclear what a
  proper economic interpretation of negative consumption might be --
  this is an important reason why CARA utility, like quadratic utility,
  is increasingly not used for serious quantitative work, though it is
  still useful for teaching purposes.}

This self-imposed constraint cannot be captured well when the
$\vFunc^{a}_{\EndStpLst}$ function is approximated by a piecewise
linear function like $\Alt{\vFunc}^{m}_{\EndStpLst}$, because there is no chance that the linear extrapolation below $\aMin$ will correctly predict $\vFunc^{a}_{\EndStpLst}(\underline{a}_{T-1})=\infty.$ To see what
will happen instead, note first that if we are approximating $\vFunc^{a}_{\EndStpLst}$ the smallest value in
\texttt{aVec} must be greater than $\underline{a}_{T-1}$
(because the expectation for any $a_{\stp-1} \leq \underline{a}_{T-1}$ is undefined).


Then when the
approximating $\vFunc^{a}_{\EndStpLst}$ function is evaluated at
some value less than the first element in \texttt{aVec}, the
approximating function will linearly extrapolate the slope that
characterized the lowest segment of the piecewise linear approximation
(between \texttt{aVec[1]} and \texttt{aVec[2]}), a
procedure that will return a positive finite number, even if the
requested $a_{T-1}$ point is below $\underline{a}_{T-1}$.  This means that the
precautionary saving motive is understated, and by an arbitrarily
large amount as the level of assets approaches its true theoretical
minimum $\underline{a}_{T-1}$.

The foregoing logic demonstrates that the marginal value of saving approaches infinity as $a_{T-1} \downarrow
\underline{a}_{T-1}=-\underline{\TranShkEmp}\RNrm_{T}^{-1}$.  But this
implies that $\lim_{{a}_{T-1} \downarrow \underline{a}_{T-1}}
\cFunc_{\EndStpLst}(a_{T-1}) = (\vFunc^{a}_{\EndStpLst}(a_{T-1}))^{-1/\CRRA} = 0$;
that is, as $a$ approaches its minimum possible value, the
corresponding amount of $c$ must approach \textit{its} minimum possible value: zero.

The upshot of this discussion is a realization that all we need to do is to
augment each of the $\vctr{a}_{T-1}$ and $\vctr{c}_{T-1}$ vectors with an extra point so that the
first element in the list used to produce our interpolation function is
$\{\underline{a}_{T-1},0.\}$. This is done in section
``The Self-Imposed `Natural' Borrowing Constraint and the $a_{T-1}$ Lower Bound'' of the notebook.%which can be seen in the defined lists \texttt{aVecBot} and \texttt{cVec3Bot}.

\hypertarget{GothVInvVSGothC}{}
\begin{figure}
  \centerline{\includegraphics[width=6in]{./Figures/GothVInvVSGothC}}
  \caption{True $\cFunc_{\EndStpLst}(a_{T-1})$ vs its approximation $\Alt{\cFunc}_{\EndStpLst}(a_{T-1})$}
  \label{fig:GothVInvVSGothC}
\end{figure}


From there, we plot the lists that have been prepended with the natural borrowing contraint and the associated minimal level of consumption. Figure\ifthenelse{\boolean{MyNotes}}{\marginpar{\tiny True $\cEndFunc$ is solid, linear approx is dashed.}}{} \ref{fig:GothVInvVSGothC} shows the result. The solid line calculates the exact numerical value of the consumed function $\cFunc_{\EndStpLst}(a_{T-1})$ while the dashed line is the linear interpolating approximation $\Alt{\cFunc}_{\EndStpLst}(a_{T-1}).$ This figure illustrates the value of the transformation: The true function is close to linear, and so the linear approximation is almost indistinguishable from the true function except at the very lowest values of $a_{T-1}$.

Figure~\ref{fig:GothVVSGothCInv} similarly shows that when we generate
$\Alt{\Alt{\vFunc}}_{\EndStpLst}^{a}(a)$ using our augmented 
$[\Alt{\cFunc}_{\EndStpLst}(a)]^{-\CRRA}$ (dashed line) we
obtain a \textit{much} closer approximation to the true function
$\vFunc^{a}_{\EndStpLst}(a)$ (solid line) than we did in
the previous program which did not do the
transformation (Figure~\ref{fig:PlotOPRawVSFOC}). (The vertical axis label uses $\mathfrak{v}^{\prime}$ as an alternative notation for what in these notes we designate as $\vFunc^{a}_{\EndStp}$).

\hypertarget{GothVVSGothCInv}{}
\begin{figure}
  \centerline{\includegraphics[width=6in]{./Figures/GothVVSGothCInv}}
  \caption{True $\vFunc^{a}_{\EndStpLst}(a_{T-1})$ vs. $\Alt{\Alt{\vFunc}}_{\EndStpLst}^{a}(a_{T-1})$ Constructed Using $\Alt{\cFunc}_{\EndStpLst}(a_{T-1})$}
  \label{fig:GothVVSGothCInv}
\end{figure}



\hypertarget{The-Method-of-Endogenous-Gridpoints}{}
\subsection{The Method of Endogenous Gridpoints}

Our solution procedure for $\cFunc_{T-1}$ still requires us, for each
point in $\vctr{m}\mathtt{_{T-1}}$ ({\mVec} in the code), to use a
numerical rootfinding algorithm to search for the value of $c_{T-1}$
that solves $\uFunc^{c}(c_{T-1}) =
\vFunc^{a}_{\EndStpLst}(m_{T-1}-c_{T-1})$.  Unfortunately, rootfinding
is a notoriously computation-intensive (that is, slow!) operation.

It turns out that there is a way to completely skip the rootfinding step.  The method can be understood by noting that any arbitrary value of $\vctr{a}\mathtt{_{T-1}}$ (greater than its lower bound value $\aVecMin$) will be associated with \textit{some} marginal valuation as of the continuation ($\cntn$) stage of $T-1$ (that is, at the end of the period), and the further observation that it is trivial to find the value of $c$ that yields the same marginal valuation, using the first order condition,
\begin{equation}\begin{gathered}\begin{aligned}
      \uFunc^{c}({\vctr{{\cNrm}}\mathtt{_{T-1}}})  & =
                                       \vFunc^{a}_{\EndStpLst}(\vctr{a}_{T-1}) \label{eq:eulerTm1}
\end{aligned}\end{gathered}\end{equation}
by using the inverse of the marginal utility function,
\begin{equation}\begin{gathered}\begin{aligned}
  c^{-\CRRA} & = \mu
  \\ c & = \mu^{-1/\CRRA}
    \end{aligned}\end{gathered}\end{equation}
which yields the level of consumption that corresponds to marginal utility of $\mu.$
Using this to invert both sides of \eqref{eq:eulerTm1}, we get
\begin{equation}\begin{gathered}\begin{aligned}
       {\vctr{{\cNrm}}\mathtt{_{T-1}}}  & = \left(\vFunc^{a}_{\EndStpLst}(\vctr{a}_{T-1})\right)^{-1/\CRRA}
%      \\  & = (\vFunc^{a}_{\EndStpLst}(a_{T-1,i}))^{-1/\CRRA}
%      \\  & \equiv  \cFunc_{\EndStpLst}(a_{T-1,i})
%      \\  & \equiv  \cFunc_{\EndStpLst,i}
    \end{aligned}\end{gathered}\end{equation}
where the $\cntn$ emphasizes that these are points on the `consumed' function (that is, the function that reveals how much an optimizing consumer must have consumed in order to end the period with $a_{T-1,i}$).

But with mutually consistent values of $\vctr{c}\mathtt{_{T-1}}$ and $\vctr{a}\mathtt{_{T-1}}$ (consistent, in the sense that they are the unique optimal
values that correspond to the solution to the problem), we can
obtain the $\vctr{m}\mathtt{_{T-1}}$ vector that corresponds to both of them from
\begin{equation}\begin{gathered}\begin{aligned}
      \vctr{m}\mathtt{_{T-1}}  & = {\vctr{{\cNrm}}\mathtt{_{T-1}}+\vctr{a}\mathtt{_{T-1}}}.
    \end{aligned}\end{gathered}\end{equation}

\ifthenelse{\boolean{ToFix}}{\marginpar{\tiny Rename gothic class: EndStp}}{}


These $m_{T-1}$ gridpoints are ``endogenous'' in contrast to the usual solution method of
specifying some \textit{ex-ante} grid of values of $m_{T-1}$ and then using a rootfinding
routine to locate the corresponding optimal $c_{T-1}$. This routine is performed in the ``Endogenous Gridpoints'' section of the notebook. First, the \texttt{gothic.C\_Tminus1} function is called for each of the pre-specfied
values of end-of-period assets stored in \texttt{aVec}. These values of consumption and assets
are used to produce the list of endogenous gridpoints, stored in the object
\texttt{mVec\_egm}. With the $\vctr{\cFunc}_{\EndStpLst}$ values in hand, the notebook can generate a set of $\vctr{m}\mathtt{_{T-1}}$ and ${\vctr{{\cNrm}}\mathtt{_{T-1}}}$
pairs that can be interpolated between in order to yield
$\Alt{\cFunc}_{\MidStpLst}(m_{\stp-1})$ at virtually zero computational cost!\footnote{This is the essential point of \cite{carrollEGM}.} %This is done in the final line of code in this block, and the following code block produces the graph of the interpolated consumption function using this procedure.

One might worry about whether the $\{{m},c\}$ points obtained in this way will provide a
good representation of the consumption function as a whole, but in
practice there are good reasons why they work well (basically, this
procedure generates a set of gridpoints that is naturally dense right
around the parts of the function with the greatest nonlinearity).
\hypertarget{PlotComparecTm1AD}{}
\begin{figure}
  \centerline{\includegraphics[width=6in]{./Figures/PlotComparecTm1AD}}
  \caption{$\cFunc_{T-1}(m_{T-1})$ (solid) versus $\Alt{\cFunc}_{T-1}(m_{T-1})$ (dashed)}
  \label{fig:ComparecTm1AD}
\end{figure}
Figure~\ref{fig:ComparecTm1AD} plots the actual consumption function
$\cFunc_{T-1}$ and the approximated consumption function $\Alt{\cFunc}_{T-1}$
derived by the method of endogenous grid points. Compared to the approximate consumption
functions illustrated in Figure~\ref{fig:PlotcTm1ABC}, $\Alt{\cFunc}_{T-1}$ is quite close
to the actual consumption function.


\ifthenelse{\boolean{MyNotes}}{\marginpar{\tiny Different transformation for $\vFunc$ than for $\vFunc^{a}$.}}{}

\hypertarget{Improving-the-a-Grid}{}
\subsection{Improving the $\aNrm$ Grid}

Thus far, we have arbitrarily used $\aNrm$ gridpoints of
$\{0.,1.,2.,3.,4.\}$ (augmented in the last subsection by
$\underline{a}_{T-1}$).  But it has been obvious from the figures that
the approximated $\Alt{\cFunc}_{\EndStpLst}$ function tends to be farthest from its true
value $\cFunc_{\EndStpLst}$ at low values of $a$.  Combining this with our insight that
$\underline{a}_{T-1}$ is a lower bound, we are now in position to
define a more deliberate method for constructing gridpoints for
$a_{T-1}$ -- a method that yields values that are more densely
spaced than the uniform grid at low values of $a$.  A pragmatic
choice that works well is to find the values such that (1) the last
value \textit{exceeds the lower bound} by the same amount $\bar{a}_{T-1}$
as our original maximum gridpoint (in our case, 4.); (2) we have the
same number of gridpoints as before; and (3) the \textit{multi-exponential growth rate} (that is, $e^{e^{e^{...}}}$ for some
number of exponentiations $n_{\TranShkEmp}$) from each point to the next point is
constant (instead of, as previously, imposing constancy of the
absolute gap between points).

\hypertarget{GothVInvVSGothCEEE}{}
\begin{figure}
  \centerline{\includegraphics[width=6in]{./Figures/GothVInvVSGothCEEE}}
  \caption{$\cFunc_{\EndStpLst}(a_{T-1})$ versus
    $\Alt{\cFunc}_{\EndStpLst}(a_{T-1})$, Multi-Exponential \texttt{aVec}}
  \label{fig:GothVInvVSGothCEE}
\end{figure}


\hypertarget{GothVVSGothCInvEEE}{}
\begin{figure}
  \includegraphics[width=6in]{./Figures/GothVVSGothCInvEEE}
  \caption{$\vFunc^{a}_{\EndStpLst}(a_{T-1})$ vs.
    $\Alt{\Alt{\vFunc}}_{\EndStpLst}^{a}(a_{T-1})$, Multi-Exponential \texttt{aVec}}
  \label{fig:GothVVSGothCInvEE}
\end{figure}

Section ``Improve the $\mathbb{A}_{grid}$'' begins by defining a function which takes as arguments the specifications of an initial grid of assets
(captured by the arguments \texttt{minval}, \texttt{maxval}, and \texttt{size}) and returns the new grid incorporating the
multi-exponential approach outlined above. Then, a call is made to this function and the improved
grid of assets is stored in the object \texttt{aVec\_eee}. Lastly, the endogenous gridpoint method described in the previous
section is performed using this new grid of assets.
Notice that the graphs depicted in Figures~\ref{fig:GothVInvVSGothCEE} and
\ref{fig:GothVVSGothCInvEE} are notably closer to their
respective truths than the corresponding figures that used the original
grid.

\MoM{
\hypertarget{The-Method-of-Moderation}{}
\subsection{The Method of Moderation}

\begin{verbatimwrite}{./cctwMoM/EndogGptsProbs.tex}
  Unfortunately, this endogenous gridpoints solution is not very
  well-behaved outside the original range of gridpoints targeted by
  the solution method.  (Though other common solution methods are no
  better outside their own predefined ranges).
  Figure~\ref{fig:ExtrapProblem} demonstrates the point by plotting
  the amount of precautionary saving implied by a linear extrapolation
  of our approximated consumption rule (the consumption of the perfect
  foresight consumer $\cFuncAbove_{T-1}$ minus our approximation to
  optimal consumption under uncertainty, $\Alt{\cFunc}_{T-1}$).
  Although theory proves that precautionary saving is always positive,
  the linearly extrapolated numerical approximation eventually
  predicts negative precautionary saving (at the point in the figure
  where the extrapolated locus crosses the horizontal axis).

  \hypertarget{ExtrapProblemPlot}{}
  \begin{figure}
    \includegraphics[width=6in]{./Figures/ExtrapProblemPlot}
    \caption{For Large Enough $m_{T-1}$, Predicted Precautionary Saving is Negative (Oops!)}
    \label{fig:ExtrapProblem}
  \end{figure}

  This error cannot be fixed by extending the upper gridpoint; in the
  presence of serious uncertainty, the consumption rule will need to be
  evaluated outside of \textit{any} prespecified grid (because starting
  from the top gridpoint, a large enough realization of the uncertain
  variable will push next period's realization of assets above that
  top; a similar argument applies below the bottom gridpoint).  While a judicious extrapolation technique can prevent this
  problem from being fatal (for example by carefully excluding negative
  precautionary saving), the problem is often dealt with using inelegant
  methods whose implications for the accuracy of the solution are
  difficult to gauge.
\end{verbatimwrite}
  Unfortunately, this endogenous gridpoints solution is not very
  well-behaved outside the original range of gridpoints targeted by
  the solution method.  (Though other common solution methods are no
  better outside their own predefined ranges).
  Figure~\ref{fig:ExtrapProblem} demonstrates the point by plotting
  the amount of precautionary saving implied by a linear extrapolation
  of our approximated consumption rule (the consumption of the perfect
  foresight consumer $\cFuncAbove_{\prd-1}$ minus our approximation to
  optimal consumption under uncertainty, $\Aprx{\cFunc}_{\prd-1}$).
  Although theory proves that precautionary saving is always positive,
  the linearly extrapolated numerical approximation eventually
  predicts negative precautionary saving (at the point in the figure
  where the extrapolated locus crosses the horizontal axis).

  \hypertarget{ExtrapProblemPlot}{}
  \begin{figure}
    \includegraphics[width=6in]{./Figures/ExtrapProblemPlot}
    \caption{For Large Enough ${m}_{\prd-1}$, Predicted Precautionary Saving is Negative (Oops!)}
    \label{fig:ExtrapProblem}
  \end{figure}

  This error cannot be fixed by extending the upper gridpoint; in the presence of serious uncertainty, the consumption rule will need to be evaluated outside of \textit{any} prespecified grid (because starting from the top gridpoint, a large enough realization of the uncertain variable will push next period's realization of assets above that top; a similar argument applies below the bottom gridpoint).  While a judicious extrapolation technique can prevent this problem from being fatal (for example by carefully excluding negative precautionary saving), the problem is often dealt with using inelegant methods whose implications for the accuracy of the solution are difficult to gauge.
\unskip


\renewcommand{\stp}{t} % For the rest of the doc, use generic t vs t+1

\begin{verbatimwrite}{./cctwMoM/MoM-Prelims.tex}
  As a preliminary to our solution, define $\hNrm_{\EndStp}$ as
  end-of-period human wealth (the present discounted value
  of future labor income) for a perfect foresight version of the problem
  of a `risk optimist:' a period-$t$ consumer who believes with perfect confidence
  that the shocks will always take their expected value of 
  \PermShkOn
  {1, $\TranShkEmp_{t+n} = \Ex[\TranShkEmp]=1~\forall~n>0$ and $\PermShk_{t+n} = \Ex[\PermShk]=1~\forall~n>0$.}
  {1, $\TranShkEmp_{t+n} = \Ex[\TranShkEmp]=1~\forall~n>0$.}
  The solution to a perfect foresight problem of this kind takes the
  form\footnote{For a derivation, see \cite{BufferStockTheory}; $\MPCmin_{t}$ is defined therein as the MPC of the perfect foresight consumer with horizon $T-t$.}
\end{verbatimwrite}
  As a preliminary to our solution, define $\hNrm_{\EndStp}$ as end-of-period human wealth (the present discounted value of future labor income) for a perfect foresight version of the problem of a `risk optimist:' a period-$t$ consumer who believes with perfect confidence that the shocks will always take their expected value of \PermShkOn {1, $\TranShkEmp_{\prd+n} = \Ex[\TranShkEmp]=1~\forall~n>0$ and $\PermShk_{\prd+n} = \Ex[\PermShk]=1~\forall~n>0$.}  {1, $\TranShkEmp_{\prd+n} = \Ex[\TranShkEmp]=1~\forall~n>0$.}  The solution to a perfect foresight problem of this kind takes the form\footnote{For a derivation, see \cite{BufferStockTheory}; $\MPCmin_{\prd}$ is defined therein as the MPC of the perfect foresight consumer with horizon $\prdT-\prdt$.}
\unskip
\begin{verbatimwrite}{./Equations/cFuncAbove.tex}
  \begin{equation}\begin{gathered}\begin{aligned}
        \cFuncAbove_{t}(\mNrm_{t})  & = (\mNrm_{t} + \hNrm_{\EndStp})\MPCmin_{t} \label{eq:cFuncAbove}
      \end{aligned}\end{gathered}\end{equation}
  for a constant minimal marginal propensity to consume $\MPCmin_{t}$ given below.
\end{verbatimwrite}
  \begin{equation}\begin{gathered}\begin{aligned}
        \cFuncAbove_{t}(\mNrm_{t})  & = (\mNrm_{t} + \hNrm_{\EndStg})\MPCmin_{t} \label{eq:cFuncAbove}
      \end{aligned}\end{gathered}\end{equation}
  for a constant minimal marginal propensity to consume $\MPCmin_{t}$ given below.
\unskip

\providecommand{\hEndMin}{\underline{\hNrm}}\renewcommand{\hEndMin}{\underline{\hNrm}}

\begin{verbatimwrite}{./cctwMoM/MoM-Words.tex}
  We similarly define $\hEndMin_{\EndStp}$ as `minimal human wealth,' the
  present discounted value of labor income if the shocks were to take on
  their worst possible value in every future period \PermShkOn
  {$\TranShkEmp_{t+n} = \TranShkEmpMin ~\forall~n>0$ and $\PermShk_{t+n} =
    \PermShkMin ~\forall~n>0$} {$\TranShkEmp_{t+n} = \TranShkEmpMin
    ~\forall~n>0$} (which we define as corresponding to the beliefs of a
  `pessimist').

  \ctw{}{We will call a `realist' the consumer who correctly perceives the true
    probabilities of the future risks and optimizes accordingly.}

  A first useful point is that, for the realist, a lower bound for the
  level of market resources is $\ushort{m}_{t} = -\hEndMin_{\EndStp}$, because
  if $m_{t}$ equalled this value then there would be a positive finite
  chance (however small) of receiving \PermShkOn
  {$\TranShkEmp_{t+n}=\TranShkEmpMin$ and $\PermShk_{t+n}=\PermShkMin$}
  {$\TranShkEmp_{t+n}=\TranShkEmpMin$}
  in
  every future period, which would require the consumer to set $c_{t}$
  to zero in order to guarantee that the intertemporal budget constraint
  holds\ctw{.}{~(this is the multiperiod generalization of the discussion in
    section \ref{subsec:LiqConstrSelfImposed} explaining the derivation of the `natural borrowing constraint' for period $T-1$,
    $\ushort{a}_{T-1}$).}  Since consumption of zero yields negative
  infinite utility, the solution to realist consumer's problem is not well
  defined for values of $m_{t} < \ushort{m}_{t}$, and the limiting
  value of the realist's $c_t$ is zero as $m_{t} \downarrow \ushort{m}_{t}$.

  Given this result, it will be convenient to define `excess' market
  resources as the amount by which actual resources exceed the lower
  bound, and `excess' human wealth as the amount by which mean expected human wealth
  exceeds guaranteed minimum human wealth:
  \begin{equation*}\begin{gathered}\begin{aligned}
        \aboveMin \mNrm_{t}  & = m_{t}+\overbrace{\hEndMin_{\EndStp}}^{=-\ushort{m}_{t}}
        \\  \aboveMin \hNrm_{\EndStp}  & = \hNrm_{\EndStp}-\hEndMin_{\EndStp}.
      \end{aligned}\end{gathered}\end{equation*}

  We can now transparently define the optimal
  consumption rules for the two perfect foresight problems, those of the
  `optimist' and the `pessimist.'  The `pessimist' perceives human
  wealth to be equal to its minimum feasible value $\hEndMin_{\EndStp}$ with certainty, so
  consumption is given by the perfect foresight solution
  \begin{equation*}\begin{gathered}\begin{aligned}
        \cFuncBelow_{t}(m_{t})  & = (m_{t}+\hEndMin_{\EndStp})\MPCmin_{t}
        \\  & = \aboveMin \mNrm_{t}\MPCmin_{t}
        .
      \end{aligned}\end{gathered}\end{equation*}

  The `optimist,' on the other hand, pretends that there is no uncertainty
  about future income, and therefore consumes
  \begin{equation*}\begin{gathered}\begin{aligned}
        \cFuncAbove_{t}(m_{t})  & = (m_{t} +\hEndMin_{\EndStp} - \hEndMin_{\EndStp} + \hNrm_{\EndStp} )\MPCmin_{t}
        \\    & = (\aboveMin \mNrm_{t} + \aboveMin \hNrm_{\EndStp})\MPCmin_{t}
        \\      & = \cFuncBelow_{t}(m_{t})+\aboveMin \hNrm_{\EndStp} \MPCmin_{t}
        .
      \end{aligned}\end{gathered}\end{equation*}

  It seems obvious that the spending of the realist will be strictly greater
  than that of the pessimist and strictly less than that of the
  optimist.  Figure~\ref{fig:IntExpFOCInvPesReaOptNeedHiPlot} illustrates the proposition for the consumption rule in period $T-1$.
\end{verbatimwrite}
  We similarly define $\hEndMin_{\EndStg}$ as `minimal human wealth,' the
  present discounted value of labor income if the shocks were to take on
  their worst possible value in every future period \PermShkOn
  {$\TranShkEmp_{t+n} = \TranShkEmpMin ~\forall~n>0$ and $\PermShk_{t+n} =
    \PermShkMin ~\forall~n>0$} {$\TranShkEmp_{t+n} = \TranShkEmpMin
    ~\forall~n>0$} (which we define as corresponding to the beliefs of a
  `pessimist').

  \ctw{}{We will call a `realist' the consumer who correctly perceives the true
    probabilities of the future risks and optimizes accordingly.}

  A first useful point is that, for the realist, a lower bound for the
  level of market resources is $\ushort{m}_{\prd} = -\hEndMin_{\EndStg}$, because
  if $m_{\prd}$ equalled this value then there would be a positive finite
  chance (however small) of receiving \PermShkOn
  {$\TranShkEmp_{t+n}=\TranShkEmpMin$ and $\PermShk_{t+n}=\PermShkMin$}
  {$\TranShkEmp_{t+n}=\TranShkEmpMin$}
  in
  every future period, which would require the consumer to set $c_{\prd}$
  to zero in order to guarantee that the intertemporal budget constraint
  holds\ctw{.}{~(this is the multiperiod generalization of the discussion in
    section \ref{subsec:LiqConstrSelfImposed} explaining the derivation of the `natural borrowing constraint' for period $\trmT-1$,
    $\ushort{a}_{\prd-1}$).}  Since consumption of zero yields negative
  infinite utility, the solution to realist consumer's problem is not well
  defined for values of $m_{\prd} < \ushort{m}_{\prd}$, and the limiting
  value of the realist's $c_t$ is zero as $m_{\prd} \downarrow \ushort{m}_{\prd}$.

  Given this result, it will be convenient to define `excess' market
  resources as the amount by which actual resources exceed the lower
  bound, and `excess' human wealth as the amount by which mean expected human wealth
  exceeds guaranteed minimum human wealth:
  \begin{equation*}\begin{gathered}\begin{aligned}
        \aboveMin \mNrm_{\prd}  & = m_{\prd}+\overbrace{\hEndMin_{\EndStg}}^{=-\ushort{m}_{\prd}}
        \\  \aboveMin \hNrm_{\EndStg}  & = \hNrm_{\EndStg}-\hEndMin_{\EndStg}.
      \end{aligned}\end{gathered}\end{equation*}

  We can now transparently define the optimal
  consumption rules for the two perfect foresight problems, those of the
  `optimist' and the `pessimist.'  The `pessimist' perceives human
  wealth to be equal to its minimum feasible value $\hEndMin_{\EndStg}$ with certainty, so
  consumption is given by the perfect foresight solution
  \begin{equation*}\begin{gathered}\begin{aligned}
        \cFuncBelow_{\prd}(m_{\prd})  & = (m_{\prd}+\hEndMin_{\EndStg})\MPCmin_{\prd}
        \\  & = \aboveMin \mNrm_{\prd}\MPCmin_{\prd}
        .
      \end{aligned}\end{gathered}\end{equation*}

  The `optimist,' on the other hand, pretends that there is no uncertainty
  about future income, and therefore consumes
  \begin{equation*}\begin{gathered}\begin{aligned}
        \cFuncAbove_{\prd}(m_{\prd})  & = (m_{\prd} +\hEndMin_{\EndStg} - \hEndMin_{\EndStg} + \hNrm_{\EndStg} )\MPCmin_{\prd}
        \\    & = (\aboveMin \mNrm_{\prd} + \aboveMin \hNrm_{\EndStg})\MPCmin_{\prd}
        \\      & = \cFuncBelow_{\prd}(m_{\prd})+\aboveMin \hNrm_{\EndStg} \MPCmin_{\prd}
        .
      \end{aligned}\end{gathered}\end{equation*}

  It seems obvious that the spending of the realist will be strictly greater
  than that of the pessimist and strictly less than that of the
  optimist.  Figure~\ref{fig:IntExpFOCInvPesReaOptNeedHiPlot} illustrates the proposition for the consumption rule in period $\trmT-1$.
\unskip
\begin{verbatimwrite}{\econtexRoot/Figures/IntExpFOCInvPesReaOptNeedHiPlot.tex}
  \hypertarget{IntExpFOCInvPesReaOptNeedHiPlot}{}
  \begin{figure}
    \includegraphics[width=6in]{./Figures/IntExpFOCInvPesReaOptNeedHiPlot}
    \caption{Moderation Illustrated: $\underline{\cFunc}_{T-1} < \Alt{\cFunc}_{T-1} < \bar{\cFunc}_{T-1}$}
    \label{fig:IntExpFOCInvPesReaOptNeedHiPlot}
  \end{figure}
\end{verbatimwrite}
  \hypertarget{IntExpFOCInvPesReaOptNeedHiPlot}{}
  \begin{figure}
    \includegraphics[width=6in]{./Figures/IntExpFOCInvPesReaOptNeedHiPlot}
    \caption{Moderation Illustrated: $\Min{\cFunc}_{\prd-1} < \Aprx{\cFunc}_{\prd-1} < \bar{\cFunc}_{\prd-1}$}
    \label{fig:IntExpFOCInvPesReaOptNeedHiPlot}
  \end{figure}
\unskip
\begin{verbatimwrite}{./cctwMoM/MoM-Words-Rest.tex}

  \indent The proof is more difficult than might be imagined, but
  the necessary work is done in \cite{BufferStockTheory} so we will take
  the proposition as a fact and proceed by manipulating the inequality:
\end{verbatimwrite}
\input{./cctwMoM/MoM-Words-Rest.tex}\unskip

\begin{verbatimwrite}{./Equations/MoM-Inequalities.tex}

  \begin{center}
\mbox{    \begin{tabular}{rcl}
      $ \aboveMin \mNrm_{t} \MPCmin_{t} < $ & $ \cFunc_{t}(\ushort{m}_{t}+\aboveMin \mNrm_{t}) $  $< (\aboveMin \mNrm_{t}+\aboveMin \hNrm_{\EndStp})\MPCmin_{t} $
      \\  $- \aboveMin \mNrm_{t} \MPCmin_{t} > $ & $ -\cFunc_{t}(\ushort{m}_{t}+\aboveMin \mNrm_{t}) $ & $> -(\aboveMin \mNrm_{t}+\aboveMin \hNrm_{\EndStp})\MPCmin_{t} $
      \\  $ \aboveMin \hNrm_{\EndStp} \MPCmin_{t} > $ & $ \bar{\cFunc}_{t}(\ushort{m}_{t}+\aboveMin \mNrm_{t})-\cFunc_{t}(\ushort{m}_{t}+\aboveMin \mNrm_{t}) $ & $> 0$
      \\  $1 > $ & $ \underbrace{\left(\frac{\bar{\cFunc}_{t}(\ushort{m}_{t}+\aboveMin \mNrm_{t})-\cFunc_{t}(\ushort{m}_{t}+\aboveMin \mNrm_{t})}{\aboveMin \hNrm_{\EndStp} \MPCmin_{t}}\right)}_{\equiv \Hi{\koppa}_{t}} $ & $> 0$
\end{tabular}
}
  \end{center}
\end{verbatimwrite}

  \begin{center}
    \begin{tabular}{rcl}
      $ \aboveMin \mNrm_{\prd} \MPCmin_{\prd} < $ & $ \cFunc_{\prd}(\ushort{m}_{\prd}+\aboveMin \mNrm_{\prd}) $  $< (\aboveMin \mNrm_{\prd}+\aboveMin \hNrm_{\EndStp})\MPCmin_{\prd} $
      \\  $- \aboveMin \mNrm_{\prd} \MPCmin_{\prd} > $ & $ -\cFunc_{\prd}(\ushort{m}_{\prd}+\aboveMin \mNrm_{\prd}) $ & $> -(\aboveMin \mNrm_{\prd}+\aboveMin \hNrm_{\EndStp})\MPCmin_{\prd} $
      \\  $ \aboveMin \hNrm_{\EndStp} \MPCmin_{\prd} > $ & $ \bar{\cFunc}_{\prd}(\ushort{m}_{\prd}+\aboveMin \mNrm_{\prd})-\cFunc_{\prd}(\ushort{m}_{\prd}+\aboveMin \mNrm_{\prd}) $ & $> 0$
      \\  $1 > $ & $ \underbrace{\left(\frac{\bar{\cFunc}_{\prd}(\ushort{m}_{\prd}+\aboveMin \mNrm_{\prd})-\cFunc_{\prd}(\ushort{m}_{\prd}+\aboveMin \mNrm_{\prd})}{\aboveMin \hNrm_{\EndStp} \MPCmin_{\prd}}\right)}_{\equiv \Hi{\koppa}_{\prd}} $ & $> 0$
    \end{tabular}
  \end{center}
\unskip

\begin{verbatimwrite}{./cctwMoM/MoM-Inequalities-Describe.tex}
\noindent  where the fraction in the middle of the last inequality is the ratio
  of actual precautionary saving (the numerator is the difference
  between perfect-foresight consumption and optimal consumption in the
  presence of uncertainty) to the maximum conceivable amount of
  precautionary saving (the amount that would be undertaken by the
  pessimist who consumes nothing out of any future income beyond the perfectly certain component).
\end{verbatimwrite}
  \noindent  where the fraction in the middle of the last inequality is the ratio
  of actual precautionary saving (the numerator is the difference
  between perfect-foresight consumption and optimal consumption in the
  presence of uncertainty) to the maximum conceivable amount of
  precautionary saving (the amount that would be undertaken by the
  pessimist who consumes nothing out of any future income beyond the perfectly certain component).
\unskip

\begin{verbatimwrite}{./Equations/MoM-KoppaOfMu.tex}
  Defining $\mu_{t} =
  \log \aboveMin \mNrm_{t}$ (which can range from $-\infty$ to $\infty$), the object in the middle of the last inequality is
  \begin{equation}\begin{gathered}\begin{aligned}
        \Hi{\koppa}_{t}(\mu_{t})   & \equiv  \left(\frac{\bar{\cFunc}_{t}(\ushort{m}_{t}+e^{\mu_{t}})-\cFunc_{t}(\ushort{m}_{t}+e^{\mu_{t}})}{\aboveMin \hNrm_{\EndStp} \MPCmin_{t}}\right), \label{eq:koppa}
      \end{aligned}\end{gathered}\end{equation}
  and we now define
  \begin{equation}\begin{gathered}\begin{aligned}
        \Hi{\chiFunc}_{t}(\mu_{t})  & = \log \left(\frac{1-\Hi{\koppa}_{t}(\mu_{t})}{\Hi{\koppa}_{t}(\mu_{t})}\right)
        \\  & = \log \left(1/\Hi{\koppa}_{t}(\mu_{t})-1\right) \label{eq:chi}
      \end{aligned}\end{gathered}\end{equation}
\end{verbatimwrite}
  Defining $\mu_{\prd} =
  \log \aboveMin \mNrm_{\prd}$ (which can range from $-\infty$ to $\infty$), the object in the middle of the last inequality is
  \begin{equation}\begin{gathered}\begin{aligned}
        \Max{\koppa}_{\prd}(\mu_{\prd})   & \equiv  \left(\frac{\bar{\cFunc}_{\prd}(\Min{m}_{\prd}+e^{\mu_{\prd}})-\cFunc_{\prd}(\Min{m}_{\prd}+e^{\mu_{\prd}})}{\aboveMin \hNrm_{\EndStg} \MPCmin_{\prd}}\right), \label{eq:koppa}
      \end{aligned}\end{gathered}\end{equation}
  and we now define
  \begin{equation}\begin{gathered}\begin{aligned}
        \Max{\chiFunc}_{\prd}(\mu_{\prd})  & = \log \left(\frac{1-\Max{\koppa}_{\prd}(\mu_{\prd})}{\Max{\koppa}_{\prd}(\mu_{\prd})}\right)
        \\  & = \log \left(1/\Max{\koppa}_{\prd}(\mu_{\prd})-1\right) \label{eq:chi}
      \end{aligned}\end{gathered}\end{equation}
\unskip
\begin{verbatimwrite}{./cctwMoM/MoM-KoppaOfMu-Describe.tex}
  which has the virtue that it is linear in the limit as $\mu_{t}$ approaches $+\infty$.

  Given $\Hi{\chiFunc}$, the consumption function can be recovered from
\end{verbatimwrite}
  which has the virtue that it is linear in the limit as $\mu_{\prd}$ approaches $+\infty$.

  Given $\Hi{\chiFunc}$, the consumption function can be recovered from
\unskip
\begin{verbatimwrite}{./Equations/cFuncHi.tex}
  \begin{equation}\begin{gathered}\begin{aligned}
        \Hi{\cFunc}_{t}  & = \bar{\cFunc}_{t}-\overbrace{\left(\frac{1}{1+\exp(\Hi{\chiFunc}_{t})}\right)}^{=\Hi{\koppa}_{t}} \aboveMin \hNrm_{\EndStp} \MPCmin_{t}. \label{eq:cFuncHi}
      \end{aligned}\end{gathered}\end{equation}
\end{verbatimwrite}
  \begin{equation}\begin{gathered}\begin{aligned}
        \Max{\cFunc}_{\prd}  & = \bar{\cFunc}_{\prd}-\overbrace{\left(\frac{1}{1+\exp(\Max{\chiFunc}_{\prd})}\right)}^{=\Max{\koppa}_{\prd}} \aboveMin \hNrm_{\EndStg} \MPCmin_{\prd}. \label{eq:cFuncHi}
      \end{aligned}\end{gathered}\end{equation}
\unskip

\begin{verbatimwrite}{./cctwMoM/MoM-End.tex}

  Thus, the procedure is to calculate $\Hi{\chiFunc}_{t}$ at the points
  $\vctr{\mu}_{t}$ corresponding to the log of the $\aboveMin
  \vctr{m}_{t}$ points defined above, and then using these to construct an
  interpolating approximation $\Alt{\Hi{\chiFunc}}_{t}$ from which we indirectly obtain our
  approximated consumption rule $\Alt{\Hi{\cFunc}}_{t}$ by substituting $\Alt{\Hi{\chiFunc}}_{t}$ for $\Hi{\chiFunc}$ in equation \eqref{eq:cFuncHi}.

  Because this method relies upon the fact that the problem is easy to
  solve if the decision maker has unreasonable views (either in the
  optimistic or the pessimistic direction), and because the correct
  solution is always between these immoderate extremes, we call our
  solution procedure the `method of moderation.'

  Results are shown in Figure~\ref{fig:ExtrapProblemSolved}; a reader
  with very good eyesight might be able to detect the barest hint of a
  discrepancy between the Truth and the Approximation at the far
  righthand edge of the figure\ctw{.}{ -- a stark contrast with the calamitous
    divergence evident in Figure~\ref{fig:ExtrapProblem}.}{}
  \hypertarget{ExtrapProblemSolvedPlot}{}
  \begin{figure}
    \includegraphics[width=6in]{./Figures/ExtrapProblemSolvedPlot}
    \caption{Extrapolated $\Alt{\Hi{\cFunc}}_{T-1}$ Constructed Using the Method of Moderation}
    \label{fig:ExtrapProblemSolved}
  \end{figure}
\end{verbatimwrite}

  Thus, the procedure is to calculate $\Hi{\chiFunc}_{\prd}$ at the points
  $\vctr{\mu}_{\prd}$ corresponding to the log of the $\aboveMin
  \vctr{m}_{\prd}$ points defined above, and then using these to construct an
  interpolating approximation $\Aprx{\Hi{\chiFunc}}_{\prd}$ from which we indirectly obtain our
  approximated consumption rule $\Aprx{\Hi{\cFunc}}_{\prd}$ by substituting $\Aprx{\Hi{\chiFunc}}_{\prd}$ for $\Hi{\chiFunc}$ in equation \eqref{eq:cFuncHi}.

  Because this method relies upon the fact that the problem is easy to
  solve if the decision maker has unreasonable views (either in the
  optimistic or the pessimistic direction), and because the correct
  solution is always between these immoderate extremes, we call our
  solution procedure the `method of moderation.'

  Results are shown in Figure~\ref{fig:ExtrapProblemSolved}; a reader
  with very good eyesight might be able to detect the barest hint of a
  discrepancy between the Truth and the Approximation at the far
  righthand edge of the figure\ctw{.}{ -- a stark contrast with the calamitous
    divergence evident in Figure~\ref{fig:ExtrapProblem}.}{}
  \hypertarget{ExtrapProblemSolvedPlot}{}
  \begin{figure}
    \includegraphics[width=6in]{./Figures/ExtrapProblemSolvedPlot}
    \caption{Extrapolated $\Aprx{\Hi{\cFunc}}_{\prd-1}$ Constructed Using the Method of Moderation}
    \label{fig:ExtrapProblemSolved}
  \end{figure}
\unskip

\hypertarget{Approximating-the-Slope-Too}{}
\subsection{Approximating the Slope Too}


Until now, we have calculated the level of consumption at
various different gridpoints and used linear interpolation\ctw{.}{ (either
  directly for $\cFunc_{T-1}$ or indirectly for, say, $\Hi{\chiFunc}_{T-1}$).}  But the
resulting piecewise linear approximations have the unattractive feature
that they are not differentiable at the `kink points' that correspond to
the gridpoints where the slope of the function changes discretely.



\cite{BufferStockTheory} proves that the true consumption function for
this problem
is `smooth:' It
exhibits a well-defined unique marginal propensity to consume at every
positive value of $m$.  This suggests that we should calculate, not
just the level of consumption, but also the marginal propensity to
consume (henceforth $\MPC$) at each gridpoint, and then find an
interpolating approximation that smoothly matches both the level and the slope
at those points.

This requires us to differentiate \eqref{eq:koppa} and \eqref{eq:chi}, yielding
\begin{equation}\begin{gathered}\begin{aligned}
      \Hi{\koppa}_{t}^{\mu}(\mu_{t})   & = (\aboveMin \hNrm_{\EndStp} \MPCmin_{t})^{-1}e^{\mu_{t}}\left(\MPCmin_{t}-\overbrace{\cFunc^{\mNrm}_{t}(\ushort{m}_{t}+e^{\mu_{t}})}^{\equiv \MPCFunc_{t}(\mNrm_{t})}\right)  \label{eq:koppaPrime}
      \\ \Hi{\chiFunc}_{t}^{\mu}(\mu_{t})  & = \left(\frac{-\Hi{\koppa}_{t}^{\mu}(\mu_{t})/\Hi{\koppa}_{t}^{2}}{1/\Hi{\koppa}_{t}(\mu_{t})-1}\right)
    \end{aligned}\end{gathered}\end{equation}
and (dropping arguments) with some algebra these can be combined to yield
\begin{equation}\begin{gathered}\begin{aligned}
      \Hi{\chiFunc}_{t}^{\mu}  & = \left(\frac{\MPCmin_{t} \aboveMin \mNrm_{t} \aboveMin \hNrm_{\EndStp} (\MPCmin_{t}-\MPC_{t})}
        {(\cFuncAbove_{t}-\cFunc_{t})(\cFuncAbove_{t}-\cFunc_{t} - \MPCmin_{t} \aboveMin \hNrm_{\EndStp})}\right).
    \end{aligned}\end{gathered}\end{equation}

To compute the vector of values of \eqref{eq:koppaPrime} corresponding
to the points in $\vctr{\mu}_{t}$, we need the marginal propensities to
consume (designated $\MPC$) at each of the gridpoints,
$\cFunc^{\mNrm}_{t}$ (the vector of such values is
$\vctr{\MPC}_{t}$).  These can be obtained by differentiating the
Euler equation \eqref{eq:upEqbetaOp} (where we define
$\mFunc_{\EndStp}(a) \equiv \cFunc_{\EndStp}(a)+{a}$, and drop the (a) arguments to reduce clutter):
\begin{equation}\begin{gathered}\begin{aligned}
      \uFunc^{c}(\cFunc_{\EndStp})   & = \hat{\vFunc}_{\EndStp}^{\aNrm}(\mFunc_{\EndStp}-\cFunc_{\EndStp}),
    \end{aligned}\end{gathered}\end{equation}
yielding a marginal propensity to
\textit{have consumed} $\cFunc_{\EndStp}^{\aNrm}$ at each gridpoint:
\begin{equation}\begin{gathered}\begin{aligned}
      \uPP(\cEndStp)\cEndStp^{\aNrm}  & = \hat{\vFunc}_{\EndStp}^{\aNrm}(\mFunc_{\EndStp}-\cFunc_{\EndStp})
      \\ \cEndStp^{\aNrm}  & = \hat{\vFunc}^{\aNrm}(\mFunc_{\EndStp}-\cFunc_{\EndStp})/\uPP(\cEndStp)
    \end{aligned}\end{gathered}\end{equation}
and the marginal propensity to consume at the beginning of the period is obtained from the marginal propensity to have consumed by differentiating the identity with respect to $\aNrm$:
\begin{equation*}\begin{gathered}\begin{aligned}
      \cEndStp  & = \mFunc_{\EndStp} - \aNrm
      \\ \cEndStp^{\aNrm}+1  & = \mFunc_{\EndStp}^{\aNrm}
    \end{aligned}\end{gathered}\end{equation*}
which, together with the chain rule $\cEndStp^{\aNrm}  = \cFunc^{\mNrm}_{\MidStp}\mFunc_{\EndStp}^{\aNrm}$, yields the MPC from
\begin{equation}\begin{gathered}\begin{aligned}
      \cFunc^{\mNrm}(\overbrace{\cEndStp^{\aNrm}+1}^{=\mFunc_{\EndStp}^{\aNrm}})  & = \cEndStp^{\aNrm}
      \\ \cFunc^{\mNrm}  & = \cEndStp^{\aNrm}/(1+\cEndStp^{\aNrm}) \label{eq:MPCfromMPTHC}.
    \end{aligned}\end{gathered}\end{equation}


Designating $\Alt{\Hi{\cFunc}}_{T-1}$ as the approximated consumption rule obtained using an interpolating polynomial approximation to $\Hi{\chiFunc}$ that matches both the level and the first derivative at the gridpoints, Figure~\ref{fig:IntExpFOCInvPesReaOptGapPlot} plots the difference between this latest approximation and the true consumption rule for period $T-1$ up to the same large value (far beyond the largest gridpoint) used in prior figures.  Of course, at the gridpoints the approximation will exactly match the true function; but this figure illustrates that the approximation is quite accurate far beyond the last gridpoint (which is the last point at which the difference touches the horizontal axis).  (We plot here the difference between the two functions rather than the level plotted in previous figures, because in levels the difference between the approximate and the exact function would not be detectable even to the most eagle-eyed reader.)



\hypertarget{IntExpFOCInvPesReaOptGapPlot}{}
\begin{figure}
  \includegraphics[width=6in]{./Figures/IntExpFOCInvPesReaOptGapPlot}
  \caption{Difference Between True $\cFunc_{T-1}$ and $\Alt{\Hi{\cFunc}}_{T-1}$ Is Minuscule}
  \label{fig:IntExpFOCInvPesReaOptGapPlot}
\end{figure}




\hypertarget{Value}{}
\subsection{Value}

\begin{verbatimwrite}{./cctwMoM/value-Intro.tex}

  Often it is useful to know the value function as well as the consumption rule.  Fortunately, many of the tricks used when solving for the consumption rule have a direct analogue in approximation of the value function.

  Consider the perfect foresight (or ``optimist's'') problem in period $T-1$.  Using the fact that in a perfect foresight model the growth factor for consumption is $(\Rfree \DiscFac)^{1/\CRRA}$, we can use the fact that $\cNrm_{T} = (\Rfree \DiscFac)^{1/\CRRA} \cNrm_{T-1}$ to calculate the value function in period $T-1$:
  \begin{equation*}\begin{gathered}\begin{aligned}
        \bar{\vFunc}_{T-1}(m_{T-1})  & \equiv  \uFunc(\cNrm_{T-1})+\DiscFac \uFunc(\cNrm_{T})
        \\  & = \uFunc(\cNrm_{T-1})\left(1+\DiscFac ((\DiscFac\Rfree)^{1/\CRRA})^{1-\CRRA}\right)
%        \\  & = \uFunc(\cNrm_{T-1})\left(1+\DiscFac (\DiscFac\Rfree)^{1/\CRRA-1}\right)
        \\  & = \uFunc(\cNrm_{T-1})\left(1+(\DiscFac\Rfree)^{1/\CRRA}/\Rfree\right)
        \\  & = \uFunc(\cNrm_{T-1})\underbrace{\mbox{PDV}_{t}^{T}(\cNrm)/\cNrm_{T-1}}_{\equiv \PDVCoverc_{T-1}^{T}}
      \end{aligned}\end{gathered}\end{equation*}
  where $\PDVCoverc_{t}^{T}=\mbox{PDV}_{t}^{T}(\cNrm)$ is the present discounted value of consumption, normalized by current consumption. Using the fact demonstrated in \cite{BufferStockTheory} that $\PDVCoverc_{t}=\MPC^{-1}_{t}$, a similar function can be constructed recursively for earlier periods, yielding the general expression \hypertarget{vFuncPF}{}
\end{verbatimwrite}

  Often it is useful to know the value function as well as the consumption rule.  Fortunately, many of the tricks used when solving for the consumption rule have a direct analogue in approximation of the value function.

  Consider the perfect foresight (or ``optimist's'') problem in period $\trmT-1$.  Using the fact that in a perfect foresight model the growth factor for consumption is $(\Rfree \DiscFac)^{1/\CRRA}$, we can use the fact that $\cNrm_{\prd} = (\Rfree \DiscFac)^{1/\CRRA} \cNrm_{\prd-1}$ to calculate the value function in period $\trmT-1$:
  \begin{equation*}\begin{gathered}\begin{aligned}
        \bar{\vFunc}_{\prd-1}(m_{\prd-1})  & \equiv  \uFunc(\cNrm_{\prd-1})+\DiscFac \uFunc(\cNrm_{\prd})
        \\  & = \uFunc(\cNrm_{\prd-1})\left(1+\DiscFac ((\DiscFac\Rfree)^{1/\CRRA})^{1-\CRRA}\right)
        % \\  & = \uFunc(\cNrm_{\prd-1})\left(1+\DiscFac (\DiscFac\Rfree)^{1/\CRRA-1}\right)
        \\  & = \uFunc(\cNrm_{\prd-1})\left(1+(\DiscFac\Rfree)^{1/\CRRA}/\Rfree\right)
        \\  & = \uFunc(\cNrm_{\prd-1})\underbrace{\mbox{PDV}_{\prd}^{T}(\cNrm)/\cNrm_{\prd-1}}_{\equiv \PDVCoverc_{\prd-1}^{T}}
      \end{aligned}\end{gathered}\end{equation*}
  where $\PDVCoverc_{\prd}^{T}=\mbox{PDV}_{\prd}^{T}(\cNrm)$ is the present discounted value of consumption, normalized by current consumption. Using the fact demonstrated in \cite{BufferStockTheory} that $\PDVCoverc_{\prd}=\MPC^{-1}_{\prd}$, a similar function can be constructed recursively for earlier periods, yielding the general expression \hypertarget{vFuncPF}{}
\unskip
\begin{verbatimwrite}{./Equations/vFuncPF.tex}
  \begin{equation}\begin{gathered}\begin{aligned}
        \bar{\vFunc}_{t}(m_{t})  & = \uFunc(\bar{\cNrm}_{t})\PDVCoverc_{t}^{T}\label{eq:vFuncPF}
        \\  & = \uFunc(\bar{c}_{t}) \MPCmin_{t}^{-1} % 20190820
        \\  & = \uFunc((\aboveMin \mNrm_{t}+\aboveMin \hNrm_{\EndStp})\MPCmin_{t}) \MPCmin_{t}^{-1} % 20190820
        \\  & = \uFunc(\aboveMin \mNrm_{t}+\aboveMin \hNrm_{\EndStp})\MPCmin_{t}^{1-\CRRA} \MPCmin_{t}^{-1} % 20190820
        \\  & = \uFunc(\aboveMin \mNrm_{t}+\aboveMin \hNrm_{\EndStp})\MPCmin_{t}^{-\CRRA}  % 20190820
      \end{aligned}\end{gathered}\end{equation}

  This can be transformed as
  \begin{equation*}\begin{gathered}\begin{aligned}
        \bar{\vInv}_{t}  & \equiv  \left((1-\CRRA)\bar{\vFunc}_{t}\right)^{1/(1-\CRRA)}
        \\  & = \cNrm_{t}(\PDVCoverc_{t}^{T})^{1/(1-\CRRA)}
        \\  & = (\aboveMin \mNrm_{t}+\aboveMin \hNrm_{\EndStp})\MPCmin_{t}^{-\CRRA/(1-\CRRA)}   % 20190820
      \end{aligned}\end{gathered}\end{equation*}
\end{verbatimwrite}
  \begin{equation}\begin{gathered}\begin{aligned}
        \bar{\vFunc}_{\prd}({m}_{\prd})  & = \uFunc(\bar{\cNrm}_{\prd})\PDVCoverc_{\prd}^{T}\label{eq:vFuncPF}
        \\  & = \uFunc(\bar{c}_{\prd}) \MPCmin_{\prd}^{-1} % 20190820
        \\  & = \uFunc((\aboveMin \mNrm_{\prd}+\aboveMin \hNrm_{\EndStep})\MPCmin_{\prd}) \MPCmin_{\prd}^{-1} % 20190820
        \\  & = \uFunc(\aboveMin \mNrm_{\prd}+\aboveMin \hNrm_{\EndStep})\MPCmin_{\prd}^{1-\CRRA} \MPCmin_{\prd}^{-1} % 20190820
        \\  & = \uFunc(\aboveMin \mNrm_{\prd}+\aboveMin \hNrm_{\EndStep})\MPCmin_{\prd}^{-\CRRA}  % 20190820
      \end{aligned}\end{gathered}\end{equation}

  This can be transformed as
  \begin{equation*}\begin{gathered}\begin{aligned}
        \bar{\vInv}_{\prd}  & \equiv  \left((1-\CRRA)\bar{\vFunc}_{\prd}\right)^{1/(1-\CRRA)}
        \\  & = \cNrm_{\prd}(\PDVCoverc_{\prd}^{T})^{1/(1-\CRRA)}
        \\  & = (\aboveMin \mNrm_{\prd}+\aboveMin \hNrm_{\EndStep})\MPCmin_{\prd}^{-\CRRA/(1-\CRRA)}   % 20190820
      \end{aligned}\end{gathered}\end{equation*}
\unskip
\begin{verbatimwrite}{./cctwMoM/value-Rest.tex}
  \MPCMatch{with derivative
    \begin{equation*}\begin{gathered}\begin{aligned}
          \bar{\vInv}_{t}^m  & = (\mathbb{C}_{t}^{T})^{1/(1-\CRRA)}\MPCmin_{t},
          \\  & = \MPCmin_{t}^{-\CRRA/(1-\CRRA)} % 20190820
        \end{aligned}\end{gathered}\end{equation*}}{}
  and since $\PDVCoverc_{t}^{T}$ is a constant while the consumption
  function is linear, $\bar{\vInv}_{t}$ will also be linear.

  We apply the same transformation to the value function for the problem with uncertainty (the ``realist's'' problem)\MPCMatch{ and differentiate}:
  \begin{equation*}\begin{gathered}\begin{aligned}
        \bar{\vInv}_{t}  & = \left((1-\CRRA)\bar{\vFunc}_{t}(m_{t})\right)^{1/(1-\CRRA)}
        \MPCMatch{\\ \bar{\vInv}^{m}_{t}  & = \left((1-\CRRA)\bar{\vFunc}_{t}(m_{t})\right)^{-1+1/(1-\CRRA)}\bar{\vFunc}_{t}^{m}(m_{t})}{}
      \end{aligned}\end{gathered}\end{equation*}
  and an excellent approximation to the value function can be obtained by
  calculating the values of $\bar{\vInv}$ at the same gridpoints used by the
  consumption function approximation, and interpolating among those points.

  However, as with the consumption approximation, we can do even better if we
  realize that the $\bar{\vInv}$ function for the optimist's problem is
  an upper bound for the ${\vInv}$ function in the presence of uncertainty, and the value function
  for the pessimist is a lower bound. Analogously to \eqref{eq:koppa}, define an upper-case
  \begin{equation}\begin{gathered}\begin{aligned}
        \hat{\Koppa}_{t}(\mu_{t})   & = \left(\frac{\bar{\vInv}_{t}(\ushort{m}_{t}+e^{\mu_{t}})-\vInv_{t}(\ushort{m}_{t}+e^{\mu_{t}})}{\aboveMin \hNrm_{\EndStp} \MPCmin_{t} (\PDVCoverc_{t}^{T})^{1/(1-\CRRA)}}\right) \label{eq:Koppa}
      \end{aligned}\end{gathered}\end{equation}
  \MPCMatch{with derivative (dropping arguments)
    \begin{equation}\begin{gathered}\begin{aligned}
          \hat{\Koppa}_{t}^{\mu}   & = (\aboveMin \hNrm_{\EndStp} \MPCmin_{t} (\PDVCoverc_{t}^{T})^{1/(1-\CRRA)})^{-1}e^{\mu_{t}}\left(\bar{\vInv}^{m}_{t}-\vInv^{m}_{t}\right) \label{eq:KoppaPrime}
          % \\  & =  (\aboveMin \hNrm_{\EndStp} \MPCmin_{t})^{-1}e^{\mu_{t}}\left((\PDVCoverc_{t}^{T})^{1/(1-\CRRA)}\MPCmin_{t}-\left((1-\CRRA)\vFunc_{t}(m_{t})\right)^{-1+1/(1-\CRRA)}\vFunc_{t}^{m}(m_{t})\right)  \notag
        \end{aligned}\end{gathered}\end{equation}}{}
  and an upper-case version of the $\chiFunc$ equation in \eqref{eq:chi}:
  \begin{equation}\begin{gathered}\begin{aligned}
        \hat{\Chi}_{t}(\mu_{t})  & = \log \left(\frac{1-\hat{\Koppa}_{t}(\mu_{t})}{\hat{\Koppa}_{t}(\mu_{t})}\right)
        \\  & = \log \left(1/\hat{\Koppa}_{t}(\mu_{t})-1\right) \label{eq:Chi}
      \end{aligned}\end{gathered}\end{equation}
  \MPCMatch{with corresponding derivative
    \begin{equation}\begin{gathered}\begin{aligned}
          \hat{\Chi}_{t}^{\mu}  & = \left(\frac{-\hat{\Koppa}_{t}^{\mu}/\hat{\Koppa}_{t}^{2}}{1/\hat{\Koppa}_{t}-1}\right)
        \end{aligned}\end{gathered}\end{equation}}{}
  and if we approximate these objects then invert them (as above with
  the $\Hi{\koppa}$ and $\Hi{\chiFunc}$ functions) we obtain a very high-quality
  approximation to our inverted value function at the same points for
  which we have our approximated value function:
  \begin{equation}\begin{gathered}\begin{aligned}
        \hat{\vInv}_{t}  & = \bar{\vInv}_{t}-\overbrace{\left(\frac{1}{1+\exp(\hat{\Chi}_{t})}\right)}^{=\hat{\Koppa}_{t}} \aboveMin \hNrm_{\EndStp} \MPCmin_{t} (\PDVCoverc_{t}^{T})^{1/(1-\CRRA) }
      \end{aligned}\end{gathered}\end{equation}
  from which we obtain our approximation to the value function\MPCMatch{ and its derivatives~}~as \hypertarget{vHatFunc}{}
  \begin{equation}\begin{gathered}\begin{aligned}
        \hat{\vFunc}_{t}  & = \uFunc(\hat{\vInv}_{t})
        \\  \hat{\vFunc}^{m}_{t}  & = \uFunc^{c}(\hat{\vInv}_{t}) \hat{\vInv}^{m}
        \MPCMatch{\\  \hat{\vFunc}^{mm}_{t}  & = \uFunc^{c{c}}(\hat{\vInv}_{t}) (\hat{\vInv}^{m})^{2} + \uFunc^{c}(\hat{\vInv}_{t})\hat{\vInv}^{mm}}{}
.
      \end{aligned}\end{gathered}\end{equation}

  Although a linear interpolation that matches the level of $\vInv$ at
  the gridpoints is simple, a Hermite interpolation that matches both
  the level and the derivative of the $\bar{\vInv}_{t}$ function at the
  gridpoints has the considerable virtue that the $\bar{\vFunc}_{t}$ derived from it numerically satisfies
  the envelope theorem at each of the gridpoints for which the problem
  has been solved.

  \MPCMatch{If we use the double-derivative calculated above to produce a higher-order Hermite polynomial, our approximation will also match
    marginal propensity to consume at the gridpoints; this would
    guarantee that the consumption function generated from the value
    function would match both the level of consumption and the
    marginal propensity to consume at the gridpoints; the numerical
    differences between the newly constructed consumption function and
    the highly accurate one constructed earlier would be negligible
    within the grid.}{}

\end{verbatimwrite}
  \MPCMatch{with derivative
    \begin{equation*}\begin{gathered}\begin{aligned}
          \bar{\vInv}_{\prd}^{m}  & = (\mathbb{C}_{\prd}^{T})^{1/(1-\CRRA)}\MPCmin_{\prd},
          \\  & = \MPCmin_{\prd}^{-\CRRA/(1-\CRRA)} % 20190820
        \end{aligned}\end{gathered}\end{equation*}}{}
  and since $\PDVCoverc_{\prd}^{T}$ is a constant while the consumption
  function is linear, $\bar{\vInv}_{\prd}$ will also be linear.

  We apply the same transformation to the value function for the problem with uncertainty (the ``realist's'' problem)\MPCMatch{ and differentiate}:
  \begin{equation*}\begin{gathered}\begin{aligned}
        \bar{\vInv}_{\prd}  & = \left((1-\CRRA)\bar{\vFunc}_{\prd}({m}_{\prd})\right)^{1/(1-\CRRA)}
        \MPCMatch{\\ \bar{\vInv}^{m}_{\prd}  & = \left((1-\CRRA)\bar{\vFunc}_{\prd}({m}_{\prd})\right)^{-1+1/(1-\CRRA)}\bar{\vFunc}_{\prd}^{m}({m}_{\prd})}{}
      \end{aligned}\end{gathered}\end{equation*}
  and an excellent approximation to the value function can be obtained by
  calculating the values of $\bar{\vInv}$ at the same gridpoints used by the
  consumption function approximation, and interpolating among those points.

  However, as with the consumption approximation, we can do even better if we
  realize that the $\bar{\vInv}$ function for the optimist's problem is
  an upper bound for the ${\vInv}$ function in the presence of uncertainty, and the value function
  for the pessimist is a lower bound. Analogously to \eqref{eq:koppa}, define an upper-case
  \begin{equation}\begin{gathered}\begin{aligned}
        \hat{\Koppa}_{\prd}(\mu_{\prd})   & = \left(\frac{\bar{\vInv}_{\prd}(\ushort{m}_{\prd}+e^{\mu_{\prd}})-\vInv_{\prd}(\ushort{m}_{\prd}+e^{\mu_{\prd}})}{\aboveMin \hNrm_{\EndStp} \MPCmin_{\prd} (\PDVCoverc_{\prd}^{T})^{1/(1-\CRRA)}}\right) \label{eq:Koppa}
      \end{aligned}\end{gathered}\end{equation}
  \MPCMatch{with derivative (dropping arguments)
    \begin{equation}\begin{gathered}\begin{aligned}
          \hat{\Koppa}_{\prd}^{\mu}   & = (\aboveMin \hNrm_{\EndStp} \MPCmin_{\prd} (\PDVCoverc_{\prd}^{T})^{1/(1-\CRRA)})^{-1}e^{\mu_{\prd}}\left(\bar{\vInv}^{m}_{\prd}-\vInv^{m}_{\prd}\right) \label{eq:KoppaPrime}
          % \\  & =  (\aboveMin \hNrm_{\EndStp} \MPCmin_{\prd})^{-1}e^{\mu_{\prd}}\left((\PDVCoverc_{\prd}^{T})^{1/(1-\CRRA)}\MPCmin_{\prd}-\left((1-\CRRA)\vFunc_{\prd}({m}_{\prd})\right)^{-1+1/(1-\CRRA)}\vFunc_{\prd}^{m}({m}_{\prd})\right)  \notag
        \end{aligned}\end{gathered}\end{equation}}{}
  and an upper-case version of the $\chiFunc$ equation in \eqref{eq:chi}:
  \begin{equation}\begin{gathered}\begin{aligned}
        \hat{\Chi}_{\prd}(\mu_{\prd})  & = \log \left(\frac{1-\hat{\Koppa}_{\prd}(\mu_{\prd})}{\hat{\Koppa}_{\prd}(\mu_{\prd})}\right)
        \\  & = \log \left(1/\hat{\Koppa}_{\prd}(\mu_{\prd})-1\right) \label{eq:Chi}
      \end{aligned}\end{gathered}\end{equation}
  \MPCMatch{with corresponding derivative
    \begin{equation}\begin{gathered}\begin{aligned}
          \hat{\Chi}_{\prd}^{\mu}  & = \left(\frac{-\hat{\Koppa}_{\prd}^{\mu}/\hat{\Koppa}_{\prd}^{2}}{1/\hat{\Koppa}_{\prd}-1}\right)
        \end{aligned}\end{gathered}\end{equation}}{}
  and if we approximate these objects then invert them (as above with
  the $\Hi{\koppa}$ and $\Hi{\chiFunc}$ functions) we obtain a very high-quality
  approximation to our inverted value function at the same points for
  which we have our approximated value function:
  \begin{equation}\begin{gathered}\begin{aligned}
        \hat{\vInv}_{\prd}  & = \bar{\vInv}_{\prd}-\overbrace{\left(\frac{1}{1+\exp(\hat{\Chi}_{\prd})}\right)}^{=\hat{\Koppa}_{\prd}} \aboveMin \hNrm_{\EndStp} \MPCmin_{\prd} (\PDVCoverc_{\prd}^{T})^{1/(1-\CRRA) }
      \end{aligned}\end{gathered}\end{equation}
  from which we obtain our approximation to the value function\MPCMatch{ and its derivatives~}~as \hypertarget{vHatFunc}{}
  \begin{equation}\begin{gathered}\begin{aligned}
        \hat{\vFunc}_{\prd}  & = \uFunc(\hat{\vInv}_{\prd})
        \\  \hat{\vFunc}^{m}_{\prd}  & = \uFunc^{{c}}(\hat{\vInv}_{\prd}) \hat{\vInv}^{m}
        \MPCMatch{\\  \hat{\vFunc}^{mm}_{\prd}  & = \uFunc^{{c}{c}}(\hat{\vInv}_{\prd}) (\hat{\vInv}^{m})^{2} + \uFunc^{{c}}(\hat{\vInv}_{\prd})\hat{\vInv}^{mm}}{}
        .
      \end{aligned}\end{gathered}\end{equation}

  Although a linear interpolation that matches the level of $\vInv$ at the gridpoints is simple, a Hermite interpolation that matches both the level and the derivative of the $\bar{\vInv}_{\prd}$ function at the gridpoints has the considerable virtue that the $\bar{\vFunc}_{\prd}$ derived from it numerically satisfies the envelope theorem at each of the gridpoints for which the problem has been solved.

  \MPCMatch{If we use the double-derivative calculated above to produce a higher-order Hermite polynomial, our approximation will also match
    marginal propensity to consume at the gridpoints; this would
    guarantee that the consumption function generated from the value
    function would match both the level of consumption and the
    marginal propensity to consume at the gridpoints; the numerical
    differences between the newly constructed consumption function and
    the highly accurate one constructed earlier would be negligible
    within the grid.}{}

\unskip

\hypertarget{Refinement-A-Tighter-Upper-Bound}{}
\subsection{Refinement: A Tighter Upper Bound}
\begin{verbatimwrite}{./cctwMoM/Tighter.tex}
  \cite{BufferStockTheory} derives an upper limit  $\MPCmax_{t}$ for the MPC as $m_{t}$
  approaches its lower bound.  Using this
  fact plus the strict concavity of the consumption function yields the
  proposition that
  \begin{equation}\begin{gathered}\begin{aligned}
        \cFunc_{t}(\ushort{m}_{t}+\aboveMin \mNrm_{t}) & < \MPCmax_{t} \aboveMin \mNrm_{t}.
      \end{aligned}\end{gathered}\end{equation}

  The solution method described above does not guarantee that
  approximated consumption will respect this constraint between gridpoints, and a failure to
  respect the constraint can occasionally cause computational problems in solving
  or simulating the model.  Here, we
  describe a method for constructing an approximation that always
  satisfies the constraint.

  \begin{comment} % Old text needs to be revised or eliminated
    That is, the realist's consumption function is bounded from above by both
    the \textit{unconstrained} optimist's problem already treated, as well as
    by the \textit{constrained} optimist's problem, which is a 45 degree line
    originating from $\ushort{m}_{t}$ on the $m$-axis, as shown in
    Figure~\ref{fig:IntExpFOCInvPesReaOptNeed45Plot}. The same is true for
    the value function, as illustrated in Figure
    \ref{fig:IntExpFOCInvPesReaOptNeed45ValuePlot}.

    \hypertarget{IntExpFOCInvPesReaOptNeed45Plot}{}
    \begin{figure}
      \includegraphics[width=6in]{./Figures/IntExpFOCInvPesReaOptNeed45Plot}
      \caption{45 Degree Line as Another Upper Bound}
      \label{fig:IntExpFOCInvPesReaOptNeed45Plot}
    \end{figure}

    \hypertarget{IntExpFOCInvPesReaOptNeed45ValuePlot}{}
    \begin{figure}
      \includegraphics[width=6in]{./Figures/IntExpFOCInvPesReaOptNeed45ValuePlot}
      \caption{A Constrained Optimist's Value Function as Another Upper Bound}
      \label{fig:IntExpFOCInvPesReaOptNeed45ValuePlot}
    \end{figure}

  \end{comment}

  \newcommand{\mtCusp}{\ensuremath{\mNrm_{t}^{\#}}}
  % \newcommand{\aboveMin \mtCusp}{\ensuremath{\aboveMin \mNrm_{t}^{\#}}}
\end{verbatimwrite}
  \cite{BufferStockTheory} derives an upper limit  $\MPCmax_{\prd}$ for the MPC as $m_{\prd}$
  approaches its lower bound.  Using this
  fact plus the strict concavity of the consumption function yields the
  proposition that
  \begin{equation}\begin{gathered}\begin{aligned}
        \cFunc_{\prd}(\Min{m}_{\prd}+\aboveMin \mNrm_{\prd}) & < \MPCmax_{\prd} \aboveMin \mNrm_{\prd}.
      \end{aligned}\end{gathered}\end{equation}

  The solution method described above does not guarantee that
  approximated consumption will respect this constraint between gridpoints, and a failure to
  respect the constraint can occasionally cause computational problems in solving
  or simulating the model.  Here, we
  describe a method for constructing an approximation that always
  satisfies the constraint.

  \begin{comment} % Old text needs to be revised or eliminated
    That is, the realist's consumption function is bounded from above by both
    the \textit{unconstrained} optimist's problem already treated, as well as
    by the \textit{constrained} optimist's problem, which is a 45 degree line
    originating from $\Min{m}_{\prd}$ on the $m$-axis, as shown in
    Figure~\ref{fig:IntExpFOCInvPesReaOptNeed45Plot}. The same is true for
    the value function, as illustrated in Figure
    \ref{fig:IntExpFOCInvPesReaOptNeed45ValuePlot}.

    \hypertarget{IntExpFOCInvPesReaOptNeed45Plot}{}
    \begin{figure}
      \includegraphics[width=6in]{./Figures/IntExpFOCInvPesReaOptNeed45Plot}
      \caption{45 Degree Line as Another Upper Bound}
      \label{fig:IntExpFOCInvPesReaOptNeed45Plot}
    \end{figure}

    \hypertarget{IntExpFOCInvPesReaOptNeed45ValuePlot}{}
    \begin{figure}
      \includegraphics[width=6in]{./Figures/IntExpFOCInvPesReaOptNeed45ValuePlot}
      \caption{A Constrained Optimist's Value Function as Another Upper Bound}
      \label{fig:IntExpFOCInvPesReaOptNeed45ValuePlot}
    \end{figure}

  \end{comment}

  \newcommand{\mtCusp}{\ensuremath{\mNrm_{\prd}^{\#}}}
  % \newcommand{\aboveMin \mtCusp}{\ensuremath{\aboveMin \mNrm_{\prd}^{\#}}}
\unskip

\begin{verbatimwrite}{./Equations/mtCusp.tex}
  Defining $\mtCusp$ as the `cusp' point where the two upper bounds
  intersect:
  \begin{equation*}\begin{gathered}\begin{aligned}
        \left(\aboveMin \mtCusp+\aboveMin \hNrm_{\EndStp}\right)\MPCmin_{t}  & =  \MPCmax_{t} \aboveMin \mtCusp \\
        \aboveMin \mtCusp  & =  \frac{\MPCmin_{t}\aboveMin \hNrm_{\EndStp}}{(1-\MPCmin_{t})\MPCmax_{t}} \\
        \mtCusp  & =  \frac{\MPCmin_{t}\hNrm_{\EndStp}-\hEndMin_{\EndStp}}{(1-\MPCmin_{t})\MPCmax_{t}},
      \end{aligned}\end{gathered}\end{equation*}
\end{verbatimwrite}
  Defining $\mtCusp$ as the `cusp' point where the two upper bounds
  intersect:
  \begin{equation*}\begin{gathered}\begin{aligned}
        \left(\aboveMin \mtCusp+\aboveMin \hNrm_{\EndStp}\right)\MPCmin_{\prd}  & =  \MPCmax_{\prd} \aboveMin \mtCusp \\
        \aboveMin \mtCusp  & =  \frac{\MPCmin_{\prd}\aboveMin \hNrm_{\EndStp}}{(1-\MPCmin_{\prd})\MPCmax_{\prd}} \\
        \mtCusp  & =  \frac{\MPCmin_{\prd}\hNrm_{\EndStp}-\hEndMin_{\EndStp}}{(1-\MPCmin_{\prd})\MPCmax_{\prd}},
      \end{aligned}\end{gathered}\end{equation*}
\unskip
\begin{verbatimwrite}{./Equations/TighterUpperBound.tex}
  we want to construct a consumption function for $m_{t} \in (\ushort{m}_{t}, \mtCusp]$ that respects the
  tighter upper bound:
  \begin{center}
    \begin{tabular}{rcl}
      $ \aboveMin \mNrm_{t} \MPCmin_{t} < $ & $ \cFunc_{t}(\ushort{m}_{t}+\aboveMin \mNrm_{t}) $  $< \MPCmax_{t} \aboveMin \mNrm_{t} $
      % \\  $-\aboveMin \mNrm_{t} \MPCmin_{t} > $ & $ -\cFunc_{t}(\ushort{m}_{t}+\aboveMin \mNrm_{t}) $ & $> -\aboveMin \mNrm_{t} $
      \\  $ \aboveMin \mNrm_{t}(\MPCmax_{t}- \MPCmin_{t}) > $ & $ \MPCmax_{t} \aboveMin \mNrm_{t}-\cFunc_{t}(\ushort{m}_{t}+\aboveMin \mNrm_{t}) $ & $> 0$
      \\  $1 > $ & $ \left(\frac{\MPCmax_{t} \aboveMin \mNrm_{t}-\cFunc_{t}(\ushort{m}_{t}+\aboveMin \mNrm_{t})}{\aboveMin \mNrm_{t}(\MPCmax_{t}- \MPCmin_{t})}\right) $ & $> 0$.
    \end{tabular}
  \end{center}
\end{verbatimwrite}
  we want to construct a consumption function for $m_{\prd} \in (\Min{m}_{\prd}, \mtCusp]$ that respects the
  tighter upper bound:
  \begin{center}
    \begin{tabular}{rcl}
      $ \aboveMin \mNrm_{\prd} \MPCmin_{\prd} < $ & $ \cFunc_{\prd}(\Min{m}_{\prd}+\aboveMin \mNrm_{\prd}) $  $< \MPCmax_{\prd} \aboveMin \mNrm_{\prd} $
      % \\  $-\aboveMin \mNrm_{\prd} \MPCmin_{\prd} > $ & $ -\cFunc_{\prd}(\Min{m}_{\prd}+\aboveMin \mNrm_{\prd}) $ & $> -\aboveMin \mNrm_{\prd} $
      \\  $ \aboveMin \mNrm_{\prd}(\MPCmax_{\prd}- \MPCmin_{\prd}) > $ & $ \MPCmax_{\prd} \aboveMin \mNrm_{\prd}-\cFunc_{\prd}(\Min{m}_{\prd}+\aboveMin \mNrm_{\prd}) $ & $> 0$
      \\  $1 > $ & $ \left(\frac{\MPCmax_{\prd} \aboveMin \mNrm_{\prd}-\cFunc_{\prd}(\Min{m}_{\prd}+\aboveMin \mNrm_{\prd})}{\aboveMin \mNrm_{\prd}(\MPCmax_{\prd}- \MPCmin_{\prd})}\right) $ & $> 0$.
    \end{tabular}
  \end{center}
\unskip

\begin{verbatimwrite}{./Equations/koppaLo.tex}
  Again defining $\mu_{t} =\log \aboveMin \mNrm_{t}$, the object in the middle of the inequality is
  \begin{equation*}\begin{gathered}\begin{aligned}
        \Lo{\koppa}_{t}(\mu_{t})  & \equiv  \frac{\MPCmax_{t}-\cFunc_{t}(\ushort{m}_{t}+e^{\mu_{t}})e^{-\mu_{t}}}{\MPCmax_{t}-\MPCmin_{t}} \label{eq:koppaL}
        \MPCMatch{\\ \Lo{\koppa}^{\mu}_{t}(\mu_{t})  & = \frac{\cFunc_{t}(\ushort{m}_{t}+e^{\mu_{t}})e^{-\mu_{t}}-\MPCFunc_{t}^{m}(\ushort{m}_{t}+e^{\mu_{t}})}{\MPCmax_{t}-\MPCmin_{t}}}{} .
      \end{aligned}\end{gathered}\end{equation*}
\end{verbatimwrite}
  Again defining $\mu_{\prd} =\log \aboveMin \mNrm_{\prd}$, the object in the middle of the inequality is
  \begin{equation*}\begin{gathered}\begin{aligned}
        \Lo{\koppa}_{\prd}(\mu_{\prd})  & \equiv  \frac{\MPCmax_{\prd}-\cFunc_{\prd}(\ushort{m}_{\prd}+e^{\mu_{\prd}})e^{-\mu_{\prd}}}{\MPCmax_{\prd}-\MPCmin_{\prd}} \label{eq:koppaL}
        \MPCMatch{\\ \Lo{\koppa}^{\mu}_{\prd}(\mu_{\prd})  & = \frac{\cFunc_{\prd}(\ushort{m}_{\prd}+e^{\mu_{\prd}})e^{-\mu_{\prd}}-\MPCFunc_{\prd}^{m}(\ushort{m}_{\prd}+e^{\mu_{\prd}})}{\MPCmax_{\prd}-\MPCmin_{\prd}}}{} .
      \end{aligned}\end{gathered}\end{equation*}
\unskip

\begin{verbatimwrite}{./cctwMoM/TighterThreeFuncs.tex}
  As $m_{t}$ approaches
  $-\ushort{m}_{t}$, $\Lo{\koppa}_{t}(\mu_{t})$ converges to zero, while as $m_{t}$
  approaches $+\infty$, $\Lo{\koppa}_{t}(\mu_{t})$ approaches $1$.

  As before, we can derive an approximated consumption function; call it
  $\Alt{\Lo{\cFunc}}_{t}$.  This function will clearly do a better job approximating the consumption
  function for low values of $\mNrm_{t}$ while the previous approximation will perform better
  for high values of $\mNrm_{t}$.

  For middling values of $\mNrm$ it is not clear which of these
  functions will perform better.  However, an alternative is available
  which performs well.  Define the highest gridpoint below $\mtCusp$ as
  $\bar{\check{\mNrm}}_{t}^{\#}$ and the lowest gridpoint above $\mtCusp$ as
  $\ushort{\hat{\mNrm}}_{t}^{\#}$.  Then there will be a unique interpolating
  polynomial that matches the level and slope of the consumption function
  at these two points.  Call this function $\tilde{\cFunc}_{t}(\mNrm)$.

  Using indicator functions that are zero everywhere except for specified intervals,
\end{verbatimwrite}
  As $m_{\prd}$ approaches
  $-\Lo{m}_{\prd}$, $\Lo{\koppa}_{\prd}(\mu_{\prd})$ converges to zero, while as $m_{\prd}$
  approaches $+\infty$, $\Lo{\koppa}_{\prd}(\mu_{\prd})$ approaches $1$.

  As before, we can derive an approximated consumption function; call it $\Aprx{\Lo{\cFunc}}_{\prd}$.  This function will clearly do a better job approximating the consumption function for low values of $\mNrm_{\prd}$ while the previous approximation will perform better for high values of $\mNrm_{\prd}$.

  For middling values of $\mNrm$ it is not clear which of these functions will perform better.  However, an alternative is available which performs well.  Define the highest gridpoint below $\mtCusp$ as $\bar{\check{\mNrm}}_{\prd}^{\#}$ and the lowest gridpoint above $\mtCusp$ as $\Lo{\hat{\mNrm}}_{\prd}^{\#}$.  Then there will be a unique interpolating polynomial that matches the level and slope of the consumption function at these two points.  Call this function $\tilde{\cFunc}_{\prd}(\mNrm)$.

  Using indicator functions that are zero everywhere except for specified intervals,
\unskip
\begin{verbatimwrite}{./Equations/TighterThreeEqns.tex}
  \begin{equation*}\begin{gathered}\begin{aligned}
        \pmb{1}_{\text{Lo}}(\mNrm)  & = 1 \text{~if $          \mNrm \leq  \bar{\check{\mNrm}}_{t}^{\#} \phantom{< \mNrm <   \ushort{\hat{\mNrm}}_{t}^{\#}          \leq \mNrm}$}
        \\  \pmb{1}_{\text{Mid}}(\mNrm)  & = 1 \text{~if $\phantom{ \mNrm \leq}~ \bar{\check{\mNrm}}_{t}^{\#}          < \mNrm <   \ushort{\hat{\mNrm}}_{t}^{\#} \phantom{\leq \mNrm}$}
        \\  \pmb{1}_{\text{Hi}}(\mNrm)  & = 1 \text{~if $\phantom{ \mNrm \leq  ~\bar{\check{\mNrm}}_{t}^{\#}          < \mNrm < } \ushort{\hat{\mNrm}}_{t}^{\#}           \leq \mNrm$}
      \end{aligned}\end{gathered}\end{equation*}
  we can define a well-behaved approximating consumption function
  \begin{equation}\begin{gathered}\begin{aligned}
        \Alt{\cFunc}_{t}  & = \pmb{1}_{\text{Lo}} \Alt{\Lo{\cFunc}}_{t} + \pmb{1}_{\text{Mid}} \Alt{\tilde{\cFunc}}_{t}+\pmb{1}_{\text{Hi}} \Alt{\Hi{\cFunc}}_{t}.
      \end{aligned}\end{gathered}\end{equation}
\end{verbatimwrite}
  \begin{equation*}\begin{gathered}\begin{aligned}
        \vctr{1}_{\text{Lo}}(\mNrm)  & = 1 \text{~if $          \mNrm \leq  \bar{\check{\mNrm}}_{\prd}^{\#} \phantom{< \mNrm <   \Lo{\hat{\mNrm}}_{\prd}^{\#}          \leq \mNrm}$}
        \\  \vctr{1}_{\text{Mid}}(\mNrm)  & = 1 \text{~if $\phantom{ \mNrm \leq}~ \bar{\check{\mNrm}}_{\prd}^{\#}          < \mNrm <   \Lo{\hat{\mNrm}}_{\prd}^{\#} \phantom{\leq \mNrm}$}
        \\  \vctr{1}_{\text{Hi}}(\mNrm)  & = 1 \text{~if $\phantom{ \mNrm \leq  ~\bar{\check{\mNrm}}_{\prd}^{\#}          < \mNrm < } \Lo{\hat{\mNrm}}_{\prd}^{\#}           \leq \mNrm$}
      \end{aligned}\end{gathered}\end{equation*}
  we can define a well-behaved approximating consumption function
  \begin{equation}\begin{gathered}\begin{aligned}
        \Aprx{\cFunc}_{\prd}  & = \vctr{1}_{\text{Lo}} \Aprx{\Lo{\cFunc}}_{\prd} + \vctr{1}_{\text{Mid}} \Aprx{\tilde{\cFunc}}_{\prd}+\vctr{1}_{\text{Hi}} \Aprx{\Hi{\cFunc}}_{\prd}.
      \end{aligned}\end{gathered}\end{equation}
\unskip

\begin{verbatimwrite}{./cctwMoM/TighterThreeFuncsExplain.tex}
  This just says that, for each interval, we use the approximation that
  is most appropriate.  The function is continuous and
  once-differentiable everywhere, and is therefore well behaved for
  computational purposes.
  \begin{comment}
    In practice, in our problem the difference due to this refinement is displayed in Figure \ref{fig:IntExpFOCInvPesReaOpt45GapPlot}.
    \hypertarget{IntExpFOCInvPesReaOpt45GapPlot}{}
    \begin{figure}
      \includegraphics[width=6in]{./Figures/IntExpFOCInvPesReaOpt45GapPlot}
      \caption{Difference Between $\Alt{\Hi{\cFunc}}_{L, T-1}$ and $\Alt{\Hi{\cFunc}}_{H,T-1}$ is Small}
      \label{fig:IntExpFOCInvPesReaOpt45GapPlot}
    \end{figure}
  \end{comment}

  We now construct an upper-bound value function implied for a consumer whose spending behavior is consistent with the refined upper-bound consumption rule.

  For $\mNrm_{t} \geq \mNrm_{t}^{\#}$, this consumption rule is the same as before,
  so the constructed upper-bound value function is also the same.  However, for
  values $\mNrm_{t} < \mNrm_{t}^{\#}$ matters are slightly more complicated.

  Start with the fact that at the cusp point,
  \begin{equation*}\begin{gathered}\begin{aligned}
        \bar{\vFunc}_{t}(\mtCusp)  & = \uFunc(\bar{\cNrm}_{t}(\mtCusp))\PDVCoverc_{t}^T \\
        & =  \uFunc(\aboveMin \mtCusp  \MPCmax_{t})\PDVCoverc_{t}^{T}
        .
      \end{aligned}\end{gathered}\end{equation*}

  But for \textit{all} $\mNrm_{t}$,
  \begin{equation*}\begin{gathered}\begin{aligned}
        \bar{\vFunc}_{t}(\mNrm)  & = \uFunc(\bar{\cNrm}_{t}(\mNrm))+ \bar{\vEnd}(\mNrm-\bar{\cNrm}_{t}(\mNrm)),
      \end{aligned}\end{gathered}\end{equation*}
  and we assume that for the consumer below the cusp point consumption is given by $\MPCmax \aboveMin \mNrm_{t}$ so for $\mNrm_{t}< \mtCusp$
  \begin{equation*}\begin{gathered}\begin{aligned}
        \bar{\vFunc}_{t}(\mNrm)  & = \uFunc( \MPCmax_{t} \aboveMin \mNrm)+ \bar{\vEnd}((1-\MPCmax_{t})\aboveMin \mNrm),
      \end{aligned}\end{gathered}\end{equation*}
  which is easy to compute because $\bar{\vEnd}(\aNrm_{t}) = \DiscFac \bar{\vFunc}_{t+1}(\aNrm_{t}\RNrm+1)$
  where $\bar{\vFunc}_{t}$ is as defined above because a consumer who ends the current period with assets exceeding
  the lower bound will not expect to be constrained next period.  (Recall again that we are merely constructing an object that is guaranteed to be an \textit{upper bound} for the value that the `realist' consumer will experience.)  At the gridpoints defined by the solution of the
  consumption problem can then construct
  \begin{equation*}\begin{gathered}\begin{aligned}
        \bar{\vInv}_{t}(\mNrm)  & = ((1-\CRRA)\bar{\vFunc}_{t}(\mNrm))^{1/(1-\CRRA)}
      \end{aligned}\end{gathered}\end{equation*}
  \MPCMatch{and its derivatives}{} which yields the appropriate vector for constructing $\check{\Chi}$ and $\check{\Koppa}$.  The rest of the procedure is analogous to that performed for the consumption rule and is thus omitted for brevity.

\end{verbatimwrite}
This just says that, for each interval, we use the approximation that
is most appropriate.  The function is continuous and
once-differentiable everywhere, and is therefore well behaved for
computational purposes.

% \begin{comment}
%     In practice, in our problem the difference due to this refinement is displayed in Figure \ref{fig:IntExpFOCInvPesReaOpt45GapPlot}.
%     \hypertarget{IntExpFOCInvPesReaOpt45GapPlot}{}
%     \begin{figure}
%       \includegraphics[width=6in]{./Figures/IntExpFOCInvPesReaOpt45GapPlot}
%       \caption{Difference Between $\Aprx{\Max{\cFunc}}_{L, T-1}$ and $\Aprx{\Max{\cFunc}}_{H,T-1}$ is Small}
%       \label{fig:IntExpFOCInvPesReaOpt45GapPlot}
%     \end{figure}
%   \end{comment}

We now construct an upper-bound value function implied for a consumer whose spending behavior is consistent with the refined upper-bound consumption rule.

For $\mNrm_{\prd} \geq \mNrm_{\prd}^{\#}$, this consumption rule is the same as before,

  so the constructed upper-bound value function is also the same.  However, for
  values $\mNrm_{\prd} < \mNrm_{\prd}^{\#}$ matters are slightly more complicated.

  Start with the fact that at the cusp point,
  \begin{equation*}\begin{gathered}\begin{aligned}
        \bar{\vFunc}_{\prd}(\mtCusp)  & = \uFunc(\bar{\cNrm}_{\prd}(\mtCusp))\PDVCoverc_{\prd}^T \\
        & =  \uFunc(\aboveMin \mtCusp  \MPCmax_{\prd})\PDVCoverc_{\prd}^{T}
        .
      \end{aligned}\end{gathered}\end{equation*}

  But for \textit{all} $\mNrm_{\prd}$,
  \begin{equation*}\begin{gathered}\begin{aligned}
        \bar{\vFunc}_{\prd}(\mNrm)  & = \uFunc(\bar{\cNrm}_{\prd}(\mNrm))+ \bar{\vEndPrd}(\mNrm-\bar{\cNrm}_{\prd}(\mNrm)),
  \end{aligned}\end{gathered}\end{equation*}
  and we assume that for the consumer below the cusp point consumption is given by $\MPCmax \aboveMin \mNrm_{\prd}$ so for $\mNrm_{\prd}< \mtCusp$
  \begin{equation*}\begin{gathered}\begin{aligned}
        \bar{\vFunc}_{\prd}(\mNrm)  & = \uFunc( \MPCmax_{\prd} \aboveMin \mNrm)+ \bar{\vEndPrd}((1-\MPCmax_{\prd})\aboveMin \mNrm),
      \end{aligned}\end{gathered}\end{equation*}
  which is easy to compute because $\bar{\vEndPrd}(\aNrm_{\prd}) = \DiscFac \bar{\vFunc}_{\prd+1}(\aNrm_{\prd}\RNrmByG+1)$ where $\bar{\vFunc}_{\prd}$ is as defined above because a consumer who ends the current period with assets exceeding the lower bound will not expect to be constrained next period.  (Recall again that we are merely constructing an object that is guaranteed to be an \textit{upper bound} for the value that the `realist' consumer will experience.)  At the gridpoints defined by the solution of the consumption problem can then construct
  \begin{equation*}\begin{gathered}\begin{aligned}
        \bar{\vInv}_{\prd}(\mNrm)  & = ((1-\CRRA)\bar{\vFunc}_{\prd}(\mNrm))^{1/(1-\CRRA)}
      \end{aligned}\end{gathered}\end{equation*}
\MPCMatch{and its derivatives}{} which yields the appropriate vector for constructing $\check{\Chi}$ and $\check{\Koppa}$.  The rest of the procedure is analogous to that performed for the consumption rule and is thus omitted for brevity.

\unskip

\hypertarget{Extension-A-Stochastic-Interest-Factor}{}
\subsection{Extension: A Stochastic Interest Factor}


Thus far we have assumed that the interest factor is constant at $\Rfree$.  Extending the
previous derivations to allow for a perfectly forecastable time-varying interest factor $\Rfree_{t}$
would be trivial.  Allowing for a stochastic interest factor is less trivial.


The easiest case is where the interest factor is i.i.d.,
\begin{verbatimwrite}{./Equations/distRisky.tex}
  \begin{equation}\begin{gathered}\begin{aligned}
        \log \Risky_{t+n} & \sim & \mathcal{N}(\rfree + \eprem - \sigma^{2}_{\risky}/2,\sigma^{2}_{\risky}) ~\forall~n>0 \label{eq:distRisky}
      \end{aligned}\end{gathered}\end{equation}
\end{verbatimwrite}
  \begin{equation}\begin{gathered}\begin{aligned}
        \log \Risky_{t+n} & \sim & \Nrml(\rfree + \eprem - \sigma^{2}_{\risky}/2,\sigma^{2}_{\risky}) ~\forall~n>0 \label{eq:distRisky}
      \end{aligned}\end{gathered}\end{equation}
\unskip
where $\eprem$ is the risk premium and the $\sigma^{2}_{\risky}/2$ adjustment to the mean log return
guarantees that an increase in $\sigma^{2}_{\risky}$ constitutes a mean-preserving spread in the level of the return.

This case is reasonably straightforward because \cite{merton:restat} and \cite{samuelson:portfolio} showed
that for a consumer without labor income (or with perfectly forecastable labor income) the consumption
function is linear, with an infinite-horizon MPC\footnote{See \handoutC{CRRA-RateRisk} for a derivation.}
\begin{equation}\begin{gathered}\begin{aligned}
      \MPC  & = 1- \left(\DiscFac  \Ex_{\BegStp}[\Risky_{t+1}^{1-\CRRA}]\right)^{1/\CRRA} \label{eq:MPCExact}
    \end{aligned}\end{gathered}\end{equation}
and in this case the previous analysis applies once we substitute this MPC for the one that characterizes
the perfect foresight problem without rate-of-return risk.

The more realistic case where the interest factor has some serial correlation is more complex.  We consider
the simplest case that captures the main features of empirical interest rate dynamics: An AR(1) process.  Thus
the specification is
\begin{equation}\begin{gathered}\begin{aligned}
      \risky_{t+1}-\risky  & = (\risky_{t}-\risky) \gamma + \epsilon_{t+1}
    \end{aligned}\end{gathered}\end{equation}
where $\risky$ is the long-run mean log interest factor, $0 < \gamma < 1$ is the AR(1) serial correlation
coefficient, and $\epsilon_{t+1}$ is the stochastic shock.

The consumer's problem in this case now has two state variables, $\mNrm_{t}$ and $\risky_{t}$, and
is described by
\begin{equation}\begin{gathered}\begin{aligned}
      \vFunc_{t}(m_{t},\risky_{t})  & = \max_{{c}_{t}} ~ \uFunc(c_{t})+
      \Ex_{\BegStp}[{\DiscFac}_{t+1}\PermGroFacAdjV{\vFunc}_{t+1}(m_{t+1},\risky_{t+1})] \label{vNormedRisky}
      \\         & \text{s.t.}   \nonumber \\
      a_{t}    & = m_{t}-c_{t} \nonumber
      \\      \risky_{t+1}-\risky  & = (\risky_{t}-\risky)\gamma + \epsilon_{t+1} \notag
      \\      \Risky_{t+1}  & = \exp(\risky_{t+1}) \notag
      \\      m_{t+1}  & = \underbrace{\left(\Risky_{t+1}/\PermGroFac_{t+1}\right)}_{\equiv \Rprod_{t+1}}a_{t}+\TranShkEmp_{t+1} \nonumber.
    \end{aligned}\end{gathered}\end{equation}

% Kiichi: I will need you to read the literature and figure out how exactly we want to choose the Markov points and transition probabilities.
% When done, you will fill in the [how] text below.

We approximate the AR(1) process by a Markov transition matrix using standard techniques.  The stochastic interest factor is allowed to take
on 11 values centered around the steady-state value $\risky$.  Given this Markov transition matrix, \textit{conditional} on the Markov AR(1) state the consumption functions for the `optimist' and the `pessimist' will still be linear,
with identical MPC's that are computed numerically.  Given these MPC's, the (conditional) realist's consumption function can be computed for each Markov state, and the converged consumption rules constitute the solution contingent on the dynamics of the stochastic
interest rate process.

In principle, this refinement should be combined with the previous one;
further exposition of this combination is omitted here because no new
insights spring from the combination of the two techniques.



\hypertarget{Imposing-Artificial-Borrowing-Constraints}{}
\subsection{Imposing `Artificial' Borrowing Constraints}

Optimization problems often come with additional constraints that must
be satisfied.  Particularly common is an `artificial' liquidity constraint that
prevents the consumer's net worth from falling below some value, often
zero.\footnote{The word artificial is chosen only because of its clarity in distinguishing
  this from the case of the `natural' borrowing constraint examined above; no derogation is
  intended -- constraints of this kind certainly exist in the real world.}  The problem then becomes
\begin{equation*}\begin{gathered}\begin{aligned}
      \vFunc_{T-1}(m_{T-1})  & = \max_{\cNrm_{T-1}} ~~ \uFunc(c_{T-1}) + \Ex_{T-1} [\DiscFac \PermGroFacAdjV{\vFunc}_{\cntn(T)}(m_{T})] \label{eq:ConstrArt}
      \\ & \mbox{s.t.}  \nonumber
      \\ a_{T-1}  & = m_{T-1} - c_{T-1}
      \\ m_{T}  & = \RNrm_{T} a_{T-1} + \TranShkEmp_{T}
      \\ a_{T-1} & \geq 0 .
    \end{aligned}\end{gathered}\end{equation*}

\ifthenelse{\boolean{MyNotes}}{\marginpar{\tiny Constraint binds whenever you would like to consume more than current resources.}}{}

By definition, the constraint will bind if the unconstrained consumer
would choose a level of spending that would violate the constraint.
Here, that means that the constraint binds if the $c_{T-1}$
that satisfies the unconstrained FOC
\begin{equation}\begin{gathered}\begin{aligned}
      c_{T-1}^{-\CRRA}  & = \vFunc^{a}_{\EndStpLst}(m_{T-1}-c_{T-1}) \label{eq:cUnc}
    \end{aligned}\end{gathered}\end{equation}
is greater than $m_{T-1}$.  Call $\grave{\cFunc}^{\ast}_{T-1}$ the approximated function
returning the level of $c_{T-1}$ that satisfies \eqref{eq:cUnc}.
Then the approximated constrained optimal consumption function will be
\begin{verbatimwrite}{./Equations/LiqCons.tex}
  \begin{equation}\begin{gathered}\begin{aligned}
        \grave{\cFunc}_{T-1}(m_{T-1})  & = \min[{m}_{T-1},\grave{\cFunc}^{\ast}_{T-1}(m_{T-1})] \label{eq:LiqCons}.
      \end{aligned}\end{gathered}\end{equation}
\end{verbatimwrite}
  \begin{equation}\begin{gathered}\begin{aligned}
        \grave{\cFunc}_{\prd-1}({m}_{\prd-1})  & = \min[{m}_{\prd-1},\grave{\cFunc}^{\ast}_{\prd-1}({m}_{\prd-1})] \label{eq:LiqCons}.
      \end{aligned}\end{gathered}\end{equation}
\unskip

\ifthenelse{\boolean{MyNotes}}{\marginpar{\tiny Read this carefully
    before class.  Intuition: consider discounted mv of saving zero.  If
    consume everything and get the same $\uFunc^{c}$, then happy.  If
    consumed $\TranShkEmp$ less, mv of saving would be $> \uFunc^{c}(c).$}}{}

The introduction of the constraint also introduces a sharp
nonlinearity in all of the functions at the point where the constraint
begins to bind.  As a result, to get solutions that are anywhere close
to numerically accurate it is useful to augment the grid of values of
the state variable to include the exact value at which the constraint
ceases to bind.  Fortunately, this is easy to calculate.  We know that
when the constraint is binding the consumer is saving nothing, which
yields marginal value of $\vFunc^{a}_{\EndStpLst}(0)$. Further, when the
constraint is binding, $c_{T-1} = m_{T-1}$.  Thus, the largest
value of consumption for which the constraint is binding will be the
point for which the marginal utility of consumption is exactly equal
to the (expected, discounted) marginal value of saving 0.  We know
this because the marginal utility of consumption is a downward-sloping
function and so if the consumer were to consume $\tinyAmount$ more,
the marginal utility of that extra consumption would be \textit{below}
the (discounted, expected) marginal utility of saving, and thus the
consumer would engage in positive saving and the constraint would no
longer be binding.  Thus the level of $m_{T-1}$ at which the
constraint stops binding is:\footnote{The logic here repeats an insight from \cite{deatonLiqConstr}.}
\begin{equation}\begin{gathered}\begin{aligned}
      \uFunc^{c}(m_{T-1})  & = \vFunc^{a}_{\EndStpLst}(0)  \nonumber \\
      m_{T-1}  & = (\vFunc^{a}_{\EndStpLst}(0))^{(-1/\CRRA)}  \nonumber
      \\        & = \cFunc_{\EndStpLst}(0). \label{eq:LCbindsTm1}
    \end{aligned}\end{gathered}\end{equation}

\hypertarget{cVScCon}{}
\begin{figure}
  \includegraphics[width=6in]{./Figures/cVScCon}
  \caption{Constrained (solid) and Unconstrained (dashed) Consumption}
  \label{fig:cVScCon}
\end{figure}

The constrained problem is solved in section ``Artifical Borrowing Constraint''
of the notebook, where the variable
\texttt{constrained} is set to be a boolean type object. If the value of \texttt{constrained}
is true, then the constraint is binding and their consumption behavior is computed to match
\eqref{eq:LiqCons}. The resulting consumption rule is shown in Figure \ref{fig:cVScCon}. For comparison purposes,
the approximate consumption rule from Figure \ref{fig:cVScCon} is
reproduced here as the solid line; this is accomplished by setting the boolean value
of \texttt{constrained} to false.

The presence of the liquidity
constraint requires three changes to the procedures outlined above:
\begin{enumerate}
\item We redefine
  $\hEndMin_{\EndStp}$, which now is the PDV of receiving
  $\TranShkEmp_{t+1}=\TranShkEmpMin$ next period and
  $\TranShkEmp_{t+n}=0~\forall~n>1$ -- that is, the pessimist believes he
  will receive nothing beyond period $t+1$
\item We augment the end-of-period \texttt{aVec} with zero and with a point with a small positive value so that the generated
  {\mVec} will the binding point $\mNrm^{\#}$ and a point just above it (so that we can better capture the curvature
  around that point)
\item We redefine the optimal consumption rule as
  in equation (\ref{eq:LiqCons}).  This ensures that the
  liquidity-constrained `realist' will consume more than the redefined
  `pessimist,' so that we will have $\koppa$ still between $0$ and $1$
  and the `method of moderation' will proceed smoothly.
\end{enumerate}

As expected, the
liquidity constraint only causes a divergence between the two
functions at the point where the optimal unconstrained consumption
rule runs into the 45 degree line.

\hypertarget{Recursions}{}
\section{Recursion}\label{sec:recursion}
\hypertarget{Theory}{}
\subsection{Theory}
Before we solve for periods earlier than $T-1$, we assume for
convenience that in each such period a liquidity constraint exists of
the kind discussed above, preventing $c$ from exceeding $m$. This
simplifies things a bit because now we can always consider an
\texttt{aVec} that starts with zero as its smallest element.

Recall now equations~(\ref{eq:vEndPrimeTm1}) and (\ref{eq:upEqbetaOp}):
\begin{equation*}\begin{gathered}\begin{aligned}
      \vPEndStp(a_{t})  & = \Ex_{\BegStp}[\DiscFac \Rfree \PermGroFac_{t+1}^{-\CRRA}
      \uFunc^{c}(\cFunc_{t+1}(\RNrm_{t+1} a_{t}+{\TranShkEmp}_{t+1}))]
      \\\uFunc^{c}(c_{t})   & = \vEnd^{a}(m_{t}-c_{t}).
    \end{aligned}\end{gathered}\end{equation*}
Assuming that the problem has been solved up to period $t+1$ (and thus
assuming that we have an approximated $\Alt{\cFunc}_{t+1}(m_{t+1})$), our solution method essentially
involves using these two equations in succession to work back
progressively from period $T-1$ to the beginning of life.  Stated
generally, the method is as follows.  (Here, we use the original, rather than the ``refined,'' method for
constructing consumption functions; the generalization of the algorithm below to use the refined method presents
no difficulties.)

\begin{enumerate}
  \ifthenelse{\boolean{MyNotes}}{\marginpar{\tiny Point out that we
      are defining $\vEnd^{a}$ here by the literal summation
      operation \eqref{eq:vEndeq}.}}{}

\item For the grid of values $a_{t,i}$ in \texttt{aVec\_eee}, numerically calculate the values
  of $\cFunc_{\overline{t}}(a_{t,i})$ and $\cFunc_{\overline{t}}^{a}(a_{t,i})$,
  \begin{verbatimwrite}{./Equations/vEndeq.tex}
    \begin{equation}\begin{gathered}\begin{aligned}
          \cFunc_{\overline{t},i}  & = \left(\vEnd^{a}(a_{t,i})\right)^{-1/\CRRA},
          \\                             & = \left(\DiscFac \Ex_{\BegStp} \left[\Rfree \PermGroFac_{t+1}^{-\CRRA}(\grave{\cFunc}_{t+1}(\RNrm_{t+1} a_{t,i} +      {\TranShkEmp}_{t+1}))^{-\CRRA}\right]\right)^{-1/\CRRA}, \label{eq:vEndeq}
          \MPCMatch{\\        \cFunc^{a}_{\overline{t},i}  & = -(1/\CRRA)\left(\vEnd^{a}(a_{t,i})\right)^{-1-1/\CRRA} \vEnd^{a{a}}(\aNrm_{t,i}),}{}
        \end{aligned}\end{gathered}\end{equation}
  \end{verbatimwrite}
      \begin{equation}\begin{gathered}\begin{aligned}
          \cFunc_{\overline{t},i}  & = \left(\vEndStp^{{a}}({a}_{t,i})\right)^{-1/\CRRA},
          \\                             & = \left(\DiscFac \Ex_{\BegStep} \left[\Rfree \PermGroFac_{\prd+1}^{-\CRRA}(\grave{\cFunc}_{\prd+1}(\RNrm_{\prd+1} {a}_{t,i} +      {\TranShkEmp}_{\prd+1}))^{-\CRRA}\right]\right)^{-1/\CRRA}, \label{eq:vEndeq}
          \MPCMatch{\\        \cFunc^{a}_{\overline{t},i}  & = -(1/\CRRA)\left(\vEndStp^{{a}}({a}_{t,i})\right)^{-1-1/\CRRA} \vEndStp^{{a}{a}}(\aNrm_{t,i}),}{}
        \end{aligned}\end{gathered}\end{equation}
  

  generating vectors of values $\vctr{\cFunc}_{t}$\MPCMatch{ and $\vctr{\cFunc}^{a}_{\overline{t}}$.}{.}

\item Construct a corresponding vector of values of $\vctr{m}_{t}=\vctr{\cNrm}_{t}+\vctr{\aNrm}_{t}$\MPCMatch{; similarly construct a corresponding list of MPC's $\vctr{\MPC}_{t}$ using equation \eqref{eq:MPCfromMPTHC}.}{.}

\item Construct a corresponding vector $\vctr{\mu_{t}}$, the levels\MPCMatch{ and first derivatives}{} of $\vctr{\koppa}_{t}$, and the levels\MPCMatch{ and first derivatives}{} of $\vctr{\chi}_{t}$.

\item Construct an interpolating approximation $\Alt{\chi}_{t}$ that\MPCMatch{ smoothly matches both the level and the slope}{the level} at those points.

\item If we are to approximate the value function, construct a corresponding list of values of $\vctr{\vFunc}_{t}$, the levels\MPCMatch{ and first derivatives of $\vctr{\Koppa}_{t}$,}{,} and the levels\MPCMatch{ and first derivatives}{} of $\hat{\vctr{\Chi}}_{t}$; and construct an interpolating approximation function $\hat{\Chi}_{t}$ that matches those points.
\end{enumerate}

With $\Alt{\chi}_{t}$ in hand, our approximate consumption function
is computed directly from the appropriate substitutions in \eqref{eq:cFuncHi}
and related equations.  With this consumption
rule in hand, we can continue the backwards recursion to period $t-1$
and so on back to the beginning of life.

Note that this loop does not contain an item for constructing $\hat{\vFunc}_{t}^{a}(m_{t})$. This is because with $\Alt{\Hi{\cFunc}}_{t}(m_{t})$ in hand, we simply \textit{define} $\hat{\vFunc}^{m}_{t}(m_{t}) = \uFunc^{c}(\Alt{\Hi{\cFunc}}_{t}(m_{t}))$ so there is no need to construct interpolating approximations - the function arises `free' (or nearly so) from our constructed $\Alt{\Hi{\cFunc}}_{t}(m_{t})$ via the usual envelope result (cf.\ \eqref{eq:envelope}).

\subsection{Program Structure}

In section ``Solve for $c_t(m_t)$ in Multiple Periods,'' the natural and artificial borrowing constraints are combined with the endogenous gridpoints method to approximate the optimal consumption function for a specific period. Then, this function is used to compute the approximated consumption in the previous period, and this process is repeated for the number of periods specified by \texttt{T}, as explained earlier.

The essential structure of the program is a loop that iteratively solves for consumption functions by working backward from an assumed final period, using the dictionary \texttt{cFunc\_life} to store the interpolated consumption functions up to the beginning period. Consumption in a given period is utilized to determine the endogenous gridpoints for the preceding period. This is the sense in which the computation of optimal consumption is done recursively.

For a realistic life cycle problem, it would also be necessary at a
minimum to calibrate a nonconstant path of expected income growth over the
lifetime that matches the empirical profile; allowing for such
a calibration is the reason we have included the $\{\PermGroFac\}_{t}^{T}$
vector in our computational specification of the problem.

\hypertarget{Results}{}
\subsection{Results}

As suggested, the code creates the relevant $\Alt{\cFunc}_{t}(m_{t})$
functions for any period in the horizon specified by the variable \texttt{T}, at the given values of $m$.
Figure \ref{fig:PlotCFuncsConverge} shows
$\Alt{\cFunc}_{T-n}(m)$ for $n=\{20,15,10,5,1\}$.  At least one
feature of this figure is encouraging: the consumption functions
converge as the horizon extends, something that \cite{BufferStockTheory}
shows must be true under certain parametric conditions that are
satisfied by the baseline parameter values being used here.


\hypertarget{PlotCFuncsConverge}{}
\begin{figure}
  \includegraphics[width=6in]{./Figures/PlotCFuncsConverge}
  \caption{Converging $\Alt{\cFunc}_{T-n}({\mNrm})$ Functions as $n$ Increases}
  \label{fig:PlotCFuncsConverge}
\end{figure}

% Habits go here

\hypertarget{Multiple-Control-Variables}{}
\section{Multiple Control Variables}

We now consider how to solve problems with multiple control variables.
(To reduce notational complexity, in this section we set $\PermGroFac_{t}=1~\forall~t$.)

\subsection{Theory}\label{subsec:MCTheory}
The new control variable that the consumer can now choose is the portion of the portfolio to invest in risky assets.  Designating the gross return on the risky asset as $\Risky_{t+1}$, and using $\varsigma_{t}$ to represent the proportion of the portfolio invested in this asset before the return is realized after the beginning of $t+1$, corresponding to an assumption that the consumer cannot be `net short' and cannot issue net equity), the overall return on the consumer's portfolio between $t$ and $t+1$ will be:
\begin{verbatimwrite}{./Equations/Rport.tex}
  \begin{equation}\begin{gathered}\begin{aligned}
        \Rport_{t+1}  & = \Rfree(1-\varsigma_{t}) + \Risky_{t+1}\varsigma_{t} \label{eq:return1}
        \\               & = \Rfree + (\Risky_{t+1}-\Rfree) \varsigma_{t} %\label{eq:return2}
      \end{aligned}\end{gathered}\end{equation}
\end{verbatimwrite}
  \begin{equation}\begin{gathered}\begin{aligned}
        \Rport_{t+1}  & = \Rfree(1-\stigma_{t}) + \Risky_{t+1}\stigma_{t} \label{eq:return1}
        \\               & = \Rfree + (\Risky_{t+1}-\Rfree) \stigma_{t} %\label{eq:return2}
      \end{aligned}\end{gathered}\end{equation}
\unskip
and the maximization problem is
\begin{verbatimwrite}{./Equations/PortProb.tex}
  \begin{equation*}\begin{gathered}\begin{aligned}
        \vFunc_{t}(m_{t})  & = \max_{\{{c}_{t},\varsigma_{t}\}}   ~~ \uFunc(c_{t}) +  \DiscFac
        \Ex_{t}[{\vFunc}_{t+1}(m_{t+1})]
        \\      & \text{s.t.} \nonumber
        \\      \Rport_{t+1}  & = \Rfree + (\Risky_{t+1}-\Rfree) \varsigma_{t}
        \\      m_{t+1}  & = (m_{t}-c_{t})\Rport_{t+1} + \TranShkEmp_{t+1}
        \\  0       \leq & \varsigma_{t}  \leq 1, \label{eq:noshorts}
      \end{aligned}\end{gathered}\end{equation*}
\end{verbatimwrite}
  \begin{equation*}\begin{gathered}\begin{aligned}
        \vFunc_{t}(m_{t})  & = \max_{\{{c}_{t},\stigma_{t}\}}   ~~ \uFunc(c_{t}) +  \DiscFac
        \ExEndStg[{\vFunc}_{t+1}(m_{t+1})]
        \\      & \text{s.t.} \nonumber
        \\      \Rport_{t+1}  & = \Rfree + (\Risky_{t+1}-\Rfree) \stigma_{t}
        \\      m_{t+1}  & = (m_{t}-c_{t})\Rport_{t+1} + \tranShkEmp_{t+1}
        \\  0       \leq & \stigma_{t}  \leq 1, \label{eq:noshorts}
      \end{aligned}\end{gathered}\end{equation*}
\unskip
or, more compactly,
\begin{equation*}\begin{gathered}\begin{aligned}
      \vFunc_{t}(m_{t})  & = \max_{\{\cFunc_{t},\varsigma_{t}\}} ~~  \uFunc(c_{t}) +  \Ex_{t}[\DiscFac \vFunc_{t+1}((m_{t}-c_{t}){\Rport}_{t+1} +        {\TranShkEmp}_{t+1})]
      \\                       & \text{s.t.} \nonumber
      \\ 0 \leq & \varsigma_{t} \leq 1
      .
    \end{aligned}\end{gathered}\end{equation*}
The first order condition with respect to $c_{t}$ is almost identical
to that in the single-control problem, equation (\ref{eq:upceqEvtp1}),
with the only difference being that the nonstochastic interest factor
$\Rfree$ is now replaced by ${\Rport}_{t+1}$,
\begin{verbatimwrite}{./Equations/valfuncFOCRtilde.tex}
  \begin{equation}\begin{gathered}\begin{aligned}
        \uFunc^{c}(c_{t})  & = \DiscFac \Ex_{t} [{\Rport}_{t+1} \vFunc^{m}_{t+1}(m_{t+1})] \label{eq:valfuncFOCRtilde},
      \end{aligned}\end{gathered}\end{equation}
\end{verbatimwrite}
  \begin{equation}\begin{gathered}\begin{aligned}
        \uFunc^{c}(c_{t})  & = \DiscFac \ExEndStg [{\Rport}_{t+1} \vFunc^{m}_{t+1}(m_{t+1})] \label{eq:valfuncFOCRtilde},
      \end{aligned}\end{gathered}\end{equation}
\unskip
and the Envelope theorem derivation remains the same,
yielding the Euler equation for consumption
\begin{verbatimwrite}{./Equations/EulercRiskyR.tex}
  \begin{equation}\begin{gathered}\begin{aligned}
        \uFunc^{c}(c_{t})  & = \Ex_{t}[\DiscFac {\Rport}_{t+1} \uFunc^{c}(c_{t+1})]. \label{eq:EulercRiskyR}
      \end{aligned}\end{gathered}\end{equation}
\end{verbatimwrite}
  \begin{equation}\begin{gathered}\begin{aligned}
        \uFunc^{c}(c_{t})  & = \ExEndStep[\DiscFac {\Rport}_{t+1} \uFunc^{c}(c_{t+1})]. \label{eq:EulercRiskyR}
      \end{aligned}\end{gathered}\end{equation}
\unskip

The first order condition with respect to the risky portfolio share is
\begin{verbatimwrite}{./Equations/FOCw.tex}
  \begin{equation}\begin{gathered}\begin{aligned}
        0  & = \Ex_{t}[{\vFunc}_{\MidStpNxt}^{m}(m_{t+1})(\Risky_{t+1}-\Rfree){a}_{t}] \notag
        \\         & = a_{t}\Ex_{t}\left[\uFunc^{c}\left(\cFunc_{t+1}(m_{t+1})\right)(\Risky_{t+1}-\Rfree)\right] \label{eq:FOCw}.
      \end{aligned}\end{gathered}\end{equation}
\end{verbatimwrite}
  \begin{equation}\begin{gathered}\begin{aligned}
        0  & = \ExEndStep[{\vFunc}_{\MidStpNxt}^{{m}}({m}_{t+1})(\Risky_{t+1}-\Rfree){a}_{t}] \notag
        \\         & = \ExEndStep\left[\uFunc^{{c}}\left(\cFunc_{t+1}({m}_{t+1})\right)(\Risky_{t+1}-\Rfree)\right]{a}_{t}
        \\         & = \ExEndStep\left[\uFunc^{{c}}\left(\cFunc_{t+1}({m}_{t+1})\right)(\Risky_{t+1}-\Rfree)\right], \label{eq:FOCw}
      \end{aligned}\end{gathered}\end{equation}
\unskip

As before, it will be useful to define $\vEnd$ as a function that
yields the expected $t+1$ value of ending period $t$ in a given state.
However, now that there are two control variables, the expectation
must be defined as a function of the chosen values of both of those
variables, because expected end-of-period value will depend not just
on how much the agent saves, but also on how the saved assets are
allocated between the risky and riskless assets.  Thus we define
\begin{equation*}\begin{gathered}\begin{aligned}
      \vEnd(a_{t},\varsigma_{t})  & = \Ex_{t}[\DiscFac \vFunc_{t+1}(m_{t+1})]
    \end{aligned}\end{gathered}\end{equation*}
which has derivatives
\begin{equation}\begin{gathered}\begin{aligned}
      \vEnd^a  & = \Ex_{t}[\DiscFac {\Rport}_{t+1}\vFunc_{t+1}^{m}(m_{t+1})] = \Ex_{t}[\DiscFac {\Rport}_{t+1}{\uFunc}_{t+1}^{c}(\cFunc_{t+1}(m_{t+1}))]
      \\      \vEnd^{\varsigma}  & = \Ex_{t}[\DiscFac (\Risky_{t+1}-\Rfree){\vFunc}_{t+1}^{m}(m_{t+1})  ]a_{t} = \Ex_{t}[\DiscFac (\Risky_{t+1}-\Rfree){\uFunc}_{t+1}^{c}(\cFunc_{t+1}(m_{t+1}))  ]a_{t} \notag
    \end{aligned}\end{gathered}\end{equation}
implying that the first order conditions (\ref{eq:EulercRiskyR}) and
(\ref{eq:FOCw}) can be rewritten
\begin{equation}\begin{gathered}\begin{aligned}
      \uFunc^{c}(c_{t})  & = \vEnd^{a}(m_{t}-c_{t},\varsigma_{t})
      \\      0  & = \vFunc^{\varsigma}_{\overline{t}}(a_{t},\varsigma_{t}).
    \end{aligned}\end{gathered}\end{equation}

\subsection{Application}\label{subsec:MCApplication}

Our first step is to specify the stochastic process for $\Risky_{t+1}$.
We follow the common practice of assuming that returns are
lognormally distributed, $\log \Risky \sim
\mathcal{N}(\eprem+\rfree-\sigma^{2}_{\eprem}/2,\sigma^{2}_{\eprem})$ where $\eprem$ is the equity premium
over the returns $\rfree$ available on the riskless asset.\footnote{This guarantees that $\Ex[\Risky] = \EPrem$ is invariant to the choice of $\sigma^{2}_{\eprem}$; see \handoutM{LogELogNorm}.}

As with labor income uncertainty, it is necessary to discretize the
rate-of-return risk in order to have a problem that is soluble in a
reasonable amount of time.  We follow the same procedure as for labor
income uncertainty, generating a set of $n_{\risky}$ equiprobable shocks to the
rate of return; in a slight abuse of notation, we will designate
the portfolio-weighted return (contingent on the
chosen portfolio share in equity, and potentially contingent on any other
aspect of the consumer's problem) simply as $\Rport_{i,j}$ (where dependence
on $i$ is allowed to permit the possibility of nonzero correlation
between the return on the risky asset and the shock to labor income (for example,
in recessions the stock market falls and labor income also declines).

The direct expressions for the derivatives of $\vEnd$ are
\begin{equation}\begin{gathered}\begin{aligned}
      \vEnd^{a}(a_{t},\varsigma_{t})  & = \DiscFac \left(\frac{1}{n_{\risky} n_{\TranShkEmp}}\right)\sum_{i=1}^{n_{\TranShkEmp}}\sum_{j=1}^{n_{\risky} }\Rport_{i,j} \left(\cFunc_{t+1}(\Rport_{i,j}a_{t}+\TranShkEmp_{i})\right)^{-\CRRA}
      \\      \vEnd^{\varsigma}(a_{t},\varsigma_{t})  & = \DiscFac \left(\frac{1}{n_{\risky} n_{\TranShkEmp}}\right)\sum_{i=1}^{n_{\TranShkEmp}}\sum_{j=1}^{n_{\risky} }(\Risky_{i,j}-\Rfree)\left(\cFunc_{t+1}(\Rport_{i,j}a_{t}+\TranShkEmp_{i})\right)^{-\CRRA}.
    \end{aligned}\end{gathered}\end{equation}

Writing these equations out explicitly makes a problem very
apparent: For every different combination of $\{{a}_{t},\varsigma_{t}\}$
that the routine wishes to consider, it must perform two
double-summations of $n_{\risky} \times n$ terms.  Once again, there is an
inefficiency if it must perform these same calculations many times
for the same or nearby values of $\{{a}_{t},\varsigma_{t}\}$, and again
the solution is to construct an approximation to the derivatives of
the $\vEnd$ function.

Details of the construction of the interpolating approximation are
given below; assume for the moment that we have the approximations
$\hat{\vFunc}_{\EndStp}^{a}$ and $\hat{\vFunc}_{\EndStp}^{\varsigma}$ in
hand and we want to proceed.  As noted above, nonlinear equation
solvers can find the
solution to a set of simultaneous equations.  Thus we could ask
Python to solve
\begin{equation}\begin{gathered}\begin{aligned}
      c_{t}^{-\CRRA}  & = \hat{\vFunc}^{a}_{\overline{t}}(m_{t}-c_{t},\varsigma_{t}) %\label{eq:FOCwrtcMultContr}
      \\      0  & = \hat{\vFunc}^{\varsigma}_{\overline{t}}(m_{t}-c_{t},\varsigma_{t}) \label{eq:FOCwrtw}
    \end{aligned}\end{gathered}\end{equation}
simultaneously for $\cNrm$ and $\varsigma$ at the set of potential $m_{t}$ values defined in {\mVec}. However, multidimensional constrained
maximization problems are difficult and sometimes quite slow to
solve.  There is a better way.  Define the problem
\providecommand{\Opt}{}
\renewcommand{\Opt}{\tilde}
\providecommand{\vOpt}{}
\renewcommand{\vOpt}{\overset{*}{\vFunc}}
\begin{equation}\begin{gathered}\begin{aligned}
      \Opt{\vFunc}_{\overline{t}}(a_{t})  & = \max_{\varsigma_{t}} ~~  \vEnd(a_{t},\varsigma_{t})
      \\      & \text{s.t.} \nonumber
      \\      0 \leq & \varsigma_{t} \leq 1
    \end{aligned}\end{gathered}\end{equation}
where the tilde over $\Opt{\vFunc}(a)$ indicates that this is the $\vFunc$ that has been optimized with
respect to all of the arguments other than the one still present
($a_{t}$).  We solve this problem for the set of gridpoints in
\texttt{aVec} and use the results to construct the interpolating
function $\Alt{\Opt{\vFunc}}_{t}^{a}(a_{t})$.\footnote{A faster solution
  could be obtained by, for each element in \texttt{aVec}, computing
  $\vEnd^{\varsigma}(m_{t}-c_{t},\varsigma)$ of a grid of
  values of $\varsigma$, and then using an approximating interpolating
  function (rather than the full expectation) in the \texttt{FindRoot}
  command.  The associated speed improvement is fairly modest,
  however, so this route was not pursued.}  With this function in
hand, we can use the first order condition from the single-control
problem
\begin{equation*}\begin{gathered}\begin{aligned}
      c_{t}^{-\CRRA}  & = \Alt{\Opt{\vFunc}}_{t}^{a}(m_{t}-c_{t})
    \end{aligned}\end{gathered}\end{equation*}
to solve for the optimal level of consumption as a function of
$m_{t}$ using the endogenous gridpoints method described above.  Thus we have transformed the multidimensional optimization
problem into a sequence of two simple optimization problems.

Note the parallel between this trick and the fundamental insight of
dynamic programming: Dynamic programming techniques transform a
multi-period (or infinite-period) optimization problem into a sequence
of two-period optimization problems which are individually much easier
to solve; we have done the same thing here, but with multiple dimensions
of controls rather than multiple periods.

\hypertarget{Implementation}{}
\subsection{Implementation}

Following the discussion from section \ref{subsec:MCTheory}, to provide a numerical solution to the problem
with multiple control variables, we must define expressions that capture the expected marginal value of end-of-period
assets with respect to the level of assets and the share invested in risky assets. This is addressed in ``Multiple Control Variables.'' Inheriting the structure of Python, we establish a new subclass of \texttt{gothic\_class} called \texttt{GothicMC}. This subclass preserves the fundamental structure required to resolve the original problem while adding new methods that capture the previously mentioned points. The essential functions in this new class are found in the final four functions that account for the expected marginal value functions with respect to each of the control variables, both for the terminal period and all earlier periods.

Having the \texttt{GothicMC} subclass available, we can proceed with implementing the steps laid out in section \ref{subsec:MCApplication} to solve the problem at hand. Initially, the two distributions that capture the uncertainty faced by consumers in this scenario are discretized. Subsequently, the \texttt{GothicMC} class is invoked with the requisite arguments to create an instance that includes the necessary functions to depict the first-order conditions of the consumer's problem. Following that, an improved grid of end-of-period assets is established.

Here is where we can see how the approach described in section \ref{subsec:MCApplication} is reflected in the code.  For the terminal period, the optimal share of risky assets is determined for each point in \texttt{aVec\_eee}, and then the endogenous gridpoints method is employed to compute the optimal consumption level given that the share in the risky asset has been chosen optimally. It's worth noting that this solution takes into account the possibility of a binding artificial borrowing constraint. Lastly, the interpolation process is executed for both the optimal consumption function and the optimal share of the portfolio in risky assets. These values are stored in their respective dictionaries (\texttt{mGridPort\_life}, \texttt{cGridPort\_life}, and \texttt{varsigmaGrid\_life}) and utilized to conduct the recursive process outlined in section \ref{sec:recursion}, thus yielding the numerical solution for all earlier periods.

\subsection{Results}

Figure~\ref{fig:PlotctMultContr} plots the first-period consumption
function generated by the program; qualitatively it does not look much
different from the consumption functions generated by the program
without portfolio choice.  Figure~\ref{fig:PlotRiskySharetOfat} plots the
optimal portfolio share as a function of the level of assets.  This
figure exhibits several interesting features.  First, even with a
coefficient of relative risk aversion of 6, an equity premium of only
4 percent, and an annual standard deviation in equity returns of 15
percent, the optimal choice is for the agent
to invest a proportion 1 (100 percent) of the portfolio in stocks (instead of the safe bank account with riskless return $\Rfree$) is
at values of $a_{t}$ less than about 2.  Second, the
proportion of the portfolio kept in stocks is \textit{declining} in the
level of wealth - i.e., the poor should hold all of their meager
assets in stocks, while the rich should be cautious, holding more of
their wealth in safe bank deposits and less in stocks.  This
seemingly bizarre (and highly counterfactual) prediction reflects the
nature of the risks the consumer faces.  Those consumers who are poor
in measured financial wealth are likely to derive a high proportion of
future consumption from their labor income.  Since by assumption labor
income risk is uncorrelated with rate-of-return risk, the covariance
between their future consumption and future stock returns is
relatively low.  By contrast, persons with relatively large wealth
will be paying for a large proportion of future consumption out of that
wealth, and hence if they invest too much of it in stocks their consumption
will have a high covariance with stock returns.  Consequently, they
reduce that correlation by holding some of their wealth in the
riskless form.

\hypertarget{PlotctMultContr}{}
\begin{figure}
  \includegraphics[width=6in]{./Figures/PlotctMultContr}
  \caption{$\cFunc(m_{1})$ With Portfolio Choice}
  \label{fig:PlotctMultContr}
\end{figure}

\hypertarget{PlotRiskySharetOfat}{}
\begin{figure}
  \includegraphics[width=6in]{./Figures/PlotRiskySharetOfat}
  \caption{Portfolio Share in Risky Assets in First Period $\varsigma(a)$}
  \label{fig:PlotRiskySharetOfat}
\end{figure}

\hypertarget{The-Infinite-Horizon}{}
\section{The-Infinite-Horizon}

All of the solution methods presented so far have involved
period-by-period iteration from an assumed last period of life, as is
appropriate for life cycle problems.  However, if the parameter values
for the problem satisfy certain conditions (detailed in
\cite{BufferStockTheory}), the consumption rules (and the rest of
the problem) will converge to a fixed rule as the horizon (remaining
lifetime) gets large, as illustrated in
Figure~\ref{fig:PlotCFuncsConverge}.  Furthermore,
Deaton~\citeyearpar{deatonLiqConstr},
Carroll~\citeyearpar{carroll:brookings,carrollBSLCPIH} and others
have argued that the `buffer-stock' saving behavior that emerges under
some further restrictions on parameter values is a good approximation
of the behavior of typical consumers over much of the lifetime.
Methods for finding the converged functions are therefore of interest,
and are dealt with in this section.

Of course, the simplest such method is to solve the problem as
specified above for a large number of periods.  This is feasible, but
there are much faster methods.

\subsection{Convergence}

In solving an infinite-horizon problem, it is necessary to have some
metric that determines when to stop because a solution that is `good
enough' has been found.

A natural metric is defined by the unique `target' level of wealth that \cite{BufferStockTheory} proves
will exist in problems of this kind \href{https://llorracc.github.io/BufferStockTheory#GICNrm}{under certain conditions}: The $\mTrgNrm$ such that
\begin{equation}
  \Ex_t [{\mNrm}_{t+1}/\mNrm_t] = 1 \mbox{~if~} \mNrm_t = \mTrgNrm  \label{eq:mTrgNrmet}
\end{equation}
where the accent is meant to signify that this is the value
that other $\mNrm$'s `point to.'

Given a consumption rule $\cFunc(\mNrm)$ it is straightforward to find
the corresponding $\mTrgNrm$.  So for our problem, a solution is declared
to have converged if the following criterion is met:
$\left|\mTrgNrm_{t+1}-\mTrgNrm_{t}\right| < \epsilon$, where $\epsilon$ is
a very small number and measures our degree of convergence tolerance.

Similar criteria can obviously be specified for other problems.
However, it is always wise to plot successive function differences and
to experiment a bit with convergence criteria to verify that the
function has converged for all practical purposes.

\begin{comment} % at suggestion of WW, this section was removed as unnecessary for the current model, which solves for the converged rule very fast
  \subsection{The Last Period}

  For the last period of a finite-horizon lifetime, in the absence of a
  bequest motive it is obvious that the optimal policy is to spend
  everything.  However, in an infinite-horizon problem there is no last
  period, and the policy of spending everything is obviously very far
  from optimal.  Generally speaking, it is much better to start off with
  a `last-period' consumption rule and value function equal to those
  corresponding to the infinite-horizon solution to the perfect
  foresight problem (assuming such a solution is known).

  For the perfect foresight infinite horizon consumption problem,
  the solution is
  \begin{equation}\begin{gathered}\begin{aligned}
        \bar{\cFunc}(m_{t})  & = \overbrace{(1-\Rfree^{-1}(\Rfree
          \DiscFac)^{1/\CRRA})}^{\equiv
          \underline{\MPC}}\left[{m}_{t}-1+\left(\frac{1}{1-1/\Rfree}\right)\right]
        \label{eq:pfinfhorc}
      \end{aligned}\end{gathered}\end{equation}
  where $\underline{\MPC}$ is the MPC in the
  infinite-horizon perfect foresight problem.  In our baseline problem,
  we set $\PermGroFac = \pLvl_{t} = 1$.  It is straightforward to show that the
  infinite-horizon perfect-foresight value function and marginal value
  function are given by
  \begin{equation}\begin{gathered}\begin{aligned}
        \bar{\vFunc}(m_{t})
        & =                                 \left(\frac{\bar{\cFunc}(m_{t})^{1-\CRRA}}{
            (1-\CRRA)\underline{\MPC} }\right)
        \\  \bar{\vFunc}^{m}(m_{t})  & =       (\bar{\cFunc}(m_{t}))^{-\CRRA}
        \\  \Opt{\vFunc}^{m}(a_{t})  & = \DiscFac \Rfree \PermGroFac_{t+1}^{-\CRRA} \bar{\vFunc}^{m}(\RNrm_{t+1} a_{t}+1).
      \end{aligned}\end{gathered}\end{equation}

  % WW delete the text on 2011-06-21 because we no longer start from the infinite horizon perfect foresight solution.
  % If we choose to pursue that starting point, we need to derive the optimist's and pessimist's consumption function,
  % when the last period is given by the infinite horizon perfect-foresight solution. That will change the program significantly.
  % In our case, with \epsilon being 10^(-4), iteration requires only 51 periods, and 0.032 minutes.
\end{comment}

\begin{comment}% At suggestion of WW this section was deleted because the technique is obvious and can be captured by the footnote that has been added
  \subsection{Coarse Then Fine \texttt{aVec} }

  The speed of each iteration is directly proportional to the number
  of gridpoints at which the problem must be solved.  Therefore
  reducing the number of points in \texttt{aVec} can increase
  the speed of solution greatly.  Of course, this also decreases the
  accuracy of the solution.  However, once the converged solution is
  obtained for a coarse \texttt{aVec}, the density of the grid
  can be increased and iteration can continue until a converged
  solution is found for the finer \texttt{aVec}.
  % WW delete the text on 2011-06-21 because we no longer need a finer \texttt{aVec}. I add a footnote in next subsection instead.

  \subsection{Coarse then Fine \texttt{$\TranShkEmp$Vec}}

  The speed of solution is roughly proportionate\footnote{It is also
    true that the speed of each iteration is directly proportional to
    the number of gridpoints in \texttt{aVec}, at which the problem must
    be solved. However given our method of moderation, now the problem
    could be solved very precisely based on five gridpoints only. Hence
    we do not pursue the process of ``Coarse then Fine \texttt{aVec}.''}
  to the number of points used in approximating the distribution of
  shocks.  At least 3 gridpoints should probably be used as an initial
  minimum, and my experience is that increasing the number of gridpoints
  beyond 7 generally yields only very small changes in the solution.  The program
  \texttt{multiperiodCon\_infhor.m}
  begins with three gridpoints, and then solves for successively finer
  \texttt{$\TranShkEmp$Vec}.
\end{comment}

}{} % end \MoM

\hypertarget{StructuralEstimation}{}
\section{Structural Estimation}\label{sec:StructEst}

This section describes how to use the methods developed above to
structurally estimate a life-cycle consumption model, following
closely the work of
\cite{cagettiWprofiles}.\footnote{Similar structural
  estimation exercises have been also performed by
  \cite{palumbo:medical} and \cite{gpLifecycle}.} The key idea of
structural estimation is to look for the parameter values (for the
time preference rate, relative risk aversion, or other parameters)
which lead to the best possible match between simulated and empirical
moments.  %(The code for the structural estimation is in the self-containedsubfolder \texttt{StructuralEstimation} in the Matlab and {\Mma} directories.)

\hypertarget{LifeCycleModel}{}
\subsection{Life Cycle Model}
\newcommand{\byage}{\hat}

Realistic calibration of a life cycle model needs to take into account a few things that we omitted from the bare-bones model described above. For example, the whole point of the life cycle model is that life is finite, so we need to include a realistic treatment of life expectancy; this is done easily enough, by assuming that utility accrues only if you live, so effectively the rising mortality rate with age is treated as an extra reason for discounting the future.  Similarly, we may want to capture the demographic evolution of the household (e.g., arrival and departure of kids).  A common way to handle that, too, is by modifying the discount factor (arrival of a kid might increase the total utility of the household by, say, 0.2, so if the `pure' rate of time preference were $1.0$ the `household-size-adjusted' discount factor might be 1.2.  We therefore modify the model presented above to allow age-varying discount factors that capture both mortality and family-size changes (we just adopt the factors used by \cite{cagettiWprofiles} directly), with the probability of remaining alive between $t$ and $t+n$ captured by $\Alive$ and with $\hat{\DiscFac}$ now reflecting all the age-varying discount factor adjustments (mortality, family-size, etc).  Using $\beth$ (the Hebrew cognate of $\beta$) for the `pure' time preference factor, the value function for the revised problem is
\begin{verbatimwrite}{./Equations/lifecyclemax.tex}
  \begin{equation}\begin{gathered}\begin{aligned}
\vFunc_{t}(\pLvl_{t},\mLvl_{t}) & =    \max_{\{\cFunc\}_{t}^{T}}~~ \uFunc(\cLvl_{t})+\Ex_{t}\left[\sum_{n=1}^{T-t}\beth^{n} \Alive_{t}^{t+n}\hat{\DiscFac}_{t}^{t+n} \uFunc(\cLvl_{t+n}) \right]   \label{eq:lifecyclemax}
\end{aligned}\end{gathered}  \end{equation}
\end{verbatimwrite}
  \begin{equation}\begin{gathered}\begin{aligned}
        \vFunc_{\prd}(\pLvl_{\prd},\mLvl_{\prd}) & =    \max_{\{\cFunc\}_{\prd}^{T}}~~ \uFunc(\cLvl_{\prd})+\ExEndPrd\left[\sum_{n=1}^{T-t} {\beth}^{n} \Alive_{\prd}^{t+n}\hat{\DiscFac}_{\prd}^{t+n} \uFunc(\cLvl_{t+n}) \right]   \label{eq:lifecyclemax}
      \end{aligned}\end{gathered}  \end{equation}
\unskip
subject to the constraints
\begin{verbatimwrite}{./Equations/dbc-with-permshk}
\begin{equation*}\begin{gathered}\begin{aligned}
      \aLvl_{t}  & = \mLvl_{t}-\cLvl_{t}
      \\      \pLvl_{t+1}  & = \PermGroFac_{t+1}\pLvl_{t}\Psi_{t+1}
      \\      \yLvl_{t+1}  & = \pLvl_{t+1}\TranShkEmp _{t+1}
      \\      \mLvl_{t+1}  & = \Rfree \aLvl_{t}+\yLvl_{t+1}
    \end{aligned}\end{gathered}\end{equation*}
\end{verbatimwrite}
  \begin{equation*}\begin{gathered}\begin{aligned}
        \aLvl_{\prd}  & = \mLvl_{\prd}-\cLvl_{\prd}
        \\      \pLvl_{\prd+1}  & = \PermGroFac_{\prd+1}\pLvl_{\prd}\Psi_{\prd+1}
        \\      \yLvl_{\prd+1}  & = \pLvl_{\prd+1}\TranShkEmp _{\prd+1}
        \\      \mLvl_{\prd+1}  & = \Rfree \aLvl_{\prd}+\yLvl_{\prd+1}
      \end{aligned}\end{gathered}\end{equation*}

where
\begin{verbatimwrite}{./Equations/subjectTo.tex}
  \begin{equation*}\begin{gathered}\begin{aligned}
        \Alive _{t}^{t+n} &:\text{probability to }\Alive\text{ive until age $t+n$ given alive at age $t$}
        \\      \hat{\DiscFac}_{t}^{t+n} &:\text{age-varying discount factor between ages $t$ and $t+n$}
        \\     \Psi_{t} &:\text{mean-one shock to permanent income}
        \\     \beth &:\text{time-invariant `pure' discount factor}
      \end{aligned}\end{gathered}\end{equation*}
\end{verbatimwrite}
  \begin{equation*}\begin{gathered}\begin{aligned}
        \Alive _{\prd}^{t+n} &:\text{probability to }\Alive\text{ive until age $t+n$ given alive at age $t$}
        \\      \hat{\DiscFac}_{\prd}^{t+n} &:\text{age-varying discount factor between ages $t$ and $t+n$}
        \\     \permShk_{\prd} &:\text{mean-one shock to permanent income}
        \\     \beth &:\text{time-invariant `pure' discount factor}
      \end{aligned}\end{gathered}\end{equation*}
\unskip
and all the other variables are defined as in section \ref{sec:the-problem}.

Households start life at age $s=25$ and live with probability 1 until retirement
($s=65$). Thereafter the survival probability shrinks every year and
agents are dead by $s=91$ as assumed by Cagetti. % Note that in addition to a typical time-invariant discount factor $\beth$, there is a time-varying discount factor $\hat{\DiscFac}_{s}$ in (\ref{eq:lifecyclemax}) which can be used to capture the effect of age-varying demographic variables (e.g.\ changes in family size).

\begin{verbatimwrite}{./Equations/shocks-for-lifecycle}
Transitory and permanent shocks are distributed as follows:
\begin{equation}\begin{gathered}\begin{aligned}
      \Xi_{s} & =
      \begin{cases}
        0\phantom{/\pZero} & \text{with probability $\pZero>0$} \\
        \TranShkEmp_{s}/\pZero      & \text{with probability $(1-\pZero)$, where $\log \TranShkEmp_{s}\thicksim \mathcal{N}(-\sigma_{\TranShkEmp}^{2}/2,\sigma_{\TranShkEmp}^{2})$}\\
      \end{cases}\\
      \log \PermShk_{s} &\thicksim \mathcal{N}(-\sigma_{\PermShk}^{2}/2,\sigma_{\PermShk}^{2})
    \end{aligned}\end{gathered}\end{equation}
where $\pZero$ is the probability of unemployment (and unemployment shocks are turned off after retirement).
\end{verbatimwrite}
  Transitory and permanent shocks are distributed as follows:
  \begin{equation}\begin{gathered}\begin{aligned}
        \Xi_{s} & =
        \begin{cases}
          0\phantom{/\pZero} & \text{with probability $\pZero>0$} \\
          \tranShkEmp_{s}/\pZero      & \text{with probability $(1-\pZero)$, where $\log \tranShkEmp_{s}\thicksim \Nrml(-\sigma_{\tranShkEmp}^{2}/2,\sigma_{\tranShkEmp}^{2})$}\\
        \end{cases}\\
        \log \permShk_{s} &\thicksim \Nrml(-\sigma_{\permShk}^{2}/2,\sigma_{\permShk}^{2})
      \end{aligned}\end{gathered}\end{equation}
  where $\pZero$ is the probability of unemployment (and unemployment shocks are turned off after retirement).


The parameter values for the shocks are taken from Carroll~\citeyearpar{carroll:brookings}, $\pZero=0.5/100$, $\sigma _{\TranShkEmp }=0.1$, and $\sigma_{\PermShk}=0.1$.\footnote{Note that $\sigma _{\TranShkEmp}=0.1$ is smaller than the estimate for college graduates estimated in
  Carroll and Samwick~\citeyearpar{carroll&samwick:nature} ($=0.197=\sqrt{0.039}$) which is used by Cagetti~\citeyearpar{cagettiWprofiles}. The reason for this choice is that Carroll and Samwick~\citeyearpar{carroll&samwick:nature} themselves argue that their estimate of $\sigma_{\TranShkEmp }$ is almost certainly increased by measurement error.} The income growth profile $\PermGroFac_{t}$ is from Carroll~\citeyearpar{carrollBSLCPIH} and the values of $\Alive_{t}$ and $\hat{\DiscFac}_{t}$ are obtained from Cagetti~\citeyearpar{cagettiWprofiles} (Figure \ref{fig:TimeVaryingParam}).\footnote{The income growth profile is the one used by Caroll for operatives. Cagetti computes the time-varying discount factor by educational groups using the methodology proposed by Attanasio et al.~\citeyearpar{AttanasioBanksMeghirWeber} and the survival probabilities from the 1995 Life Tables (National Center for Health Statistics 1998).} The interest rate is assumed to equal $1.03$. The model parameters are included in Table \ref{table:StrEstParams}.

\hypertarget{PlotTimeVaryingParam}{}
\begin{figure}[h]
  \includegraphics[width=6in]{./Figures/PlotTimeVaryingParam}
  \caption{Time Varying Parameters}
  \label{fig:TimeVaryingParam}
\end{figure}

\begin{table}[h]
  \caption{Parameter Values}\label{table:StrEstParams}
  \begin{center}
    \begin{tabular}{ccl}
      \hline\hline
      $\sigma _{\TranShkEmp}$    & $0.1$ & Carroll~\citeyearpar{carroll:brookings}
      \\ $\sigma _{\PermShk}$   & $0.1$ & Carroll~\citeyearpar{carroll:brookings}
      \\ $\pZero$           & $0.005$  & Carroll~\citeyearpar{carroll:brookings}
      \\ $\PermGroFac_{s}$        & figure \ref{fig:TimeVaryingParam} & Carroll~\citeyearpar{carrollBSLCPIH}
      \\ $\hat{\DiscFac}_{s},\Alive_{s}$ & figure \ref{fig:TimeVaryingParam} & Cagetti~\citeyearpar{cagettiWprofiles}
      \\$\Rfree$            & $1.03$  & Cagetti~\citeyearpar{cagettiWprofiles}\\
      \hline
    \end{tabular}
  \end{center}
\end{table}

The structural estimation of the parameters $\beth$ and $\CRRA$ is carried out using
the procedure specified in the following section, which is then implemented in
the \texttt{StructEstimation.py} file. This file consists of two main components. The
first section defines the objects required to execute the structural estimation procedure,
while the second section executes the procedure and various optional
experiments with their corresponding commands. The next section elaborates on the procedure
and its accompanying code implementation in greater detail.

\subsection{Estimation}

When economists say that they are performing ``structural estimation''
of a model like this, they mean that they have devised a
formal procedure for searching for values for the parameters $\beth$
and $\CRRA$ at which some measure of the model's outcome (like
``median wealth by age'') is as close as possible to an empirical measure
of the same thing. Here, we choose to match the median of the
wealth to permanent income ratio across 7 age groups, from age $26-30$
up to $56-60$.\footnote{\cite{cagettiWprofiles}
  matches wealth levels rather than wealth to income ratios. We
  believe it is more appropriate to match ratios both because the
  ratios are the state variable in the theory and because empirical
  moments for ratios of wealth to income are not influenced by the
  method used to remove the effects of inflation and productivity
  growth.} The choice of matching the medians rather the means is
motivated by the fact that the wealth distribution is much more
concentrated at the top than the model is capable of explaining using a single
set of parameter values.  This means that in practice one must pick
some portion of the population who one wants to match well; since the
model has little hope of capturing the behavior of Bill Gates, but
might conceivably match the behavior of Homer Simpson, we choose to
match medians rather than means.

As explained in section \ref{sec:normalization}, it is convenient to work with the normalized version of the model which can be written in Bellman form as:
\begin{verbatimwrite}{./Equations/LifeCycleMaxNormed.tex}
  \begin{equation*}\begin{gathered}\begin{aligned}
        \vFunc_{t}(m_{t})  & = \max_{{c}_{t}}~~~ \uFunc(c_{t})+\beth\Alive_{t+1}\hat{\DiscFac}_{t+1}
          \Ex_{t}[(\PermShk_{t+1}\PermGroFac_{t+1})^{1-\CRRA}\vFunc_{t+1}(m_{t+1})]   \\
        & \text{s.t.}   \nonumber \\
        a_{t}    & = m_{t}-c_{t} \nonumber
        \\      m_{t+1}  & = a_{t}\underbrace{\left(\frac{\Rfree}{\PermShk_{t+1}\PermGroFac_{t+1}}\right)}_{\equiv \RNrm_{t+1}}+ ~\TranShkEmp_{t+1}
      \end{aligned}\end{gathered}\end{equation*}
\end{verbatimwrite}
  \begin{equation*}\begin{gathered}\begin{aligned}
        \vFunc_{\prd}(m_{\prd})  & = \max_{{c}_{\prd}}~~~ \uFunc(c_{\prd})+\beth\Alive_{\prd+1}\hat{\DiscFac}_{\prd+1}
        \Ex_{\prd}[(\permShk_{\prd+1}\PermGroFac_{\prd+1})^{1-\CRRA}\vFunc_{\prd+1}(m_{\prd+1})]   \\
        & \text{s.t.}   \nonumber \\
        a_{\prd}    & = m_{\prd}-c_{\prd} \nonumber
        \\      m_{\prd+1}  & = a_{\prd}\underbrace{\left(\frac{\Rfree}{\permShk_{\prd+1}\PermGroFac_{\prd+1}}\right)}_{\equiv \RNrm_{\prd+1}}+ ~\tranShkEmp_{\prd+1}
      \end{aligned}\end{gathered}\end{equation*}
\unskip
with the first order condition:
\begin{verbatimwrite}{./Equations/FOCLifeCycle}
\begin{equation}\begin{gathered}\begin{aligned}
      \uFunc^{c}(c_{t}) & = \beth\Alive_{t+1}\hat{\DiscFac}_{t+1}\Rfree \Ex_{t}\left[\uFunc^{c}\left(\PermShk_{t+1}\PermGroFac_{t+1}\cFunc_{t+1}\left(a_{t}\RNrm_{t+1}+\TranShkEmp_{t+1}\right)\right)\right]\label{eq:FOCLifeCycle}
      .
    \end{aligned}\end{gathered}\end{equation}
\end{verbatimwrite}
  \begin{equation}\begin{gathered}\begin{aligned}
        \uFunc^{{c}}({c}_{\prd}) & = \beth\Alive_{\prd+1}\hat{\DiscFac}_{\prd+1}\Rfree \Ex_{\prd}\left[\uFunc^{{c}}\left(\permShk_{\prd+1}\PermGroFac_{\prd+1}\cFunc_{\prd+1}\left({a}_{\prd}\RNrm_{\prd+1}+\TranShkEmp_{\prd+1}\right)\right)\right]\label{eq:FOCLifeCycle}
        .
      \end{aligned}\end{gathered}\end{equation}


The first substantive step in this estimation procedure is
to solve for the consumption functions at each age. We need to
discretize the shock distribution and solve for the policy
functions by backward induction using equation (\ref{eq:FOCLifeCycle})
following the procedure in sections \ref{sec:NextToLast} and
\ref{sec:recursion}. The latter routine
is slightly complicated by the fact that we are considering a
life-cycle model and therefore the growth rate of permanent income,
the probability of death, the time-varying discount factor and the
distribution of shocks will be different across the years. We thus
must ensure that at each backward iteration the right parameter
values are used.

Correspondingly, the first part of the \texttt{StructEstimation.py} file begins by defining
the agent type by inheriting from the baseline agent type \texttt{IndShockConsumerType},
with the modification to include time-varying discount factors. Next, an instance of
this ``life-cycle'' consumer is created for the estimation procedure.
The number of periods for the life cycle of a given agent is set and, following Cagetti,
~\citeyearpar{cagettiWprofiles}, we
initialize the wealth to income ratio of agents at age $25$ by
randomly assigning the equal probability values to $0.17$, $0.50$ and
$0.83$. In particular, we
consider a population of agents at age 25 and follow their consumption
and wealth accumulation dynamics as they reach the age of $60$, using
the appropriate age-specific consumption functions and the age-varying
parameters. The simulated medians are obtained by taking the medians
of the wealth to income ratio of the $7$ age groups.

To complete the creation of the consumer type needed for the simulation,
a history of shocks is drawn for each agent across all periods by invoking the
\texttt{make\_shock\_history} function. This involves
discretizing the shock distribution for as many points as the number
of agents we want to simulate and then randomly permuting this
shock vector as many times as we need to simulate the model for. In this way,
we obtain a time varying shock for each agent. This is much more time efficient than
drawing at each time from the shock distribution a shock for each
agent, and also ensures a stable distribution of shocks across the
simulation periods even for a small number of agents. (Similarly, in
order to speed up the process, at each backward iteration we compute
the consumption function and other variables as a vector at once.)

With the age-varying consumption functions derived from the life-cycle agent,
we can proceed to generate simulated data and compute the corresponding medians.
Estimating the model involves comparing these simulated medians with empirical medians,
measuring the model's success by calculating the difference between the two.
However, before performing the necessary steps of solving and simulating the model to
generate simulated moments, it's important to note a difficulty in producing the
target moments using the available data.

Specifically, defining $\xi$ as the set of parameters
to be estimated (in the current case $\xi =\{\CRRA ,\beth\}$), we could search for
the parameter values which solve
\begin{verbatimwrite}{./Equations/naivePowell.tex}
  \begin{equation}
    \begin{gathered}
      \begin{aligned}
        \min_{\xi} \sum_{\tau=1}^{7} |\varsigma^{\tau} -\mathbf{s}^{\tau}(\xi)|  \label{eq:naivePowell}
      \end{aligned}
    \end{gathered}
  \end{equation}
\end{verbatimwrite}
  \begin{equation}
    \begin{gathered}
      \begin{aligned}
        \min_{\xi} \sum_{\tau=1}^{7} |\Shr^{\tau} -\mathbf{s}^{\tau}(\xi)|  \label{eq:naivePowell}
      \end{aligned}
    \end{gathered}
  \end{equation}
\unskip
where $\varsigma^{\tau }$ and $\mathbf{s}^{\tau}$ are respectively the empirical
and simulated medians of the wealth to permanent income ratio for age group $\tau$.
A drawback of proceeding in this way is that it treats the empirically
estimated medians as though they reflected perfect measurements of the
truth. Imagine, however, that one of the age groups happened to have
(in the consumer survey) four times as many data observations as
another age group; then we would expect the median to be more
precisely estimated for the age group with more observations; yet
\eqref{eq:naivePowell} assigns equal importance to a deviation between
the model and the data for all age groups.

We can get around this problem (and a variety of others) by instead minimizing a slightly more complex object:
\begin{verbatimwrite}{./Equations/StructEstim.tex}
  \begin{equation}
    \min_{\xi}\sum\limits_{i}^{N}\weight _{i}\left|\varsigma_{i}^{\tau }-\mathbf{s}^{\tau}(\xi )\right|\label{eq:StructEstim}
  \end{equation}
\end{verbatimwrite}
  \begin{equation}
    \min_{\xi}\sum\limits_{i}^{N}\weight _{i}\left|\Shr_{i}^{\tau }-\mathbf{s}^{\tau}(\xi )\right|\label{eq:StructEstim}
  \end{equation}
\unskip
where $\weight_{i}$ is the weight of household $i$ in the entire
population,\footnote{The Survey of Consumer Finances includes many
  more high-wealth households than exist in the population as a whole;
  therefore if one wants to produce population-representative
  statistics, one must be careful to weight each observation by the
  factor that reflects its ``true'' weight in the population.} and
$\varsigma_{i}^{\tau }$ is the empirical wealth to permanent income
ratio of household $i$ whose head belongs to age group
$\tau$. $\weight _{i}$ is needed because unequal weight is assigned to
each observation in the Survey of Consumer Finances (SCF). The
absolute value is used since the formula is based on the fact that the
median is the value that minimizes the sum of the absolute deviations
from itself.

% In the absence of observation specific weights, equation (\ref{eq:MinStructEstim}) can be simplified to require the minimization of the distance between the empirical and simulated medians.

With this in mind, we turn our attention to the computation
of the weighted median wealth target moments for each age cohort
using this data from the 2004 Survery of Consumer Finances on household
wealth. The objects necessary to accomplish this task are \texttt{weighted\_median} and
\texttt{get\_targeted\_moments}. The actual data are taken from several waves of the SCF and the medians
and means for each age category are plotted in figure \ref{fig:MeanMedianSCF}.
More details on the SCF data are included in appendix \ref{app:SCFdata}.

\hypertarget{PlotMeanMedianSCFcollegeGrads}{}
\begin{figure}
  % \includegraphics[width=6in]{./Figures/PlotMeanMedianSCF}} % weird mean value
  \includegraphics[width=6in]{./Figures/PlotMeanMedianSCFcollegeGrads}
  \caption{Wealth to Permanent Income Ratios from SCF (means (dashed) and medians (solid))}
  \label{fig:MeanMedianSCF}
\end{figure}

We now turn our attention to the the two key functions in this
section of the code file. The first, \texttt{simulate\_moments}, executes the solving (\texttt{solve})
and simulation (\texttt{simulation}) steps for the defined life-cycle agent.
Subsequently, the function uses the agents' tracked levels of wealth based on
their optimal consumption behavior to compute and store the simulated median
wealth to income ratio for each age cohort. The second function, \texttt{smmObjectiveFxn},
calls the \texttt{simulate\_moments} function to create the objective function
described in (\ref{eq:StructEstim}), which is necessary to perform the SMM estimation.


%\begin{verbatimwrite}{./Equations/GapEmpiricalSimulatedMedians.tex}
%  \begin{equation}\begin{gathered}\begin{aligned}
%        \lefteqn{    \texttt{GapEmpiricalSimulatedMedians$[\CRRA,\beth]$:=}}    \nonumber \\
%        &[&\texttt{ConstructcFuncLife$[\CRRA,\beth]$;}\nonumber\\
%        &\texttt{Simulate;}\nonumber\\
%        &\sum\limits_{i}^{N}\weight _{i}\left|\varsigma_{i}^{\tau }-\mathbf{s}^{\tau}(\xi )\right| \nonumber\\
%        &];&\nonumber
%      \end{aligned}\end{gathered}\end{equation}
%\end{verbatimwrite}
%\input{./Equations/GapEmpiricalSimulatedMedians.tex}\unskip

Thus, for a given pair of the parameters to be estimated, the single
call to the function \texttt{smmObjectiveFxn} executes the following:
\begin{enumerate}
\item solves for the consumption functions for the life-cycle agent
\item simulates the data and computes the simulated medians
\item returns the value of equation (\ref{eq:StructEstim})
\end{enumerate}

We delegate the task of finding the coefficients that minimize the
\texttt{smmObjectiveFxn} function to the \texttt{minimize\_nelder\_mead}
function, which is defined elsewhere and called in the second part of this file.
This task can be quite time demanding and rather problematic if the
\texttt{smmObjectiveFxn} function has very flat regions
or sharp features. It is thus wise to verify the accuracy of the
solution, for example by experimenting with a variety of alternative starting values for the
parameter search.

The final object defined in this first part of the \texttt{StructEstimation.py}
file is \texttt{calculateStandardErrorsByBootstrap}. As the name suggsts, the
purpose of this function is to compute the standard errors by bootstrap.\footnote{For a
  treatment of the advantages of the bootstrap see
  Horowitz~\citeyearpar{horowitzBootstrap}} This involves:
\begin{enumerate}
\item drawing new shocks for the simulation
\item drawing a random sample (with replacement) of actual data from the SCF
\item obtaining new estimates for $\CRRA$ and $\beth$
\end{enumerate}
We repeat the above procedure several times (\texttt{Bootstrap}) and
take the standard deviation for each of the estimated parameters across the various bootstrap iterations.

\subsubsection{An Aside to Computing Sensitivity Measures}\label{subsubsec:sensmeas}

A common drawback in commonly used structural estimation procedures is a lack of transparency in its estimates.
As \cite{andrews2017measuring} notes, a researcher employing such structural empirical methods may be interested
in how alternative assumptions (such as misspecification or measurement bias in the data) would ``change the moments
of the data that the estimator uses as inputs, and how changes in these moments affect the estimates.'' The authors
provide a measure of sensitivity for given estimator that makes it easy to map the effects of different assumptions
on the moments into predictable bias in the estimates for non-linear models.

In the language of \cite{andrews2017measuring}, section \ref{sec:StructEst} is aimed at providing an
estimator $\xi =\{\CRRA ,\beth\}$ that has some true value $\xi_0 $ by assumption. Under the assumption $a_0$ of the
researcher, the empirical targets computed from the SCF is measured accurately. These moments of the data are precisely
what determine our estimate $\hat{\xi}$, which minimizes (\ref{eq:StructEstim}). Under alternative assumptions $a$,
such that a given cohort is mismeasured in the survey, a different estimate is computed. Using the plug-in estimate
provided by the authors, we can see quantitatively how our estimate changes under these alternative assumptions $a$ which correspond
to mismeasurement in the median wealth to income ratio for a given age cohort.

\subsection{Results}
The second part of the file \texttt{StructEstimation.py}
defines a function \texttt{main} which produces our $\CRRA$ and
$\beth$ estimates with standard errors using 10,000 simulated
agents by setting the positional arguments \texttt{estimate\_model} and
\texttt{compute\_standard\_errors} to true.\footnote{The procedure is: First we calculate the $\CRRA$ and
  $\beth$ estimates as the minimizer of equation
  (\ref{eq:StructEstim}) using the actual SCF data. Then, we apply the
  \texttt{Bootstrap} function several times to obtain the standard
  error of our estimates.} Results are reported in Table
\ref{tab:EstResults}.\footnote{Differently from Cagetti
  ~\citeyearpar{cagettiWprofiles} who estimates a different set of
  parameters for college graduates, high school graduates and high
  school dropouts graduates, we perform the structural estimation on
  the full population.}


\begin{verbatimwrite}{./Tables/EstResults.tex}
  \begin{table}[h]
    \caption{Estimation Results}\label{tab:EstResults}
    \center
    \begin{tabular}{cc}
      \hline
      $\CRRA $ & $\beth$\\
      \hline
      $3.69$ & $0.88$\\
      $(0.047)$ & $(0.002)$\\
      \hline
    \end{tabular}
  \end{table}
\end{verbatimwrite}
  \begin{table}[h]
    \caption{Estimation Results}\label{tab:EstResults}
    \center
    \begin{tabular}{cc}
      \hline
      $\CRRA $ & ${\beth}$\\
      \hline
      $3.69$ & $0.88$\\
      $(0.047)$ & $(0.002)$\\
      \hline
    \end{tabular}
  \end{table}
\unskip

The literature on consumption and savings behavior over the lifecycle in
the presenece of labor income uncertainty \footnote{For example, see \cite{gpLifecycle} for an
exposition of this.} warns us to be careful in disentangling the effect of time preference and
risk aversion when describing the optimal behavior of households in this setting.
Since the precautionary saving motive dominates in the early stages of life, the coefficient of relative
risk aversion (as well as expected labor income growth) has a larger effect on optimal consumption and
saving behavior through their magnitude relative to the
interest rate. Over time, life-cycle considerations (such as saving for retirement) become more important and
the time preference factor plays a larger role in determining optimal behavior for this cohort.

Using the positional argument \texttt{compute\_sensitivity}, Figure \ref{fig:PlotSensitivityMeasure}
provides a plot of the plug-in estimate of the sensitivity measure described in \ref{subsubsec:sensmeas}. As you can see from
the figure the inverse relationship between $\rho$ and $\beth$ over the life-cycle
is retained by the sensitivity measure. Specifically, under the alternative
assumption that \textit{a particular cohort is mismeasured in the SCF dataset}, we see that the
y-axis suggests that our estimate of $\rho$ and $\beth$ change in a predictable way.

Suppose that there are not enough observations of the oldest cohort of households in the
sample. Suppose further that the researcher predicts that adding more observations of these households
to correct this mismeasurement would correspond to a higher median wealth to income ratio
for this cohort. In this case, our estimate of the time preference factor should increase:
the behavior of these older households is driven by their time preference,
so a higher value of $\beth$ is required to match the affected wealth to income targets under this
alternative assumption. Since risk aversion is less important in explaining the behavior of this cohort,
a lower value of $\rho$ is required to match the affected empirical moments.

To recap, the sensitivity measure not only matches our intuition about the inverse relationship between
$\rho$ and $\beth$ over the life-cycle, but provides a quantitative estimate of what would happen to
our estimates of these parameters under the alternative assumption that the data is mismeasured in some way.

\hypertarget{PlotSensitivityMeasure}{}
\begin{figure}
  \includegraphics[width=6in]{./Figures/Sensitivity.pdf}
  \caption{Sensitivty of Estimates $\{\CRRA,\beth\}$ regarding Alternative Mismeasurement Assumptions.}
  \label{fig:PlotSensitivityMeasure}
\end{figure}

By setting the positional argument
\texttt{make\_contour\_plot} to true, Figure \ref{fig:PlotContourMedianStrEst} shows
the contour plot of the \texttt{smmObjectiveFxn} function
and the parameter estimates. The contour plot shows equally spaced
isoquants of the \texttt{smmObjectiveFxn} function,
i.e.\ the pairs of $\CRRA$ and $\beth$ which lead to the same
deviations between simulated and empirical medians (equivalent values
of equation (\ref{eq:StructEstim})). Interestingly, there is a large rather flat region; or, more formally speaking,
there exists a broad set of parameter pairs which leads to similar
simulated wealth to income ratios. Intuitively, the flatter and larger
is this region, the harder it is for the structural estimation
procedure to precisely identify the parameters.

\section{Conclusion}

There are many choices that can be made for solving microeconomic dynamic stochastic optimization problems.
The set of techniques, and associated programs, described in these notes represents an approach that I have found to be powerful, flexible, and efficient, but other problems may require other techniques.  For a much broader treatment of many of the issues considered here, see Judd~\citeyearpar{judd:book}.

\hypertarget{PlotContourMedianStrEst}{}
\begin{figure}
  \includegraphics[width=6in]{./Figures/SMMcontour.pdf}
  \caption{Contour Plot (larger values are shown lighter) with $\{\CRRA,\beth\}$ Estimates (red dot).}
  \label{fig:PlotContourMedianStrEst}
\end{figure}

\clearpage\vfill\eject

\appendix

\centerline{\LARGE Appendices}\vspace{0.2in}

\section{Further Details on SCF Data}\label{app:SCFdata}

Data used in the estimation is constructed using the SCF 1992, 1995, 1998, 2001 and 2004 waves. The definition of wealth is net worth including housing wealth, but excluding pensions and social securities. The data set contains only households whose heads are aged 26-60 and excludes singles, following Cagetti~\citeyearpar{cagettiWprofiles}.\footnote{Cagetti~\citeyearpar{cagettiWprofiles}\ argues that younger households should be dropped since educational choice is not modeled. Also, he drops singles, since they include a large number of single mothers whose saving behavior is influenced by welfare.} Furthermore, the data set contains only households whose heads are college graduates. The total sample size is 4,774.

In the waves between 1995 and 2004 of the SCF, levels of
\textit{normal} income are reported. The question in the questionnaire
is "About what would your income have been if it had been a normal
year?" We consider the level of normal income as corresponding to the
model's theoretical object $P$, permanent noncapital income. Levels of
normal income are not reported in the 1992 wave. Instead, in this wave
there is a variable which reports whether the level of income is
normal or not. Regarding the 1992 wave, only observations which report
that the level of income is normal are used, and the levels of income
of remaining observations in the 1992 wave are interpreted as the
levels of permanent income.

Normal income levels in the SCF are before-tax figures. These
before-tax permanent income figures must be rescaled so that the median of
the rescaled permanent income of each age group matches the median of
each age group's income which is assumed in the simulation. This
rescaled permanent income is interpreted as after-tax permanent
income. This rescaling is crucial since in the estimation empirical
profiles are matched with simulated ones which are generated using
after-tax permanent income (remember the income process assumed in the
main text). Wealth / permanent income ratio is computed by dividing
the level of wealth by the level of (after-tax) permanent income, and
this ratio is used for the estimation.\footnote{Please refer to the archive code for details of
  how these after-tax measures of $P$ are constructed.}

\vfill\clearpage

\write18{if [ ! -f \texname.bib ]; then touch \texname.bib  ; fi}\write18{if [ ! -f \texname-Add.bib ]; then touch \texname-Add.bib  ; fi}\bibliography{economics,\texname,\texname-Add}

\trp{
\pagebreak
\hypertarget{Appendices}{} % Allows link to [url-of-paper]#Appendices
\ifthenelse{\boolean{Web}}{}{% Web version has no page headers
  \chead[Appendices]{Appendices}      % but PDF version does
  \appendixpage % Reset formatting for appendices
}
\appendix
\addcontentsline{toc}{section}{Appendices} % Say "Appendices"

\subfile{TRP_aInU}
}{}


\end{document}\endinput

% Local Variables:
% TeX-master-file: t
% eval: (setq TeX-command-list  (assq-delete-all (car (assoc "BibTeX" TeX-command-list)) TeX-command-list))
% eval: (setq TeX-command-list  (assq-delete-all (car (assoc "Biber"  TeX-command-list)) TeX-command-list))
% eval: (setq TeX-command-list  (remove '("BibTeX" "%(bibtex) %s"    TeX-run-BibTeX nil t :help "Run BibTeX") TeX-command-list))
% eval: (setq TeX-command-list  (remove '("BibTeX"    "bibtex %s"    TeX-run-BibTeX nil (plain-tex-mode latex-mode doctex-mode ams-tex-mode texinfo-mode context-mode)  :help "Run BibTeX") TeX-command-list))
% eval: (setq TeX-command-list  (remove '("BibTeX" "bibtex %s"    TeX-run-BibTeX nil t :help "Run BibTeX") TeX-command-list))
% eval: (add-to-list 'TeX-command-list '("BibTeX" "bibtex %s" TeX-run-BibTeX nil t                                                                              :help "Run BibTeX") t)
% eval: (add-to-list 'TeX-command-list '("BibTeX" "bibtex %s" TeX-run-BibTeX nil (plain-tex-mode latex-mode doctex-mode ams-tex-mode texinfo-mode context-mode) :help "Run BibTeX") t)
% TeX-PDF-mode: t
% TeX-file-line-error: t
% TeX-debug-warnings: t
% LaTeX-command-style: (("" "%(PDF)%(latex) %(file-line-error) %(extraopts) -output-directory=. %S%(PDFout)"))
% TeX-source-correlate-mode: t
% TeX-parse-self: t
% TeX-parse-all-errors: t
% eval: (cond ((string-equal system-type "darwin") (progn (setq TeX-view-program-list '(("Skim" "/Applications/Skim.app/Contents/SharedSupport/displayline -b %n %o %b"))))))
% eval: (cond ((string-equal system-type "gnu/linux") (progn (setq TeX-view-program-list '(("Evince" "evince --page-index=%(outpage) %o"))))))
% eval: (cond ((string-equal system-type "gnu/linux") (progn (setq TeX-view-program-selection '((output-pdf "Evince"))))))
% End:
