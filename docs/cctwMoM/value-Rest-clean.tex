  \MPCMatch{with derivative
    \begin{equation*}\begin{gathered}\begin{aligned}
          \bar{\vInv}_{\prd}^m  & = (\mathbb{C}_{\prd}^{T})^{1/(1-\CRRA)}\MPCmin_{\prd},
          \\  & = \MPCmin_{\prd}^{-\CRRA/(1-\CRRA)} % 20190820
        \end{aligned}\end{gathered}\end{equation*}}{}
  and since $\PDVCoverc_{\prd}^{T}$ is a constant while the consumption
  function is linear, $\bar{\vInv}_{\prd}$ will also be linear.

  We apply the same transformation to the value function for the problem with uncertainty (the ``realist's'' problem)\MPCMatch{ and differentiate}:
  \begin{equation*}\begin{gathered}\begin{aligned}
        \bar{\vInv}_{\prd}  & = \left((1-\CRRA)\bar{\vFunc}_{\prd}(m_{\prd})\right)^{1/(1-\CRRA)}
        \MPCMatch{\\ \bar{\vInv}^{m}_{\prd}  & = \left((1-\CRRA)\bar{\vFunc}_{\prd}(m_{\prd})\right)^{-1+1/(1-\CRRA)}\bar{\vFunc}_{\prd}^{m}(m_{\prd})}{}
      \end{aligned}\end{gathered}\end{equation*}
  and an excellent approximation to the value function can be obtained by
  calculating the values of $\bar{\vInv}$ at the same gridpoints used by the
  consumption function approximation, and interpolating among those points.

  However, as with the consumption approximation, we can do even better if we
  realize that the $\bar{\vInv}$ function for the optimist's problem is
  an upper bound for the ${\vInv}$ function in the presence of uncertainty, and the value function
  for the pessimist is a lower bound. Analogously to \eqref{eq:koppa}, define an upper-case
  \begin{equation}\begin{gathered}\begin{aligned}
        \hat{\Koppa}_{\prd}(\mu_{\prd})   & = \left(\frac{\bar{\vInv}_{\prd}(\ushort{m}_{\prd}+e^{\mu_{\prd}})-\vInv_{\prd}(\ushort{m}_{\prd}+e^{\mu_{\prd}})}{\aboveMin \hNrm_{\EndStg} \MPCmin_{\prd} (\PDVCoverc_{\prd}^{T})^{1/(1-\CRRA)}}\right) \label{eq:Koppa}
      \end{aligned}\end{gathered}\end{equation}
  \MPCMatch{with derivative (dropping arguments)
    \begin{equation}\begin{gathered}\begin{aligned}
          \hat{\Koppa}_{\prd}^{\mu}   & = (\aboveMin \hNrm_{\EndStg} \MPCmin_{\prd} (\PDVCoverc_{\prd}^{T})^{1/(1-\CRRA)})^{-1}e^{\mu_{\prd}}\left(\bar{\vInv}^{m}_{\prd}-\vInv^{m}_{\prd}\right) \label{eq:KoppaPrime}
          % \\  & =  (\aboveMin \hNrm_{\EndStg} \MPCmin_{\prd})^{-1}e^{\mu_{\prd}}\left((\PDVCoverc_{\prd}^{T})^{1/(1-\CRRA)}\MPCmin_{\prd}-\left((1-\CRRA)\vFunc_{\prd}(m_{\prd})\right)^{-1+1/(1-\CRRA)}\vFunc_{\prd}^{m}(m_{\prd})\right)  \notag
        \end{aligned}\end{gathered}\end{equation}}{}
  and an upper-case version of the $\chiFunc$ equation in \eqref{eq:chi}:
  \begin{equation}\begin{gathered}\begin{aligned}
        \hat{\Chi}_{\prd}(\mu_{\prd})  & = \log \left(\frac{1-\hat{\Koppa}_{\prd}(\mu_{\prd})}{\hat{\Koppa}_{\prd}(\mu_{\prd})}\right)
        \\  & = \log \left(1/\hat{\Koppa}_{\prd}(\mu_{\prd})-1\right) \label{eq:Chi}
      \end{aligned}\end{gathered}\end{equation}
  \MPCMatch{with corresponding derivative
    \begin{equation}\begin{gathered}\begin{aligned}
          \hat{\Chi}_{\prd}^{\mu}  & = \left(\frac{-\hat{\Koppa}_{\prd}^{\mu}/\hat{\Koppa}_{\prd}^{2}}{1/\hat{\Koppa}_{\prd}-1}\right)
        \end{aligned}\end{gathered}\end{equation}}{}
  and if we approximate these objects then invert them (as above with
  the $\Hi{\koppa}$ and $\Hi{\chiFunc}$ functions) we obtain a very high-quality
  approximation to our inverted value function at the same points for
  which we have our approximated value function:
  \begin{equation}\begin{gathered}\begin{aligned}
        \hat{\vInv}_{\prd}  & = \bar{\vInv}_{\prd}-\overbrace{\left(\frac{1}{1+\exp(\hat{\Chi}_{\prd})}\right)}^{=\hat{\Koppa}_{\prd}} \aboveMin \hNrm_{\EndStg} \MPCmin_{\prd} (\PDVCoverc_{\prd}^{T})^{1/(1-\CRRA) }
      \end{aligned}\end{gathered}\end{equation}
  from which we obtain our approximation to the value function\MPCMatch{ and its derivatives~}~as \hypertarget{vHatFunc}{}
  \begin{equation}\begin{gathered}\begin{aligned}
        \hat{\vFunc}_{\prd}  & = \uFunc(\hat{\vInv}_{\prd})
        \\  \hat{\vFunc}^{m}_{\prd}  & = \uFunc^{c}(\hat{\vInv}_{\prd}) \hat{\vInv}^{m}
        \MPCMatch{\\  \hat{\vFunc}^{mm}_{\prd}  & = \uFunc^{c{c}}(\hat{\vInv}_{\prd}) (\hat{\vInv}^{m})^{2} + \uFunc^{c}(\hat{\vInv}_{\prd})\hat{\vInv}^{mm}}{}
        .
      \end{aligned}\end{gathered}\end{equation}

  Although a linear interpolation that matches the level of $\vInv$ at the gridpoints is simple, a Hermite interpolation that matches both the level and the derivative of the $\bar{\vInv}_{\prd}$ function at the gridpoints has the considerable virtue that the $\bar{\vFunc}_{\prd}$ derived from it numerically satisfies the envelope theorem at each of the gridpoints for which the problem has been solved.

  \MPCMatch{If we use the double-derivative calculated above to produce a higher-order Hermite polynomial, our approximation will also match
    marginal propensity to consume at the gridpoints; this would
    guarantee that the consumption function generated from the value
    function would match both the level of consumption and the
    marginal propensity to consume at the gridpoints; the numerical
    differences between the newly constructed consumption function and
    the highly accurate one constructed earlier would be negligible
    within the grid.}{}

