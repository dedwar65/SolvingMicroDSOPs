% Add the listed directories to the search path
% (allows easy moving of files around later)
% these paths are searched AFTER local config kpsewhich
\makeatletter
\def\input@path{{./.resources/latex//}{./.resources/texlive//}{./.resources/texmf-local/tex/latex//}{./resources/texmf-local/tex/bibtex//}{./.resources/econ-ark//}{./Code/Python/snippets/}}
\makeatother 
\documentclass[SolvingMicroDSOPs]{subfiles}
% econtexRoot gets obliterated by the documentclass command 
%% -*- mode: LaTeX; TeX-PDF-mode: t; -*- # Tell emacs the file type (for syntax)
% LaTeX path to the root directory of the current project, from the directory in which this file resides
% and path to econtexPaths which defines the rest of the paths like \FigDir
\providecommand{\econtexRoot}{}\renewcommand{\econtexRoot}{.}
\providecommand{\econtexPaths}{}\renewcommand{\econtexPaths}{\econtexRoot/Resources/econtexPaths}
% -*- mode: LaTeX; TeX-PDF-mode: t; -*- 
% The \commands below are required to allow sharing of the same base code via Github between TeXLive on a local machine and Overleaf (which is a proxy for "a standard distribution of LaTeX").  This is an ugly solution to the requirement that custom LaTeX packages be accessible, and that Overleaf prohibits symbolic links
\providecommand{\packages}{\econtexRoot/Resources/texmf-local/tex/latex}
\providecommand{\econtex}{\packages/econtex}
\providecommand{\econark}{\econtexRoot/Resources/texmf-local/tex/latex/econark}
\providecommand{\econtexSetup}{\econtexRoot/Resources/texmf-local/tex/latex/econtexSetup}
\providecommand{\econtexShortcuts}{\econtexRoot/Resources/texmf-local/tex/latex/econtexShortcuts}
\providecommand{\econtexBibMake}{\econtexRoot/Resources/texmf-local/tex/latex/econtexBibMake}
\providecommand{\econtexBibStyle}{\econtexRoot/Resources/texmf-local/bibtex/bst/econtex}
\providecommand{\econtexBib}{economics}
\providecommand{\notes}{\econtexRoot/Resources/texmf-local/tex/latex/handout}
\providecommand{\handoutSetup}{\econtexRoot/Resources/texmf-local/tex/latex/handoutSetup}
\providecommand{\handoutShortcuts}{\econtexRoot/Resources/texmf-local/tex/latex/handoutShortcuts}
\providecommand{\handoutBibMake}{\econtexRoot/Resources/texmf-local/tex/latex/handoutBibMake}
\providecommand{\handoutBibStyle}{\econtexRoot/Resources/texmf-local/bibtex/bst/handout}

\providecommand{\FigDir}{\econtexRoot/Figures}
\providecommand{\CodeDir}{\econtexRoot/Code}
\providecommand{\DataDir}{\econtexRoot/Data}
\providecommand{\SlideDir}{\econtexRoot/Slides}
\providecommand{\TableDir}{\econtexRoot/Tables}
\providecommand{\ApndxDir}{\econtexRoot/Appendices}

\providecommand{\ResourcesDir}{\econtexRoot/Resources}
\providecommand{\rootFromOut}{..} % APFach back to root directory from output-directory
\providecommand{\LaTeXGenerated}{\econtexRoot/LaTeX} % Put generated files in subdirectory
\providecommand{\econtexPaths}{\econtexRoot/Resources/econtexPaths}
\providecommand{\LaTeXInputs}{\econtexRoot/Resources/LaTeXInputs}
\providecommand{\LtxDir}{LaTeX/}
\providecommand{\EqDir}{\econtexRoot/Equations} % Put generated files in subdirectory

\providecommand{\local}{Resources/LaTeXInputs/local}

\documentclass[\econtexRoot/SolvingMicroDSOPs]{subfiles}
% -*- mode: LaTeX; TeX-PDF-mode: t; -*- # Tell emacs the file type (for syntax)
% LaTeX path to the root directory of the current project, from the directory in which this file resides
% and path to econtexPaths which defines the rest of the paths like \FigDir
\providecommand{\econtexRoot}{}\renewcommand{\econtexRoot}{.}
\providecommand{\econtexPaths}{}\renewcommand{\econtexPaths}{\econtexRoot/Resources/econtexPaths}
% -*- mode: LaTeX; TeX-PDF-mode: t; -*- 
% The \commands below are required to allow sharing of the same base code via Github between TeXLive on a local machine and Overleaf (which is a proxy for "a standard distribution of LaTeX").  This is an ugly solution to the requirement that custom LaTeX packages be accessible, and that Overleaf prohibits symbolic links
\providecommand{\packages}{\econtexRoot/Resources/texmf-local/tex/latex}
\providecommand{\econtex}{\packages/econtex}
\providecommand{\econark}{\econtexRoot/Resources/texmf-local/tex/latex/econark}
\providecommand{\econtexSetup}{\econtexRoot/Resources/texmf-local/tex/latex/econtexSetup}
\providecommand{\econtexShortcuts}{\econtexRoot/Resources/texmf-local/tex/latex/econtexShortcuts}
\providecommand{\econtexBibMake}{\econtexRoot/Resources/texmf-local/tex/latex/econtexBibMake}
\providecommand{\econtexBibStyle}{\econtexRoot/Resources/texmf-local/bibtex/bst/econtex}
\providecommand{\econtexBib}{economics}
\providecommand{\notes}{\econtexRoot/Resources/texmf-local/tex/latex/handout}
\providecommand{\handoutSetup}{\econtexRoot/Resources/texmf-local/tex/latex/handoutSetup}
\providecommand{\handoutShortcuts}{\econtexRoot/Resources/texmf-local/tex/latex/handoutShortcuts}
\providecommand{\handoutBibMake}{\econtexRoot/Resources/texmf-local/tex/latex/handoutBibMake}
\providecommand{\handoutBibStyle}{\econtexRoot/Resources/texmf-local/bibtex/bst/handout}

\providecommand{\FigDir}{\econtexRoot/Figures}
\providecommand{\CodeDir}{\econtexRoot/Code}
\providecommand{\DataDir}{\econtexRoot/Data}
\providecommand{\SlideDir}{\econtexRoot/Slides}
\providecommand{\TableDir}{\econtexRoot/Tables}
\providecommand{\ApndxDir}{\econtexRoot/Appendices}

\providecommand{\ResourcesDir}{\econtexRoot/Resources}
\providecommand{\rootFromOut}{..} % APFach back to root directory from output-directory
\providecommand{\LaTeXGenerated}{\econtexRoot/LaTeX} % Put generated files in subdirectory
\providecommand{\econtexPaths}{\econtexRoot/Resources/econtexPaths}
\providecommand{\LaTeXInputs}{\econtexRoot/Resources/LaTeXInputs}
\providecommand{\LtxDir}{LaTeX/}
\providecommand{\EqDir}{\econtexRoot/Equations} % Put generated files in subdirectory

\providecommand{\local}{Resources/LaTeXInputs/local}

\onlyinsubfile{% https://tex.stackexchange.com/questions/463699/proper-reference-numbers-with-subfiles
    \csname @ifpackageloaded\endcsname{xr-hyper}{%
      \externaldocument{\econtexRoot/SolvingMicroDSOPs}% xr-hyper in use; optional argument for url of main.pdf for hyperlinks
    }{%
      \externaldocument{\econtexRoot/SolvingMicroDSOPs}% xr in use
    }%
    \renewcommand\labelprefix{}%
    % Initialize the counters via the labels belonging to the main document:
}



% Get xrefs; only works properly if main file has already been successfully compiled
\onlyinsubfile{\externaldocument{SolvingMicroDSOPs}} 



\begin{document}
\hypertarget{normalization}{}
\section{Normalization}\label{sec:normalization}

The single most powerful method for speeding the solution of such models is to redefine the problem in a way that reduces the number of state variables (if at all possible).  In the consumption context, the obvious idea is to see whether the problem can be rewritten in terms of the ratio of various variables to permanent noncapital (`labor') income $\pLvl_{\prd}$ (henceforth for brevity, `permanent income.')

In the last {\interval} of life $\trmT$, there is no future value, $\vLvl_{\trmT+1} = 0$, so the optimal plan is to consume everything:
\begin{equation}\begin{gathered}\begin{aligned}
      \vFuncLvl_{\trmT}(\mLvl_{\trmT},\pLvl_{\trmT})  & = \frac{\mLvl_{\trmT}^{1-\CRRA}}{1-\CRRA}. \label{eq:levelTm1}
    \end{aligned}\end{gathered}\end{equation}
Now define nonbold variables as the bold variable divided by the level of permanent income in the same period, so that, for example, $\mNrm_{\trmT}=\mLvl_{\trmT}/\pLvl_{\trmT}$; and define $\vFunc_{\trmT}(\mNrm_{\trmT}) = \uFunc(\mNrm_{\trmT})$.\footnote{Nonbold value is bold value divided by $\pLvl^{1-\CRRA}$ rather than $\pLvl$.}  For our CRRA utility function, $\uFunc(xy)=x^{1-\CRRA}\uFunc(y)$, so (\ref{eq:levelTm1}) can be rewritten as
\begin{equation}\begin{gathered}\begin{aligned}
      \vFuncLvl_{\trmT}(\mLvl_{\trmT},\pLvl_{\trmT}) & = \pLvl_{\trmT}^{1-\CRRA}\frac{\mNrm_{\trmT}^{1-\CRRA}}{1-\CRRA}                       \\
%                                                & = (\pLvl_{\trmT-1}\PermGroFac_{\trmT})^{1-\CRRA}\frac{{\mNrm}_{\trmT}^{1-\CRRA}}{1-\CRRA} \\
                                                &= \pLvl_{\trmT-1}^{1-\CRRA}\PermGroFac_{\trmT}^{1-\CRRA}\vFunc_{\trmT}(\mNrm_{\trmT}). \label{eq:vT}
    \end{aligned}\end{gathered}\end{equation}

Now define a new optimization problem:
\begin{verbatimwrite}{./Equations/vNormed.tex}
  \begin{equation}\begin{gathered}\begin{aligned}
        \vFunc_{\prd}(\mNrm_{\prd}) & = \max_{{\cNrm}_{\prd}} ~~ \uFunc(\cNrm_{\prd})+\DiscFac \Ex_{\prd}[ \PermGroFac_{\prd+1}^{1-\CRRA}\vFunc_{\prd+1}(\mNrm_{\prd+1})] \label{eq:vNormed}                   \\
                                         & \text{s.t.}                                                                                 \\
        \aNrm_{\prd}                       & = \mNrm_{\prd}-\cNrm_{\prd}                                                                     \\
        \kNrm_{\prd+1}                     & = \aNrm_{\prd}                                                                                \\
        \bNrm_{\prd+1}                     & = \underbrace{\left(\Rfree/\PermGroFac_{\prd+1}\right)}_{\equiv \RNrmByG_{\prd+1}}\kNrm_{\prd+1} \\
        \mNrm_{t+1}                        & = \bNrm_{t+1}+\tranShkEmp_{t+1},
      \end{aligned}\end{gathered}\end{equation}
\end{verbatimwrite}
  \begin{equation}\begin{gathered}\begin{aligned}
        {\vFunc}_{\prd}({m}_{\prd}) & = \max_{{c}_{\prd}} ~~ \uFunc({c}_{\prd})+
        {\DiscFac}\Ex_{\prd}[ \PermGroFac_{\prd+1}^{1-\CRRA}{\vFunc}_{\prd+1}({m}_{\prd+1})] \label{eq:vNormed}                   \\
                                         & \text{s.t.}                                                                                 \\
        {a}_{\prd}                       & = {m}_{\prd}-{c}_{\prd}                                                                     \\
        {k}_{\prd+1}                     & = {a}_{\prd}                                                                                \\
        {b}_{\prd+1}                     & = \underbrace{\left(\Rfree/\PermGroFac_{\prd+1}\right)}_{\equiv \RNrm_{\prd+1}}{k}_{\prd+1} \\
        {m}_{t+1}                        & = {b}_{t+1}+\TranShkEmp_{t+1},
      \end{aligned}\end{gathered}\end{equation}
\unskip
where division by $\PermGroFac$ in second-to-last equation yields a normalized return factor $\RNrmByG$ which is the consequence of the fact that we have divided $\prd+1$ level variables by $\pLvl_{\prd+1}=\PermGroFac_{\prd+1}\pLvl_{\prd}$.

\ifpseudo{ % pseudocode goes here
\lstinputlisting{equiprobable-make.py}\nopagebreak
  }{}
  
Then it is easy to see that for $\prd=\trmT-1$, we can write boldface (nonnormalized) $\vFuncLvl$ as a function of $\vFunc$ (normalized value) and permanent income:
\begin{equation}\begin{gathered}\begin{aligned}
      \vFuncLvl_{\prd}(\mLvl_{\prd},\pLvl_{\prd}) & =  \pLvl_{\prd}^{1-\CRRA}\vFunc_{\prdt}(\mNrm_{\prdt}), \label{eq:vLvlFromvFunc}
    \end{aligned}\end{gathered}\end{equation}
and so on back to all earlier periods.  Hence, if we solve the problem \eqref{eq:vNormed} which has only a single state variable $\mNrm_{\prd}$, we can obtain the levels of the value function from \eqref{eq:vLvlFromvFunc}, and of consumption and all other variables from the corresponding permanent-income-normalized solution objects by multiplying each by $\pLvl_{\prd}$, e.g.\
\begin{equation*}\begin{gathered}\begin{aligned}
  \cFunc_{\prd}(\mLvl_{\prd},\pLvl_{\prd})=\pLvl_{\prd}\cFunc_{\prd}(\overbrace{\mLvl_{\prd}/\pLvl_{\prd}}^{\mNrm_{\prd}}).
    \end{aligned}\end{gathered}\end{equation*}
%(or, for the value function, $\vFuncLvl _{\prd}(\mLvl_{\prd},\pLvl_{\prd}) = \pLvl_{\prd}^{1-\CRRA}\vFunc_{\prd}(\mNrm_{\prd}))$.

We have thus reduced the problem from two continuous state variables to one (and thereby enormously simplified its solution).

For future reference it is useful to write \eqref{eq:vNormed} in the traditional way, by substituting $\bNrm_{\prdt+1},\kNrm_{\prdt+1},$ and $\aNrm_{\prdt}$ into $\mNrm_{t+1}$:
\begin{equation}\begin{gathered}\begin{aligned}
      \vFunc_{\prdt}(\mNrm_{\prdt}) & = \max_{\cNrm} ~~ \uFunc(\cNrm)+ \DiscFac \Ex_{\prdt}[ \PermGroFac_{\prdt+1}^{1-\CRRA}\vFunc_{\prdt+1}(\overbrace{(\mNrm_{t}-\cNrm)(\Rfree/\PermGroFac_{t+1})+\tranShkEmp_{t+1}}^{m_{t+1}})] \label{eq:vusual}.
    \end{aligned}\end{gathered}\end{equation}

\end{document}
