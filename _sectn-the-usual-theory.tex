% Add the listed directories to the search path
% (allows easy moving of files around later)
% these paths are searched AFTER local config kpsewhich
\makeatletter
\def\input@path{{./.resources/latex//}{./.resources/texlive//}{./.resources/texmf-local/tex/latex//}{./resources/texmf-local/tex/bibtex//}{./.resources/econ-ark//}{./Code/Python/snippets/}}
\makeatother 
\documentclass[SolvingMicroDSOPs]{subfiles}
% econtexRoot gets obliterated by the documentclass command 
%% -*- mode: LaTeX; TeX-PDF-mode: t; -*- # Tell emacs the file type (for syntax)
% LaTeX path to the root directory of the current project, from the directory in which this file resides
% and path to econtexPaths which defines the rest of the paths like \FigDir
\providecommand{\econtexRoot}{}\renewcommand{\econtexRoot}{.}
\providecommand{\econtexPaths}{}\renewcommand{\econtexPaths}{\econtexRoot/Resources/econtexPaths}
% -*- mode: LaTeX; TeX-PDF-mode: t; -*- 
% The \commands below are required to allow sharing of the same base code via Github between TeXLive on a local machine and Overleaf (which is a proxy for "a standard distribution of LaTeX").  This is an ugly solution to the requirement that custom LaTeX packages be accessible, and that Overleaf prohibits symbolic links
\providecommand{\packages}{\econtexRoot/Resources/texmf-local/tex/latex}
\providecommand{\econtex}{\packages/econtex}
\providecommand{\econark}{\econtexRoot/Resources/texmf-local/tex/latex/econark}
\providecommand{\econtexSetup}{\econtexRoot/Resources/texmf-local/tex/latex/econtexSetup}
\providecommand{\econtexShortcuts}{\econtexRoot/Resources/texmf-local/tex/latex/econtexShortcuts}
\providecommand{\econtexBibMake}{\econtexRoot/Resources/texmf-local/tex/latex/econtexBibMake}
\providecommand{\econtexBibStyle}{\econtexRoot/Resources/texmf-local/bibtex/bst/econtex}
\providecommand{\econtexBib}{economics}
\providecommand{\notes}{\econtexRoot/Resources/texmf-local/tex/latex/handout}
\providecommand{\handoutSetup}{\econtexRoot/Resources/texmf-local/tex/latex/handoutSetup}
\providecommand{\handoutShortcuts}{\econtexRoot/Resources/texmf-local/tex/latex/handoutShortcuts}
\providecommand{\handoutBibMake}{\econtexRoot/Resources/texmf-local/tex/latex/handoutBibMake}
\providecommand{\handoutBibStyle}{\econtexRoot/Resources/texmf-local/bibtex/bst/handout}

\providecommand{\FigDir}{\econtexRoot/Figures}
\providecommand{\CodeDir}{\econtexRoot/Code}
\providecommand{\DataDir}{\econtexRoot/Data}
\providecommand{\SlideDir}{\econtexRoot/Slides}
\providecommand{\TableDir}{\econtexRoot/Tables}
\providecommand{\ApndxDir}{\econtexRoot/Appendices}

\providecommand{\ResourcesDir}{\econtexRoot/Resources}
\providecommand{\rootFromOut}{..} % APFach back to root directory from output-directory
\providecommand{\LaTeXGenerated}{\econtexRoot/LaTeX} % Put generated files in subdirectory
\providecommand{\econtexPaths}{\econtexRoot/Resources/econtexPaths}
\providecommand{\LaTeXInputs}{\econtexRoot/Resources/LaTeXInputs}
\providecommand{\LtxDir}{LaTeX/}
\providecommand{\EqDir}{\econtexRoot/Equations} % Put generated files in subdirectory

\providecommand{\local}{Resources/LaTeXInputs/local}

\documentclass[\econtexRoot/SolvingMicroDSOPs]{subfiles}
% -*- mode: LaTeX; TeX-PDF-mode: t; -*- # Tell emacs the file type (for syntax)
% LaTeX path to the root directory of the current project, from the directory in which this file resides
% and path to econtexPaths which defines the rest of the paths like \FigDir
\providecommand{\econtexRoot}{}\renewcommand{\econtexRoot}{.}
\providecommand{\econtexPaths}{}\renewcommand{\econtexPaths}{\econtexRoot/Resources/econtexPaths}
% -*- mode: LaTeX; TeX-PDF-mode: t; -*- 
% The \commands below are required to allow sharing of the same base code via Github between TeXLive on a local machine and Overleaf (which is a proxy for "a standard distribution of LaTeX").  This is an ugly solution to the requirement that custom LaTeX packages be accessible, and that Overleaf prohibits symbolic links
\providecommand{\packages}{\econtexRoot/Resources/texmf-local/tex/latex}
\providecommand{\econtex}{\packages/econtex}
\providecommand{\econark}{\econtexRoot/Resources/texmf-local/tex/latex/econark}
\providecommand{\econtexSetup}{\econtexRoot/Resources/texmf-local/tex/latex/econtexSetup}
\providecommand{\econtexShortcuts}{\econtexRoot/Resources/texmf-local/tex/latex/econtexShortcuts}
\providecommand{\econtexBibMake}{\econtexRoot/Resources/texmf-local/tex/latex/econtexBibMake}
\providecommand{\econtexBibStyle}{\econtexRoot/Resources/texmf-local/bibtex/bst/econtex}
\providecommand{\econtexBib}{economics}
\providecommand{\notes}{\econtexRoot/Resources/texmf-local/tex/latex/handout}
\providecommand{\handoutSetup}{\econtexRoot/Resources/texmf-local/tex/latex/handoutSetup}
\providecommand{\handoutShortcuts}{\econtexRoot/Resources/texmf-local/tex/latex/handoutShortcuts}
\providecommand{\handoutBibMake}{\econtexRoot/Resources/texmf-local/tex/latex/handoutBibMake}
\providecommand{\handoutBibStyle}{\econtexRoot/Resources/texmf-local/bibtex/bst/handout}

\providecommand{\FigDir}{\econtexRoot/Figures}
\providecommand{\CodeDir}{\econtexRoot/Code}
\providecommand{\DataDir}{\econtexRoot/Data}
\providecommand{\SlideDir}{\econtexRoot/Slides}
\providecommand{\TableDir}{\econtexRoot/Tables}
\providecommand{\ApndxDir}{\econtexRoot/Appendices}

\providecommand{\ResourcesDir}{\econtexRoot/Resources}
\providecommand{\rootFromOut}{..} % APFach back to root directory from output-directory
\providecommand{\LaTeXGenerated}{\econtexRoot/LaTeX} % Put generated files in subdirectory
\providecommand{\econtexPaths}{\econtexRoot/Resources/econtexPaths}
\providecommand{\LaTeXInputs}{\econtexRoot/Resources/LaTeXInputs}
\providecommand{\LtxDir}{LaTeX/}
\providecommand{\EqDir}{\econtexRoot/Equations} % Put generated files in subdirectory

\providecommand{\local}{Resources/LaTeXInputs/local}

\onlyinsubfile{% https://tex.stackexchange.com/questions/463699/proper-reference-numbers-with-subfiles
    \csname @ifpackageloaded\endcsname{xr-hyper}{%
      \externaldocument{\econtexRoot/SolvingMicroDSOPs}% xr-hyper in use; optional argument for url of main.pdf for hyperlinks
    }{%
      \externaldocument{\econtexRoot/SolvingMicroDSOPs}% xr in use
    }%
    \renewcommand\labelprefix{}%
    % Initialize the counters via the labels belonging to the main document:
}



% Get xrefs; only works properly if main file has already been successfully compiled
\onlyinsubfile{\externaldocument{SolvingMicroDSOPs}} 



\begin{document}

\hypertarget{the-usual-theory}{}
\section{The Usual Theory, and a Bit More Notation}\label{sec:the-usual-theory}


For reference and to illustrate our new notation, we will now derive the Euler equation and other standard results for the problem described above.
Since we can write value as of the end of the consumption stage as a function of $\aNrm$:
\begin{equation*}\begin{gathered}\begin{aligned}
  \vEndStg(\aNrm) & \leftassign \vFunc_{\prd_\cntn}(\aNrm) \leftassign \DiscFac \vBegPrdNxt(\aNrm) = \DiscFac \Ex_{\BegPrdNxt}[\PermGroFac_{\prd+1}^{1-\CRRA}\vFunc_{\prd+1}(\overbrace{a (\Rfree / \PermGroFac_{\prd+1})+\tranShkEmp_{\prd+1}}^{\mNrm_{\prd+1}})],
\end{aligned}\end{gathered}\end{equation*}
the first order condition for \eqref{eq:vusual} with respect to $\aNrm$ (given $\mNrm_{\prd}$) is
\begin{equation}\begin{gathered}\begin{aligned}
      \uFunc^{\cNrm}(\mNrm_{\prd}-\aNrm) = \vEndPrd^{\aNrm}(a) & = \Ex_{\BegPrdNxt}[\DiscFac \RNrmByG_{\prd+1}\PermGroFac_{\prd+1}^{1-\CRRA}{\vFunc}^\mNrm_{\prd+1}(\mNrm_{\prd+1})]  \label{eq:upceqEvtp1}
      \\                        & =  \Ex_{\BegPrdNxt}[\DiscFac\Rfree\phantom{._{\prd+1}}\PermGroFac_{\prd+1}^{\phantom{1}-\CRRA}{\vFunc}^{\mNrm}_{\prd+1}(\mNrm_{\prd+1})]
    \end{aligned}\end{gathered}\end{equation}
and because the \handoutC{Envelope} theorem tells us that
\begin{equation}\begin{gathered}\begin{aligned}
      {\vFunc}^{\mNrm}_{\prd}(\mNrm_{\prd})  & =  \Ex_{\BegPrdNxt} [\DiscFac \Rfree \PermGroFac_{\prd+1}^{-\CRRA}{\vFunc}^{\mNrm}_{\prd+1}(\mNrm_{\prd+1})] \label{eq:envelope}
    \end{aligned}\end{gathered}\end{equation}
we can substitute the LHS of \eqref{eq:envelope} for the RHS of
(\ref{eq:upceqEvtp1}) to get
\begin{verbatimwrite}{./Equations/Envelope.tex}
  \begin{equation}\begin{gathered}\begin{aligned}
        \uFunc^{\cNrm}(\cNrm_{\prd})  & = {\vFunc}^{\mNrm}_{\prd}(\mNrm_{\prd})\label{eq:upcteqvtp}
      \end{aligned}\end{gathered}\end{equation}
\end{verbatimwrite}
  \begin{equation}\begin{gathered}\begin{aligned}
        \uFunc^{c}(c_{\prd})  & =\vFunc^{m}_{\MidStp}(m_{\prd})\label{eq:upcteqvtp}
      \end{aligned}\end{gathered}\end{equation}
\unskip
and rolling forward one {\interval},
\begin{equation}\begin{gathered}\begin{aligned}
      \uFunc^{\cNrm}(\cNrm_{\prd+1})  & = \vFunc^{\mNrm}_{\prd+1}({a}_{\prd}\RNrmByG_{\prd+1}+\tranShkEmp_{\prd+1}) \label{eq:upctp1EqVpxtp1}
    \end{aligned}\end{gathered}\end{equation}
so that substituting the LHS in equation (\ref{eq:upceqEvtp1}) finally gives us the Euler equation for consumption:
\begin{verbatimwrite}{./Equations/cEuler.tex}
  \begin{equation}\begin{gathered}\begin{aligned}
        \uFunc^{\cNrm}(\cNrm_{\prd})  & = \ExEndPrd[\DiscFac \Rfree \PermGroFac_{\prd+1}^{-\CRRA}\uFunc^{\cNrm}(\cNrm_{\prd+1})] \label{eq:cEuler}.
      \end{aligned}\end{gathered}\end{equation}
\end{verbatimwrite}
  \begin{equation}\begin{gathered}\begin{aligned}
        \uFunc^{c}(c_{\prd})  & = \ExEndStp[\DiscFac\Rfree \PermGroFac_{\prd+1}^{-\CRRA}\uFunc^{c}(c_{\prd+1})] \label{eq:cEuler}.
      \end{aligned}\end{gathered}\end{equation}
\unskip

We can now restate the problem \eqref{eq:vusual} with our new within-stage notation:
\begin{equation}\begin{gathered}\begin{aligned}
      \vFunc(m) & = \max_{\cNrm} ~~ \uFunc(\cNrm)+ \vEndStg(\mNrm-\cNrm)
\end{aligned}\end{gathered}\end{equation}
whose first order condition with respect to $\cNrm$ is
\begin{equation}\begin{gathered}\begin{aligned}
  \uFunc^{\cNrm}(\cNrm) &= \vEndStg^{\aNrm}(\mNrm-\cNrm)  \label{eq:upEqbetaOp} % \label{eq:FOCnew} 
\end{aligned}\end{gathered}\end{equation}
which is mathematically equivalent to the usual Euler equation for consumption.

We will revert to this formulation when we reach section~\ref{subsec:egm}.

\begin{comment}
  \subsection{Implementation in Python}

  The code implementing the tasks outlined each of the sections to come is available in the \texttt{\href{https://econ-ark.org/materials/SolvingMicroDSOPs}{SolvingMicroDSOPs}} jupyter notebook, written in \href{https://python.org}{Python}. The notebook imports various modules, including the standard \texttt{numpy} and \texttt{scipy} modules used for numerical methods in Python, as well as some user-defined modules designed to provide numerical solutions to the consumer's problem from the previous section. Before delving into the computational exercise, it is essential to touch on the practicality of these custom modules.

  \subsubsection{Useful auxilliary files}

  In this exercise, two primary user-defined modules are frequently imported and utilized. The first is the \texttt{gothic\_class} module, which contains functions describing the end-of-period value functions found in equations \eqref{eq:vBegStg} - \eqref{eq:EndPrd} (and the corresponding first and second derivatives). %The advantage of defining functions in the code which decompose the consumer's optimal behavior in a given period will become evident in section \ref{subsec:transformation}

  The \texttt{resources} module is also used repeatedly throughout the notebook. This file has three primary objectives: (i) providing functions that discretize the continuous distributions from the theoretical model that describe the uncertainty a consumer faces, (ii) defining the utility function over consumption under a number of specifications, and (iii) enhancing the grid of end-of-period assets for which functions (such as those from the \texttt{gothic\_class} module) will be defined. These objectives will be discussed in greater detail and with respect to the numerical methods used to the problem in subsequent sections of this document.
\end{comment}

\end{document}


% Local Variables:
% eval: (setq prettify-symbols-unprettify-at-point 'right-edge)
% End:
% coding: utf-8
