  We similarly define $\hEndMin_{\EndStp}$ as `minimal human wealth,' the
  present discounted value of labor income if the shocks were to take on
  their worst possible value in every future period \PermShkOn
  {$\TranShkEmp_{t+n} = \TranShkEmpMin ~\forall~n>0$ and $\PermShk_{t+n} =
    \PermShkMin ~\forall~n>0$} {$\TranShkEmp_{t+n} = \TranShkEmpMin
    ~\forall~n>0$} (which we define as corresponding to the beliefs of a
  `pessimist').

  \ctw{}{We will call a `realist' the consumer who correctly perceives the true
    probabilities of the future risks and optimizes accordingly.}

  A first useful point is that, for the realist, a lower bound for the
  level of market resources is $\ushort{m}_{\prd} = -\hEndMin_{\EndStp}$, because
  if $m_{\prd}$ equalled this value then there would be a positive finite
  chance (however small) of receiving \PermShkOn
  {$\TranShkEmp_{t+n}=\TranShkEmpMin$ and $\PermShk_{t+n}=\PermShkMin$}
  {$\TranShkEmp_{t+n}=\TranShkEmpMin$}
  in
  every future period, which would require the consumer to set $c_{\prd}$
  to zero in order to guarantee that the intertemporal budget constraint
  holds\ctw{.}{~(this is the multiperiod generalization of the discussion in
    section \ref{subsec:LiqConstrSelfImposed} explaining the derivation of the `natural borrowing constraint' for period $\trmT-1$,
    $\ushort{a}_{\prd-1}$).}  Since consumption of zero yields negative
  infinite utility, the solution to realist consumer's problem is not well
  defined for values of $m_{\prd} < \ushort{m}_{\prd}$, and the limiting
  value of the realist's $c_t$ is zero as $m_{\prd} \downarrow \ushort{m}_{\prd}$.

  Given this result, it will be convenient to define `excess' market
  resources as the amount by which actual resources exceed the lower
  bound, and `excess' human wealth as the amount by which mean expected human wealth
  exceeds guaranteed minimum human wealth:
  \begin{equation*}\begin{gathered}\begin{aligned}
        \aboveMin \mNrm_{\prd}  & = m_{\prd}+\overbrace{\hEndMin_{\EndStp}}^{=-\ushort{m}_{\prd}}
        \\  \aboveMin \hNrm_{\EndStp}  & = \hNrm_{\EndStp}-\hEndMin_{\EndStp}.
      \end{aligned}\end{gathered}\end{equation*}

  We can now transparently define the optimal
  consumption rules for the two perfect foresight problems, those of the
  `optimist' and the `pessimist.'  The `pessimist' perceives human
  wealth to be equal to its minimum feasible value $\hEndMin_{\EndStp}$ with certainty, so
  consumption is given by the perfect foresight solution
  \begin{equation*}\begin{gathered}\begin{aligned}
        \cFuncBelow_{\prd}(m_{\prd})  & = (m_{\prd}+\hEndMin_{\EndStp})\MPCmin_{\prd}
        \\  & = \aboveMin \mNrm_{\prd}\MPCmin_{\prd}
        .
      \end{aligned}\end{gathered}\end{equation*}

  The `optimist,' on the other hand, pretends that there is no uncertainty
  about future income, and therefore consumes
  \begin{equation*}\begin{gathered}\begin{aligned}
        \cFuncAbove_{\prd}(m_{\prd})  & = (m_{\prd} +\hEndMin_{\EndStp} - \hEndMin_{\EndStp} + \hNrm_{\EndStp} )\MPCmin_{\prd}
        \\    & = (\aboveMin \mNrm_{\prd} + \aboveMin \hNrm_{\EndStp})\MPCmin_{\prd}
        \\      & = \cFuncBelow_{\prd}(m_{\prd})+\aboveMin \hNrm_{\EndStp} \MPCmin_{\prd}
        .
      \end{aligned}\end{gathered}\end{equation*}

  It seems obvious that the spending of the realist will be strictly greater
  than that of the pessimist and strictly less than that of the
  optimist.  Figure~\ref{fig:IntExpFOCInvPesReaOptNeedHiPlot} illustrates the proposition for the consumption rule in period $\trmT-1$.
