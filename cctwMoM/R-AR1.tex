The more realistic case where the interest factor has some serial correlation is more complex.  We consider
the simplest case that captures the main features of empirical interest rate dynamics: An AR(1) process.  Thus
the specification is
\begin{eqnarray}
  \risky_{t+1}-\risky & = & (\risky_{t}-\risky) \gamma + \epsilon_{t+1}
\end{eqnarray}
where $\risky$ is the long-run mean log interest factor, $0 < \gamma < 1$ is the AR(1) serial correlation
coefficient, and $\epsilon_{t+1}$ is the stochastic shock.

The consumer's problem in this case now has two state variables, $\mRat_{t}$ and $\risky_{t}$, and
is described by
\begin{eqnarray}
        \vFunc_{t}(m_{t},\risky_{t}) & = & \max_{{c}_{t}} ~ \util(c_{t})+
        \Ex_{t}[{\Discount}_{t+1}\PGro_{t+1}^{1-\CRRA}\vFunc_{t+1}(m_{t+1},\risky_{t+1})] \label{vtNormRisky}
\\         & \text{s.t.} &   \nonumber \\
    a_{t}   & = & m_{t}-c_{t} \nonumber
\\      \risky_{t+1}-\risky & = & (\risky_{t}-\risky)\gamma + \epsilon_{t+1} \notag
\\      \Risky_{t+1} & = & \exp(\risky_{t+1}) \notag
\\      m_{t+1} & = & \underbrace{\left(\Risky_{t+1}/\PGro_{t+1}\right)}_{\equiv \Rprod_{t+1}}a_{t}+\tShkEmp_{t+1} \nonumber.
\end{eqnarray}

% Kiichi: I will need you to read the literature and figure out how exactly we want to choose the Markov points and transition probabilities.
% When done, you will fill in the [how] text below.

We approximate the AR(1) process by a Markov transition matrix using standard techniques.  The stochastic interest factor is allowed to take
on 11 values centered around the steady-state value $\risky$ and chosen [how?].  Given this Markov transition matrix,
{\it conditional} on the Markov AR(1) state the consumption functions for the `optimist' and the `pessimist' will still be linear,
with identical MPC's that are computed numerically.  Given these MPC's, the (conditional) realist's consumption function can be computed for each Markov state, and the converged consumption rules constitute the solution contingent on the dynamics of the stochastic
interest rate process.

In principle, this refinement should be combined with the previous one;
further exposition of this combination is omitted here because no new
insights spring from the combination of the two techniques.

